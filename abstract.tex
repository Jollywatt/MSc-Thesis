\chapter*{Abstract}
% \addcontentsline{toc}{section}{Abstract}

This thesis is a study of geometric algebra and its application of relativistic physics.
Geometric algebra (or real Clifford algebra) serves as an efficient language for describing rotations in vector spaces of arbitrary metric signature, including Lorentzian spacetime.
By adopting the rotor formalism of geometric algebra, we derive an explicit \x{BCH} formula for composing Lorentz transformations in terms of their generators --- much more easily than with traditional matrix representations.
This is used to straightforwardly derive the composition law for Lorentz boosts and the concomitant Wigner angle.
Later, we include a gentle introduction to differential geometry, noting how the Lie derivative and covariant derivative assume compact forms when expressed with geometric algebra.
Curvature is studied as an obstruction to the integrability of the parallel transport equations, and we present a surface-ordered Stokes' theorem relating the `enclosed curvature' in a surface to the holonomy around its boundary.