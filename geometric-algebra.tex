\chapter{The Geometric Algebra}
\label{cha:geometric-algebra}

In \cref{cha:preliminary-theory}, we defined the metric-independent exterior algebra of multivectors over a vector space $V$.
Metrical operations can be achieved by introducing the Hodge dual (of \cref{sec:metrical-exterior-alg}), tacked onto $\EA{V}$.
The geometric algebra is a generalisation of $\EA{V}$ which has the metric (and its concomitant notions of orientation and duality) built-in.

Informally put, the geometric algebra is obtained by enforcing the single rule
\begin{align}
	\label{eqn:square-is-innerprod}
	𝒖^2 = \ip{𝒖, 𝒖}
\end{align}
for any vector $𝒖$, along with the associative algebra axioms of \cref{def:associative-algebra}.
The rich algebraic structure which follows from this is remarkable.
Formally, we may give the geometric algebra as a quotient, just like our presentation of $\EA{V}$.
\begin{definition}
	Let $V$ be a finite-dimensional real vector space with metric.
	The \textdef{geometric algebra} over $V$ is
	\begin{align}
		\GA(V, η) ≔ \quot{\TA{V}}{\gen{𝒖⊗𝒖 - \ip{𝒖, 𝒖}}}
	.\end{align}
\end{definition}
The ideal defines the congruence generated by $𝒖⊗𝒖 \sim \ip{𝒖, 𝒖}$ for any vector $𝒖 ∈ V$, encoding \cref{eqn:square-is-innerprod}.
This uniquely defines the associative (but not generally commutative) \emph{geometric product} which we denote by juxtaposition.

As $2^n$-dimensional vector spaces, $\GA(V, η)$ and $\EA{V}$ are isomorphic, each with a $\binom{n}{k}$-dimensional subspace for each grade $k$.
If the inner product is completely degenerate (i.e., $\ip{𝒖, 𝒗}_0 = 0$ for all vectors), then the algebras coincide, $\GA(V, 0) \cong \EA{V}$, showing that geometric algebra is more general.
A qualitative difference between $\GA(V, η)$ and $\EA{V}$ is that, while inhomogeneous multivectors find little use in exterior algebra,\sidenote{
	In fact, some authors \cite{flanders1963differential} leave sums of terms of differing grade undefined.
} such elements have the significant geometrical role as reflections and rotations in $\GA(V, η)$.

Note that $\GA(V, η)$ is not $\ZZ$-graded, since it is a quotient by \emph{inhomogeneous} elements $𝒖 ⊗ 𝒖 - \ip{𝒖, 𝒖} ∈ \TA[2]{V} ⊕ \TA[0]{V}$; therefore the geometric product of a $p$-vector and a $q$-vector is not generally a $(p + q)$-vector.
However, the congruence \emph{is} homogeneous with respect to the $\ZZ_2$-grading, so $\GA(V, η)$ is $\ZZ_2$-graded.
This shows that the algebra $\GA(V, η) = \GA[+] ⊕ \GA[-]$ separates into `even' and `odd' subspaces, and that $\GA[+]$ is closed under the geometric product and forms the \emph{even subalgebra}.




\section{The Geometric Product of Vectors}

By expanding $(𝒖 + 𝒗)^2 = ⟨𝒖 + 𝒗, 𝒖 + 𝒗⟩$, it follows\sidenote{
	$𝒖^2 + 𝒗𝒖 + 𝒖𝒗 + 𝒗^2 = ⟨𝒖,𝒖⟩ + 2⟨𝒖,𝒗⟩ + ⟨𝒗,𝒗⟩$
} that
\begin{align}
	⟨𝒖, 𝒗⟩ = \frac12(𝒖𝒗 + 𝒗𝒖)
.\end{align}
We recognise this as the symmetrised product of two vectors.
The remaining antisymmetric part coincides with the \emph{alternating} or \emph{wedge} product familiar from exterior algebra
\begin{align}
	𝒖 ∧ 𝒗 = \frac12(𝒖𝒗 - 𝒗𝒖)
.\end{align}
This is a $2$-vector, or bivector, in $\GA[2](V, η)$.
Thus, the geometric product on vectors is
\begin{align}
	\label{eqn:geometric-prod-of-vectors}
	𝒖𝒗 = ⟨𝒖, 𝒗⟩ + 𝒖∧𝒗
,\end{align}
and some important features are immediate:
\begin{itemize}
	\item \emph{Parallel vectors commute, and vice versa.}
	If $𝒖 = λ𝒗$, then $𝒖∧𝒗 = 0$ and $𝒖𝒗 = ⟨𝒖,𝒗⟩ = ⟨𝒗,𝒖⟩ = 𝒗𝒖$.
	\item \emph{Orthogonal vectors anti-commute, and vice versa.}
	If $⟨𝒖,𝒗⟩ = 0$, then $𝒖𝒗 = 𝒖∧𝒗 = -𝒗∧𝒖 = -𝒗𝒖$.

	\item \emph{Vectors are invertible under the geometric product}.
	If $𝒖$ is a vector for which the scalar $𝒖^2$ is non-zero, then $𝒖^{-1} = 𝒖/𝒖^2$.

	\item \emph{Geometric multiplication produces objects of mixed grade.}
	The product $𝒖𝒗$ has a scalar part $⟨𝒖,𝒗⟩$ and a bivector part $𝒖∧𝒗$.

\end{itemize}