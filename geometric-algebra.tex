\chapter{The Geometric Algebra}
\label{cha:geometric-algebra}

In \cref{cha:preliminary-theory}, we defined the metric-independent exterior algebra of multivectors over a vector space $V$.
Metrical operations can be achieved by introducing the Hodge dual (of \cref{sec:metrical-exterior-alg}) and tacking onto $\EA{V}$.
The geometric algebra is a generalisation of $\EA{V}$ which has the metric (and its concomitant notions of orientation and duality) directly built-in.

Geometric algebras are also known as real \emph{Clifford algebras} $Cl(V, q)$ after their first inventor.
Especially in mathematics, Clifford algebras are defined in terms of a quadratic form $q$, and the vector space $V$ is usually complex.
However, in physics, where $V$ is taken to be real and a metric $η$ is usually supplied instead of $q$, the name ``geometric algebra'' is preferred.\sidenote{
	The newer name was coined by David Hestenes in the 1970s, who popularised Clifford algebra for physics \cite{hestenes1986ga-unified-language,hestenes1968ga}.
}

\subsubsection{Construction as a quotient algebra}

Informally put, the geometric algebra is obtained by enforcing the single rule
\begin{align}
	\label{eqn:square-is-innerprod}
	𝒖^2 = \ip{𝒖, 𝒖}
\end{align}
for any vector $𝒖$, along with the associative algebra axioms of \cref{def:associative-algebra}.
The rich algebraic structure which follows from this is remarkable.
Formally, we may give the geometric algebra as a quotient, just like our presentation of $\EA{V}$.
\begin{definition}
	Let $V$ be a finite-dimensional real vector space with metric.
	The \textdef{geometric algebra} over $V$ is
	\begin{align}
		\GA(V, η) ≔ \quot{\TA{V}}{\gen{𝒖⊗𝒖 - \ip{𝒖, 𝒖}}}
	.\end{align}
\end{definition}
The ideal defines the congruence generated by $𝒖⊗𝒖 \sim \ip{𝒖, 𝒖}$ for any vector $𝒖 ∈ V$, encoding \cref{eqn:square-is-innerprod}.
This uniquely defines the associative (but not generally commutative) \emph{geometric product} which we denote by juxtaposition.

As $2^n$-dimensional vector spaces, $\GA(V, η)$ and $\EA{V}$ are isomorphic, each with a $\binom{n}{k}$-dimensional subspace for each grade $k$.
Denoting the $k$-grade subspace $\GA[k](V, η)$,
\begin{align}
	\GA(V, η) = \bigoplus_{k = 0}^∞ \GA[k](V, η)
.\end{align}
If the inner product is completely degenerate (i.e., $\ip{𝒖, 𝒗}_0 = 0$ for all vectors), then there is an algebra isomorphism $\GA(V, 0) \cong \EA{V}$ --- so the geometric algebra is more general.
A qualitative difference between $\GA(V, η)$ and $\EA{V}$ is that, while inhomogeneous multivectors find little use in exterior algebra,\sidenote{
	In fact, some authors \cite{flanders1963differential} leave sums of terms of differing grade undefined.
} such elements have the significant geometrical role as reflections and rotations in $\GA(V, η)$.

Note that $\GA(V, η)$ is not $\ZZ$-graded, since it is a quotient by \emph{inhomogeneous} elements $𝒖 ⊗ 𝒖 - \ip{𝒖, 𝒖} ∈ \TA[2]{V} ⊕ \TA[0]{V}$; therefore the geometric product of a $p$-vector and a $q$-vector is not generally a $(p + q)$-vector.
However, the congruence \emph{is} homogeneous with respect to the $\ZZ_2$-grading, so $\GA(V, η)$ is $\ZZ_2$-graded.
This shows that the algebra separates into `even' and `odd' subspaces
\begin{align}
	\GA(V, η) = \GA[+](V, η) ⊕ \GA[-](V, η)
	\qqtext{where}
	\begin{cases}
		\GA[+](V, η) = \bigoplus_{k = 0}^∞ \GA[2k](V, η)
	\\	\GA[+](V, η) = \bigoplus_{k = 0}^∞ \GA[2k + 1](V, η)	
	\end{cases}
\end{align}
and also implies that $\GA[+](V, η)$ is closed under the geometric product, forming the \textdef{even subalgebra}.




\subsubsection{The geometric product of vectors}

By expanding $(𝒖 + 𝒗)^2 = ⟨𝒖 + 𝒗, 𝒖 + 𝒗⟩$, it follows\sidenote{
	$𝒖^2 + 𝒗𝒖 + 𝒖𝒗 + 𝒗^2 = ⟨𝒖,𝒖⟩ + 2⟨𝒖,𝒗⟩ + ⟨𝒗,𝒗⟩$
} that
\begin{align}
	⟨𝒖, 𝒗⟩ = \frac12(𝒖𝒗 + 𝒗𝒖)
.\end{align}
We recognise this as the symmetrised product of two vectors.
The remaining antisymmetric part coincides with the \emph{alternating} or \emph{wedge} product familiar from exterior algebra
\begin{align}
	𝒖 ∧ 𝒗 = \frac12(𝒖𝒗 - 𝒗𝒖)
.\end{align}
This is a $2$-vector, or bivector, in $\GA[2](V, η)$.
Thus, the geometric product on vectors is
\begin{align}
	\label{eqn:geometric-prod-of-vectors}
	𝒖𝒗 = ⟨𝒖, 𝒗⟩ + 𝒖∧𝒗
,\end{align}
and some important features are immediate:
\begin{itemize}
	\item \emph{Parallel vectors commute, and vice versa.}
	If $𝒖 = λ𝒗$, then $𝒖∧𝒗 = 0$ and $𝒖𝒗 = ⟨𝒖,𝒗⟩ = ⟨𝒗,𝒖⟩ = 𝒗𝒖$.
	\item \emph{Orthogonal vectors anti-commute, and vice versa.}
	If $⟨𝒖,𝒗⟩ = 0$, then $𝒖𝒗 = 𝒖∧𝒗 = -𝒗∧𝒖 = -𝒗𝒖$.

	\item \emph{Vectors are invertible under the geometric product}.
	If $𝒖$ is a vector for which the scalar $𝒖^2$ is non-zero, then $𝒖^{-1} = 𝒖/𝒖^2$.

	\item \emph{Geometric multiplication produces objects of mixed grade.}
	The product $𝒖𝒗$ has a scalar part $⟨𝒖,𝒗⟩$ and a bivector part $𝒖∧𝒗$.

\end{itemize}


\subsubsection{Higher--grade elements}
\todo{}


\subsubsection{Fundamental algebra automorphisms}

Operations such complex conjugation $\overline{AB} = \overline{A}\,\overline{B}$ or matrix transposition $(AB)\trans = B\trans A\trans$ are useful because they preserve or reverse multiplication.
Linear functions with this property are called algebra automorphisms or antiautomorphisms, respectively.
The geometric algebra possesses this \paren{anti}automorphism operations.

Isometries of $(V, η)$ are linear functions $f : V → V$ which preserve the metric, so that $\ip{f(𝒖), f(𝒗)} = \ip{𝒖, 𝒗}$ for any $𝒖, 𝒗 ∈ V$.
Vector spaces always possess the involution isometry $ι(𝒖) = -𝒖$, as well as the trivial isometry.
An isometry extends uniquely to an algebra \paren{anti}automorphism by defining $f(AB) = f(A)f(B)$ or $f(AB) = f(B)f(A)$.
Thus, by extending the two fundamental isometries of $(V, η)$ in the two possible ways, we obtain four fundamental \paren{anti}automorphisms of $\GA(V, η)$.

\begin{definition}
	\ 
	\begin{itemize}
		\item \textdef{Reversion $\dagger$} is the identity map on vectors $\rev{𝒖} = 𝒖$ extended to general multivectors by the rule $\rev{(AB)} = \rev{B}\rev{A}$.
		
		\item \textdef{Grade involution $ι$} is the extension of the involution $ι(𝒖) = -𝒖$ to general multivectors by the rule $ι(AB) = ι(A)ι(B)$.
	\end{itemize}
\end{definition}
If $A ∈ \GA[k](V, η)$ is a $k$-vector, then $ι(A) = (-1)^k A$ and $\rev{A} = \revsign{k}A$ where
\begin{align}
	\label{eqn:reversion-sign}
	\revsign{k} = (-1)^{\frac{(k - 1)k}2}
\end{align}
is the sign of the reverse permutation on $k$ symbols.

Reversion and grade involution together generate the four fundamental automorphisms
\begin{center}
	\renewcommand{\arraystretch}{1.2}
	\begin{tabular}{c|cl}
	$\op{id}$ & $ι$ & automorphisms \\
	\cline{1-2}
	\marginnote{$ι\circ\dagger$ is sometimes referred to as the \textdef{Clifford conjugate}}
	$\dagger$ & $ι\circ\dagger$ & anti-automorphisms
	\end{tabular}
\end{center}
which form a group isomorphic to $\ZZ_2^2$ under composition.



These operations are very useful in practice.
In particular, the following result follows easily from reasoning about grades.
\begin{lemma}
	\label{lem:grades-of-square}
	If $A ∈ \GA[k](V, η)$ is a $k$-vector, then $A^2$ is a $4\NN$-multivector, i.e., a sum of blades of grade $\qty{0, 4, 8, \dots}$ only.
\end{lemma}
\begin{proof}
	The multivector $A^2$ is its own reverse, since $\rev{(A^2)} = (\rev{A})^2 = (±A)^2 = A^2$, and hence has parts of grade $\set{4n, 4n + 1 | n ∈ \NN}$.
	Similarly, $A^2$ is self-involutive, since $ι(A^2) = ι(A)^2 = (±A)^2 = A^2$.
	It is thus of even grade, leaving the possible grades $\set{0, 4, 8, ...}$.
\end{proof}




\section{Relations with Other Algebras}

An efficient way to become familiar with geometric algebras is to study their relations to other common algebras encountered in physics.

\todo{Include $\CC$, $\HH$ and exterior forms!}




\section{Rotors and Associated Lie Groups}
\label{sec:rotors}



There is a consistent pattern in the algebra isomorphisms listed above.
Note how the complex numbers $\CC$ are fit for describing $\SO(2)$ rotations in the plane, and the quaternions $\HH$ describe $\SO(3)$ rotations in $\RR^3$.
Common to both the respective isomorphisms with $\GA[+](2)$ and $\GA[+](3)$ is the identification of each ``imaginary unit'' in $\CC$ or $\HH$ with a \emph{unit bivector} in $\GA(n)$.
\begin{itemize}
	\item In $2$d, there is one linearly independent bivector, $𝒆_1𝒆_2$, and one imaginary unit, $i$.
	\item In $3$d, there are $\dim \GA[2](3) = \binom{3}{2} = 3$ such bivectors, and so three imaginary units $\set{\ii, \jj, \kk}$ are needed.
	\item In $(1+3)$d, we have $\dim \GA[2](1,3) = \binom{4}{2} = 6$, corresponding to three `spacelike' $\set{\ii, \jj, \kk}$ and three `timelike' $\set{i\ii, i\jj, i\kk}$ units of $\CC ⊗ \HH$.
\end{itemize}
The interpretation of a bivector is clear: it generates a rotation through the oriented plane which it spans.

To see how bivectors act as rotations, observe that rotations in the $ℂ$-plane may be described as mappings
\begin{math}
	z ↦ e^{θi}z
,\end{math}
while $\RR^3$ rotations are described in $\HH$ using a double-sided transformation law,
\begin{math}
	% u → e^{\fracθ2\hat n}ue^{-\fracθ2\hat n}
	u ↦ e^{θ\hat n/2}ue^{-θ\hat n/2}
,\end{math}
where $\hat n ∈ \spanof{\ii, \jj, \kk}$ is a unit quaternion defining the plane of rotation.
Due to the commutativity of $\CC$, the double-sided transformation law is actually general to both $\CC$ and $\HH$.

Similarly, rotations in a geometric algebra are described as
\begin{align}
	\label{eqn:rotor-application}
	% u &↦ \rev{R}uR
	u &↦ e^{-θ\hat{b}/2}ue^{θ\hat{b}/2}
,\end{align}
where $\hat{b} ∈ \GA[2](V, η)$ is a unit bivector.
Multivectors of the form $R = e^σ$ for $σ ∈ \GA[2](V, η)$ are called \emph{rotors}.
Immediate advantages to geometric algebra's rotor formalism are clear:
\begin{itemize}
	\item \emph{It is general to $n$ dimensions, and to any metric signature.}
	Rotors describe generalised rotations,\sidenote{a.k.a., proper orthogonal transformations} depending on the metric and algebraic properties of the generating unitbivector $σ$.
	If $σ^2 < 0$, then $e^σ$ describes a Euclidean rotation; if $σ^2 > 0$, then $e^σ$ is a hyperbolic rotation or \emph{Lorentz boost}.

	\item \emph{Vectors are distinguished from bivectors.}
	One of the subtler points about quaternions is their transformation properties under reflection.
	A quaternion `vector' $v = x\ii + y\jj + z\kk$ reflects through the origin as $v \mapsto -v$, but a quaternion `rotor' of the same value is invariant --- vectors and pseudovectors are confused with the same kind of object.
	Not so in the geometric algebra: vectors are vectors in $\GA[1]$, and $\RR^3$ pseudovectors are bivectors in $\GA[2]$.
	The price to pay for the introduction of more objects is not a price but a benefit: the generalisation to arbitrary dimensions is immediate and the geometric role of objects becomes clear.
	\sidenote{
		See \cite{lasenby2016ga-unified-language,hestenes1986ga-unified-language,chappell2016quat-history} for similarly impassioned testaments to the elegance of geometric algebra.
	}
\end{itemize}



\subsubsection{The rotor groups}

We will now see more rigorously how the rotor formalism arises.
An orthogonal transformation in $n$ dimensions may be achieved by the composition of at most $n$ reflections.\sidenote{This is the Cartan--Dieudonné theorem \cite{cartan-dieudonne-theorem}.}
A reflection may be described in the geometric algebra by conjugation with an invertible vector.
For instance, the linear map
\begin{align}
	A \mapsto -𝒗A𝒗^{-1}
	\label{eqn:multivector-reflection}
\end{align}
reflects the multivector $A$ along the vector $𝒗$, that is, across the hyperplane with normal $𝒗$.
By composing reflections of this form, any orthogonal transformation may be built, acting on multivectors as
\begin{align}
	\label{eqn:multivector-conjugation-by-inverse}
	A \mapsto \pm RAR^{-1}
\end{align}
for some $R = 𝒗_1𝒗_2\cdots 𝒗_3$, where the sign is positive for an even number of reflections, and negative for odd.

Scaling the axis of reflection $𝒗$ by a non-zero scalar $λ$ does not affect the reflection map \eqref{eqn:multivector-reflection}, since $𝒗 ↦ λ𝒗$ is cancelled out by $𝒗^{-1} ↦ λ^{-1}𝒗^{-1}$.
Therefore, a more direct correspondence exists between reflections and normalised vectors $\hat{𝒗}^2 = ±1$ (although there still remains an overall ambiguity in sign).
For an orthogonal transformation built using normalised vectors,
\begin{align}
	R^{-1} = \hat{𝒗}_3^{-1}\cdots \hat{𝒗}_2^{-1}\hat{𝒗}_1^{-1} = ±\rev{R}	
\end{align}
since $\hat{𝒗}^{-1} = ±\hat{𝒗}$, and hence \cref{eqn:multivector-conjugation-by-inverse} may be written with the reversion instead of inversion:
\begin{align}
	\label{eqn:multivector-conjugation}
	A ↦ \pm RA\rev{R}
\end{align}


All such elements $R^{-1} = ±R^†$ taken together form a group under the geometric product.
This is called the \emph{pin} group:
\begin{align}
	\op{Pin}(p, q) ≔ \qty{R \in \GA(p, q) \mid RR^† = ±1}
\end{align}
There are two ``pinors'' for each orthogonal transformation, since $+R$ and $-R$ give the same map \eqref{eqn:multivector-conjugation}.
Thus, the pin group forms a double cover of the orthogonal group $\operatorname{O}(p,q)$.

Furthermore, the even-grade elements of $\op{Pin}(p,q)$ form a subgroup, called the \emph{spin} group:
\begin{align}
	\op{Spin}(p, q) ≔ \qty{R \in \GA[+](p, q) \mid RR^† = ±1}
\end{align}
This forms a double cover of $\SO(p, q)$.

Finally, the additional requirement that $RR^† = 1$ defines the restricted spinor group, or the \emph{rotor} group:
\begin{align}
	\op{Spin}^+(p, q) ≔ \qty{R \in \GA[+](p, q) \mid RR^† = 1}
\end{align}
%
\begin{marginfigure}
\begin{tikzcd}[column sep=tiny, row sep=small]
	\op{Spin}^+ & \op{Spin} & \op{Pin} \\
	\SO^+ & \SO & \operatorname{O}
	\arrow["⊆"{description}, draw=none, from=1-1, to=1-2]
	\arrow["⊂"{description}, draw=none, from=1-2, to=1-3]
	\arrow["⊆"{description}, draw=none, from=2-1, to=2-2]
	\arrow["⊂"{description}, draw=none, from=2-2, to=2-3]
	\arrow[two heads, from=1-1, to=2-1]
	\arrow[two heads, from=1-2, to=2-2]
	\arrow[two heads, from=1-3, to=2-3]
\end{tikzcd}
\caption{Relationships between Lie groups associated with a geometric algebra. An arrow $a \surject b$ signifies that $a$ is a double-cover of $b$.}
\end{marginfigure}
%
The rotor group is a double cover of the restricted special orthogonal group $\SO^+(p, q)$.
Except for the degenerate case of $\op{Spin}^+(1, 1)$, the rotor group is simply connected to the identity.


\subsubsection{The bivector subalgebra}

The multivector commutator product
\begin{align}
	\label{eqn:commutator-prod}
	A × B ≔ \frac12(AB - BA)
\end{align}
forms a Lie bracket on the space of bivectors $\GA[2]$.
\begin{proof}
	The commutator product $A \mapsto A × σ$ with a bivector $σ$ is a grade-preserving operation.
	If $A = \grade[k]{A}$ then $Aσ$ and $σA$ are $\set{k - 2, k, k + 2}$-multivectors.
	The $k ± 2$ parts are
	\begin{align}
		\grade[k ± 2]{Α×σ} = \frac12\qty(\grade[k±2]{Aσ} - \grade[k±2]{σA})
	.\end{align}
	However,
	\begin{math}
		\grade[k±2]{σA} = \revsign{k ± 2}\grade[k ± 2]{\rev{A}\rev{σ}} = -\revsign{k ± 2}\revsign{k}\grade[k ± 2]{Aσ}
	\end{math}
	and the reversion signs\sidenote{
		Recall from \cref{eqn:reversion-sign} that $\rev{A} = \revsign{k}A$ for a $k$-vector where
		\begin{math}
			\revsign{k} = (-1)^{\frac{(k - 1)k}2}
		.\end{math}
	} satisfy $s_{k±2}s_k = -1$ for any $k$.
	Hence, $\grade[k±2]{A×σ} = 0$, leaving only the grade $k$ part, $A × σ = \grade[k]{A × σ}$.
	Clearly \cref{eqn:commutator-prod} is bilinear and satisfies the Jacobi identity, so $(\GA[2], ×)$ is closed and forms a Lie algebra.
\end{proof}

Because the even subalgebra $\GA[+] \supset \GA[2]$ is closed under the geometric product The exponential of a bivector
\begin{math}
	e^σ = 1 + σ + \frac12 σ^2 + \cdots
\end{math}
is an even multivector.