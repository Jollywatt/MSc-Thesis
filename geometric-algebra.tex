\chapter{The Geometric Algebra}
\label{cha:geometric-algebra}

In \cref{cha:preliminary-theory}, we defined the metric-independent exterior algebra of multivectors over a vector space $V$.
Metrical operations can be achieved by introducing the Hodge dual (of \cref{sec:metrical-exterior-alg}), tacked onto $\EA{V}$.
The geometric algebra is a generalisation of $\EA{V}$ which has the metric (and its concomitant notions of orientation and duality) built-in.

Informally put, the geometric algebra is obtained by enforcing the single rule
\begin{align}
	\label{eqn:square-is-innerprod}
	𝒖^2 = \ip{𝒖, 𝒖}
\end{align}
for any vector $𝒖$, along with the associative algebra axioms of \cref{def:associative-algebra}.
The rich algebraic structure which follows from this is remarkable.
Formally, we may give the geometric algebra as a quotient, just like our presentation of $\EA{V}$.
\begin{definition}
	Let $V$ be a finite-dimensional real vector space with metric.
	The \textdef{geometric algebra} over $V$ is
	\begin{align}
		\GA(V, η) ≔ \quot{\TA{V}}{\gen{𝒖⊗𝒖 - \ip{𝒖, 𝒖}}}
	.\end{align}
\end{definition}
The ideal defines the congruence generated by $𝒖⊗𝒖 \sim \ip{𝒖, 𝒖}$ for any vector $𝒖 ∈ V$, encoding \cref{eqn:square-is-innerprod}.
This uniquely defines the associative (but not generally commutative) \emph{geometric product} which we denote by juxtaposition.

As $2^n$-dimensional vector spaces, $\GA(V, η)$ and $\EA{V}$ are isomorphic, each with a $\binom{n}{k}$-dimensional subspace for each grade $k$.
Denoting the $k$-grade subspace $\GA[k](V, η)$,
\begin{align}
	\GA(V, η) = \bigoplus_{k = 0}^∞ \GA[k](V, η)
.\end{align}
If the inner product is completely degenerate (i.e., $\ip{𝒖, 𝒗}_0 = 0$ for all vectors), then there is an algebra isomorphism $\GA(V, 0) \cong \EA{V}$ --- so the geometric algebra is more general.
A qualitative difference between $\GA(V, η)$ and $\EA{V}$ is that, while inhomogeneous multivectors find little use in exterior algebra,\sidenote{
	In fact, some authors \cite{flanders1963differential} leave sums of terms of differing grade undefined.
} such elements have the significant geometrical role as reflections and rotations in $\GA(V, η)$.

Note that $\GA(V, η)$ is not $\ZZ$-graded, since it is a quotient by \emph{inhomogeneous} elements $𝒖 ⊗ 𝒖 - \ip{𝒖, 𝒖} ∈ \TA[2]{V} ⊕ \TA[0]{V}$; therefore the geometric product of a $p$-vector and a $q$-vector is not generally a $(p + q)$-vector.
However, the congruence \emph{is} homogeneous with respect to the $\ZZ_2$-grading, so $\GA(V, η)$ is $\ZZ_2$-graded.
This shows that the algebra separates into `even' and `odd' subspaces
\begin{align}
	\GA(V, η) = \GA[+](V, η) ⊕ \GA[-](V, η)
	\qqtext{where}
	\begin{cases}
		\GA[+](V, η) = \bigoplus_{k = 0}^∞ \GA[2k](V, η)
	\\	\GA[+](V, η) = \bigoplus_{k = 0}^∞ \GA[2k + 1](V, η)	
	\end{cases}
\end{align}
and also implies that $\GA[+](V, η)$ is closed under the geometric product, forming the \textdef{even subalgebra}.




\subsubsection{The Geometric Product of Vectors}

By expanding $(𝒖 + 𝒗)^2 = ⟨𝒖 + 𝒗, 𝒖 + 𝒗⟩$, it follows\sidenote{
	$𝒖^2 + 𝒗𝒖 + 𝒖𝒗 + 𝒗^2 = ⟨𝒖,𝒖⟩ + 2⟨𝒖,𝒗⟩ + ⟨𝒗,𝒗⟩$
} that
\begin{align}
	⟨𝒖, 𝒗⟩ = \frac12(𝒖𝒗 + 𝒗𝒖)
.\end{align}
We recognise this as the symmetrised product of two vectors.
The remaining antisymmetric part coincides with the \emph{alternating} or \emph{wedge} product familiar from exterior algebra
\begin{align}
	𝒖 ∧ 𝒗 = \frac12(𝒖𝒗 - 𝒗𝒖)
.\end{align}
This is a $2$-vector, or bivector, in $\GA[2](V, η)$.
Thus, the geometric product on vectors is
\begin{align}
	\label{eqn:geometric-prod-of-vectors}
	𝒖𝒗 = ⟨𝒖, 𝒗⟩ + 𝒖∧𝒗
,\end{align}
and some important features are immediate:
\begin{itemize}
	\item \emph{Parallel vectors commute, and vice versa.}
	If $𝒖 = λ𝒗$, then $𝒖∧𝒗 = 0$ and $𝒖𝒗 = ⟨𝒖,𝒗⟩ = ⟨𝒗,𝒖⟩ = 𝒗𝒖$.
	\item \emph{Orthogonal vectors anti-commute, and vice versa.}
	If $⟨𝒖,𝒗⟩ = 0$, then $𝒖𝒗 = 𝒖∧𝒗 = -𝒗∧𝒖 = -𝒗𝒖$.

	\item \emph{Vectors are invertible under the geometric product}.
	If $𝒖$ is a vector for which the scalar $𝒖^2$ is non-zero, then $𝒖^{-1} = 𝒖/𝒖^2$.

	\item \emph{Geometric multiplication produces objects of mixed grade.}
	The product $𝒖𝒗$ has a scalar part $⟨𝒖,𝒗⟩$ and a bivector part $𝒖∧𝒗$.

\end{itemize}



\subsubsection{Fundamental algebra automorphisms}

Operations such complex conjugation $\overline{AB} = \overline{A}\,\overline{B}$ or matrix transposition $(AB)\trans = B\trans A\trans$ are useful because they preserve or reverse multiplication.
Linear functions with this property are called algebra automorphisms or antiautomorphisms, respectively.
The geometric algebra possesses this \paren{anti}automorphism operations.

Isometries of $(V, η)$ are linear functions $f : V → V$ which preserve the metric, so that $\ip{f(𝒖), f(𝒗)} = \ip{𝒖, 𝒗}$ for any $𝒖, 𝒗 ∈ V$.
Vector spaces always possess the involution isometry $ι(𝒖) = -𝒖$, as well as the trivial isometry.
An isometry extends uniquely to an algebra \paren{anti}automorphism by defining $f(AB) = f(A)f(B)$ or $f(AB) = f(B)f(A)$.
Thus, by extending the two fundamental isometries of $(V, η)$ in the two possible ways, we obtain four fundamental \paren{anti}automorphisms of $\GA(V, η)$.

\begin{definition}
	\ 
	\begin{itemize}
		\item \textdef{Reversion $\dagger$} is the identity map on vectors $\rev{𝒖} = 𝒖$ extended to general multivectors by the rule $\rev{(AB)} = \rev{B}\rev{A}$.
		
		\item \textdef{Grade involution $ι$} is the extension of the involution $ι(𝒖) = -𝒖$ to general multivectors by the rule $ι(AB) = ι(A)ι(B)$.
	\end{itemize}
\end{definition}
If $A ∈ \GA[k](V, η)$ is a $k$-vector, then $ι(A) = (-1)^k A$ and $\rev{A} = \revsign{k}A$ where
\begin{align}
	\label{eqn:reversion-sign}
	\revsign{k} = (-1)^{\frac{(k - 1)k}2}
\end{align}
is the sign of the reverse permutation on $k$ symbols.

Reversion and grade involution together generate the four fundamental automorphisms
\begin{center}
	\renewcommand{\arraystretch}{1.2}
	\begin{tabular}{c|cl}
	$\op{id}$ & $ι$ & automorphisms \\
	\cline{1-2}
	\marginnote{$ι\circ\dagger$ is sometimes referred to as the \textdef{Clifford conjugate}}
	$\dagger$ & $ι\circ\dagger$ & anti-automorphisms
	\end{tabular}
\end{center}
which form a group isomorphic to $\ZZ_2^2$ under composition.



These operations are very useful in practice.
In particular, the following result follows easily from reasoning about grades.
\begin{lemma}
	\label{lem:grades-of-square}
	If $A ∈ \GA[k](V, η)$ is a $k$-vector, then $A^2$ is a $4\NN$-multivector, i.e., a sum of blades of grade $\qty{0, 4, 8, \dots}$ only.
\end{lemma}
\begin{proof}
	The multivector $A^2$ is its own reverse, since $\rev{(A^2)} = (\rev{A})^2 = (±A)^2 = A^2$, and hence has parts of grade $\set{4n, 4n + 1 | n ∈ \NN}$.
	Similarly, $A^2$ is self-involutive, since $ι(A^2) = ι(A)^2 = (±A)^2 = A^2$.
	It is thus of even grade, leaving the possible grades $\set{0, 4, 8, ...}$.
\end{proof}




\section{Relations with Other Algebras}

An efficient way to become familiar with geometric algebras is to study their relations to other common algebras encountered in physics.