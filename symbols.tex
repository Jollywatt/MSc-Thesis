% Set builder notation (usage: `\set{a | P(a)}` or `\set[\big]{1, 2}`)
\usepackage{xparse}
\usepackage{ifthen}

\renewcommand{\op}[1]{{\operatorname{#1}}}

\usepackage{mathtools}
\DeclarePairedDelimiterX{\setdelim}[1]{\{}{\}}{\setargs{#1}}
\DeclarePairedDelimiterX{\dblsetdelim}[1]{\{\!\{}{\}\!\}}{\setargs{#1}}
\NewDocumentCommand{\setargs}{>{\SplitArgument{1}{|}}m}{\setargsaux#1}
\NewDocumentCommand{\setargsaux}{mm}{\IfNoValueTF{#2}{#1}{#1\nonscript\:\delimsize\vert\allowbreak\nonscript\:\mathopen{}#2}}%
\newcommand{\set}[2][*]{\ifthenelse{\equal{\detokenize{#1}}{*}}{\setdelim*{#2}}{\setdelim[#1]{#2}}}
\newcommand{\dblset}[2][*]{\ifthenelse{\equal{\detokenize{#1}}{*}}{\dblsetdelim*{#2}}{\setdelim[#1]{#2}}}


% quaternions
\newcommand{\ii}{\vb{\skew{1}\hat{\clipbox{0pt 0pt 0pt 2.7pt}{$i$}} }}
\newcommand{\jj}{\vb{\skew{3}\hat{\clipbox{0pt 0pt 0pt 2.7pt}{$j$}} }}
\newcommand{\kk}{\vb{\hat k}}


% basis vectors
\renewcommand{\vb}[1]{\symbfit{#1}}
\newcommand{\ve}{\vb{e}}
\newcommand{\vg}{\vb{γ}}
\newcommand{\vs}{\vec{σ}}
\newcommand{\dx}{\dd x}

\newcommand{\spanof}{\op{span}\set}

\renewcommand{\ip}[1]{\left⟨ #1 \right⟩}

% special structures
\newcommand{\manif}{\mathcal}	% manifold
\newcommand{\cat}{\mathbf}	% category
\newcommand{\liealg}{\mathfrak}	% Lie algebra
\newcommand{\lin}{\mathrm}	% linear map
\newcommand{\rotor}{\mathscr}	% geometric algebra rotors

% fields
\usepackage{dsfont}
\newcommand{\FF}{\mathds{F}}
\newcommand{\RR}{\mathds{R}}
\newcommand{\CC}{\mathds{C}}
\newcommand{\HH}{\mathds{H}}
\newcommand{\ZZ}{\mathds{Z}}
\newcommand{\NN}{\mathds{N}}

% groups
\DeclareMathOperator{\SO}{SO}
\DeclareMathOperator{\GL}{GL}

% quotient structure
\newcommand{\quot}[2]{\left. #1 \middle/ #2 \right.}

% generator
\newcommand{\gen}{\dblset}

% transpose superscript
\newcommand{\trans}{^{\mkern-1.5mu\mathsf{T}}}



% algebras
\newcommand{\TA}[2][]{{#2}^{⊗#1}}
\usepackage{scalerel}
\newcommand{\EA}[1][]{\scalerel*{\wedge}{V}^{#1}}
\newcommand{\SA}[1][\,]{𝒮^{\!#1}}
\newcommand{\GA}[1][]{𝒢_{#1}}

\newcommand{\forms}[1][]{\op{Ω}^{#1}}


% shortcuts

% easy dotted sequence: \etc{𝒖_{\i}}{⊗}{k} → 𝒖_1 ⊗ ··· ⊗ 𝒖_k
\newcommand{\etc}[4][1]{
	{\def\i{#1} #2}
	#3 \if,#3 \dots \else \cdots \fi #3
	{\def\i{#4} #2}
}


% geometric algebra
\DeclarePairedDelimiter{\angbr}{⟨}{⟩}
\newcommand{\grade}[2][]{\angbr*{#2}_{#1}}

\newcommand{\vol}{\mathds{I}}

\newcommand{\rev}[1]{#1^\dagger}
\newcommand{\revsign}[1]{s_{#1}}

% "fat dot" product
% https://tex.stackexchange.com/a/235120/105570
\makeatletter
\newcommand*\fatdot{\mathpalette\bigcdot@{.5}}
\newcommand*\bigcdot@[2]{\mathbin{\vcenter{\hbox{\scalebox{#2}{$\m@th#1\bullet$}}}}}
\makeatother






% DIFFERENTIAL GEOMETRY


% topological sphere
\newcommand{\Sphere}{\mathscr{S}}


% tangent bundle, vertical bundle
\DeclareMathOperator{\TT}{T}
\DeclareMathOperator{\VV}{V}

% section of bundle
\DeclareMathOperator{\secs}{Γ}

% injection, surjection, bijection
\newcommand{\inject}{\hookrightarrow} % ↪︎
\newcommand{\surject}{\twoheadrightarrow} % ↠
\usepackage{trimclip}
\newcommand{\biject}{\mathrel{%
\mathrlap{\clipbox*{0 -1ex {0.5\width} {\height}}{$\inject$}}%
\surject}}

% fibre bundle A ↪︎ B ↠ C
\newcommand{\fibrebundle}[4][]{#2 \inject #3 \overset{#1}{\surject} #4}

