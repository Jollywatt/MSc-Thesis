% STRUCTURE AND LAYOUT

\documentclass[
	paper=a4,
	fontsize=13pt,
	open=any,
	headings=twolinechapter,
]{scrbook}


\newcommand{\notewidth}{4.2cm}
\newcommand{\notesep}{5mm}
\usepackage[
	\ifdebug showframe,\fi
	inner=20mm,
	marginparwidth=\notewidth,
	marginparsep=\notesep,
	outer=\dimexpr10mm + \notewidth + \notesep,
	bottom=4cm,
]{geometry}

\usepackage{layout}

\usepackage[strict]{changepage}
\newenvironment{fullwidth}
  {\begin{adjustwidth*}{}{\dimexpr-\marginparwidth-\marginparsep\relax}\vspace{-\baselineskip}}
  {\end{adjustwidth*}}


\usepackage{mparhack} % improved margin note positioning
\usepackage{sidenotes}

	% minimum vertical space between sidenotes
	\setlength\marginparpush{15pt}

	% make sidenote font size smaller
	% https://tex.stackexchange.com/q/361622/105570
	\makeatletter
	\RenewDocumentCommand\sidenotetext{ o o +m }{%	  
		\IfNoValueOrEmptyTF{#1}{%
			\@sidenotes@placemarginal{#2}{\textsuperscript{\thesidenote}{}~\footnotesize#3}%
			\refstepcounter{sidenote}%
		}{%
			\@sidenotes@placemarginal{#2}{\textsuperscript{#1}~#3}%
		}%
	}
	\makeatother

	% make marginnote font size smaller
	\renewcommand*{\marginfont}{\footnotesize}

	% make sidenotes ragged
	% https://tex.stackexchange.com/a/359580/105570
	\makeatletter
	\RenewDocumentCommand \@sidenotes@placemarginal { m m }
	{% <- Important!!!
		\IfNoValueOrEmptyTF{#1}
		{\marginpar[{\raggedleftmarginnote #2}]{\raggedrightmarginnote #2}}
		{\marginnote{#2}[#1]}%
	}
	\makeatother





% DEBUGGING

\usepackage{lipsum}
\usepackage{layout}

\newcommand{\toself}[1]{\iffinal\else\textcolor{blue}{\{\textsc{to self:} #1\}}\fi}
\newcommand{\todo}[1]{\iffinal\else\textcolor{purple}{\{\textsc{to do:} #1\}}\fi}
\newcommand{\urgent}[1]{\iffinal\else\textcolor{red}{\{\textsc{urgent:} #1\}}\fi}



% MATHEMATICS

\usepackage{quiver} % must precede {physics}
\usepackage{centernot}
\usepackage[
	math-style=ISO,
	warnings-off={mathtools-colon,mathtools-overbracket},
]{unicode-math}
\usepackage{physics}
\usepackage{empheq}

% give \displaystyle operators \limits by default
\removenolimits{\int}
\removenolimits{\oint}

\usepackage{mathtools}


\renewcommand{\op}[1]{{\operatorname{#1}}}

\DeclarePairedDelimiter{\ceil}{\lceil}{\rceil}
\DeclarePairedDelimiter{\floor}{\lfloor}{\rfloor}

\DeclareMathOperator{\arctanh}{arctanh}
% trig
\newcommand{\Co}{\mathrm{C}_}
\newcommand{\Si}{\mathrm{S}_}
\newcommand{\Ta}{\mathrm{T}_}



% Set builder notation (usage: `\set{a | P(a)}` or `\set[\big]{1, 2}`)
\usepackage{xparse}
\usepackage{ifthen}
\DeclarePairedDelimiterX{\setdelim}[1]{\{}{\}}{\setargs{#1}}
\NewDocumentCommand{\setargs}{>{\SplitArgument{1}{|}}m}{\setargsaux#1}
\NewDocumentCommand{\setargsaux}{mm}{\IfNoValueTF{#2}{#1}{#1\nonscript\:\delimsize\vert\allowbreak\nonscript\:\mathopen{}#2}}%
\newcommand{\set}[2][*]{\ifthenelse{\equal{\detokenize{#1}}{*}}{\setdelim*{#2}}{\setdelim[#1]{#2}}}


% quaternions
\newcommand{\ii}{\vb{\skew{1}\hat{\clipbox{0pt 0pt 0pt 2.7pt}{$i$}} }}
\newcommand{\jj}{\vb{\skew{3}\hat{\clipbox{0pt 0pt 0pt 2.7pt}{$j$}} }}
\newcommand{\kk}{\vb{\hat k}}


% basis vectors
\renewcommand{\vb}[1]{\symbfit{#1}}
\newcommand{\ve}{\vb{e}}
\newcommand{\vg}{\vb{γ}}
\newcommand{\vs}{\vec{σ}}
\newcommand{\dx}{\dd x}
\newcommand{\∂}{\symbf{∂}}

\newcommand{\spanof}{\op{span}\set}

\renewcommand{\ip}[1]{\left⟨ #1 \right⟩}

% special structures
\newcommand{\manif}{\mathcal}	% manifold
\newcommand{\cat}{\mathbf}	% category
\newcommand{\liealg}{\mathfrak}	% Lie algebra
\newcommand{\lin}{\mathrm}	% linear map
\newcommand{\rotor}{\mathscr}	% geometric algebra rotors

% fields
\usepackage{dsfont}
\newcommand{\FF}{\mathds{F}}
\newcommand{\RR}{\mathds{R}}
\newcommand{\CC}{\mathds{C}}
\newcommand{\HH}{\mathds{H}}
\newcommand{\ZZ}{\mathds{Z}}
\newcommand{\NN}{\mathds{N}}

% groups
\DeclareMathOperator{\SO}{SO}
\DeclareMathOperator{\GL}{GL}

% quotient structure
\newcommand{\quot}[3][*]{
	\if*#1
		\left. #2 \middle/ #3 \right.
	\else
		#2 #1/ #3
	\fi
}

% generator
\newcommand{\gen}[1]{\set{\!\set{#1}\!}}

% transpose superscript
\newcommand{\transpose}{^{\mkern-1.5mu\mathsf{T}}}


% path-ordered exponential
\newcommand{\Pexpl}[1][]{\accentset{←}{\mathds{P}}_{#1}\exp}
\newcommand{\Pexpr}[1][]{\accentset{→}{\mathds{P}}_{#1}\exp}


% algebras
\newcommand{\TA}[2][]{{#2}^{⊗#1}}
\usepackage{scalerel}
\newcommand{\EA}[1][]{\scalerel*{\wedge}{V}^{#1}}
\newcommand{\SA}[1][\,]{𝒮^{\!#1}}
\newcommand{\GA}[1][]{𝒢_{#1}}

\newcommand{\forms}[1][]{\op{Ω}^{#1}}


% shortcuts

% easy dotted sequence: \etc{𝒖_{\i}}{⊗}{k} → 𝒖_1 ⊗ ··· ⊗ 𝒖_k
\newcommand{\etc}[4][1]{
	{\def\i{#1} #2}
	#3 \if,#3 \dots \else \cdots \fi #3
	{\def\i{#4} #2}
}
\newcommand{\etcskip}[5][1]{
	{\def\i{#1} #2}
	#3 \if,#3 \dots \else \cdots \fi #3
	{\def\i{#4} \widehat{#2}}
	#3 \if,#3 \dots \else \cdots \fi #3
	{\def\i{#5} #2}
}
\newcommand{\etcmid}[5][1]{
	{\def\i{#1} #2}
	#4 \if,#4 \dots \else \cdots \fi #4
	{#3}
	#4 \if,#3 \dots \else \cdots \fi #4
	{\def\i{#5} #2}
}


\makeatletter
\newcommand{\sig}[1]{(\@tfor\elem:=#1\do{{\elem}})}
\makeatother




% geometric algebra
\DeclarePairedDelimiter{\angbr}{⟨}{⟩}
\newcommand{\grade}[2][]{\angbr*{#2}_{#1}}

\newcommand{\vol}{\mathds{I}}

\newcommand{\rev}[1]{#1^\dagger}
\newcommand{\revsign}[1]{s_{#1}}

\newcommand{\invol}[1]{#1^\star}

\newcommand{\bch}[2]{#1 \odot #2}

% self-reverse and anti-self-reverse projections
\newcommand{\srev}[1]{\qty{\!\!\qty{#1}\!\!}}
\newcommand{\arev}[1]{\qty[\!\qty[#1]\!]}

% "fat dot" product
% https://tex.stackexchange.com/a/235120/105570
\makeatletter
\newcommand*\fatdot{\mathpalette\bigcdot@{.8}}
\newcommand*\bigcdot@[2]{\mathbin{\vcenter{\hbox{\scalebox{#2}{$\m@th#1\bullet$}}}}}
\makeatother

\newcommand{\wedot}{\mathrel{\ooalign{\hfil$∧$\hfil\cr\hfil$.$\hfil}}}

% left and right contractions
\newcommand{\lcontr}{\mathrel{\rfloor}}
\newcommand{\rcontr}{\mathrel{\lfloor}}






% DIFFERENTIAL GEOMETRY


% topological sphere
\newcommand{\Sphere}{\mathscr{S}}


% tangent bundle, vertical bundle
\DeclareMathOperator{\TT}{T}
\DeclareMathOperator{\VV}{V}

% section of bundle
\DeclareMathOperator{\secs}{Γ}

% injection, surjection, bijection
\newcommand{\inject}{\hookrightarrow} % ↪︎
\newcommand{\surject}{\twoheadrightarrow} % ↠
\usepackage{trimclip}
\newcommand{\biject}{\mathrel{%
\mathrlap{\clipbox*{0 -1ex {0.5\width} {\height}}{$\inject$}}%
\surject}}

% fibre bundle A ↪︎ B ↠ C
\newcommand{\fibrebundle}[4][]{#2 \inject #3 \overset{#1}{\surject} #4}


% transport operator
\newcommand{\trans}{\operatorname{trans}\displaylimits}


% derivatives
\DeclareMathOperator{\DD}{D}
\newcommand{\vd}{\symbf{∂}}
\newcommand{\VD}{\symbfcal{D}}
\newcommand{\bd}{\symbfup{d}}

\newcommand{\lie}{\pounds}

% explicit differential form
\usepackage{accents}
\newcommand{\df}[1]{\underaccent{\sim}{#1}}



\usepackage{amsthm}
	\newtheorem{definition}{Definition}
	\newtheorem{theorem}{Theorem}
	\newtheorem{lemma}{Lemma}
	\newtheorem{corollary}{Corollary}
	\newtheorem{proposition}{Proposition}


% TYPOGRAPHY

\usepackage{hyphenat}

\setmainfont{Libertinus Serif}
% \setsansfont{Libertinus Sans}

\iftrue
\setmathfont{Libertinus Math}
\else
\setmathfont{latinmodern-math.otf}
\setmathfont[range=\varnothing]{Libertinus Math}
\fi

\linespread{1.07}

\newcommand{\textdef}{\textsc}

\usepackage{testhyphens}
\hyphenation{
	auto-morph-ism
	auto-morph-isms
	anti-auto-morph-ism
	anti-auto-morph-isms
	diffeo-morph-ic
}

% correct hyphenation for parenthesised prefixes.
\newcommand{\paren}[1]{(#1\discretionary{-)}{}{)}\nolinebreak\hspace{0pt}}
% Usage: \paren{anti}disestablishment produces hyphenation like:
% (anti)disestablishment (anti-) %
% disestablishment (anti)disest- %
% ablishment...



% shortcuts (usage: define with `\newshortcut{abc}{expansion}` and use with `\x{abc}`.)
\makeatletter
\newcommand\newshortcut[2]{\@namedef{\detokenize{#1}}{#2}}
\newcommand\x[1]{\@nameuse{\detokenize{#1}}}
\makeatother

\newshortcut{BCH full}{Baker--Campbell--Hausdorff--Dynkin}
\newshortcut{BCH}{BCHD}



\usepackage{caption}
\usepackage{subcaption}

\setcapindent{0pt} % remove indentation for multiline (table) captions
\captionsetup{
	font=footnotesize,
}

\DeclareCaptionJustification{raggedauto}{\Ifthispageodd{\raggedright}{\raggedleft}}% <- changed
\DeclareCaptionStyle{marginfigure}{
	font=footnotesize,
	justification=raggedauto,
}
\captionsetup[subfigure]{
	justification=centering,
}


% \captionsetup{labelfont={bf}, font={it,footnotesize}, justification=raggedauto, singlelinecheck=false, position=above}


\usepackage{enumitem}



\renewcommand*\dictumwidth{\linewidth}
\renewcommand*\dictumauthorformat[1]{--- #1}
\renewcommand*\dictumrule{}


% FIGURES

\newcommand{\includefigure}[2][\columnwidth]{
	\graphicspath{{figures/}}
	\def\svgwidth{#1}
	\input{figures/#2.pdf_tex}
}

\usepackage{array}
\newcolumntype{C}{>{$}c<{$}} % math-mode version of "c" column type
\newcolumntype{L}{>{$}l<{$}} % math-mode version of "c" column type


% REFERENCING

\usepackage[numbers,sort&compress]{natbib}
\bibliographystyle{my-thesis-style}
\usepackage{doi}

\usepackage{xcolor}
\usepackage{hyperref}
	\hypersetup{
		colorlinks,
		linkcolor={red!50!black},
		urlcolor={blue!60!black},
		citecolor={green!50!black},
	}

\usepackage{cleveref}

% AUTONUM HACKS %
	% supress dumb warning
	% https://tex.stackexchange.com/a/285953/105570
	\let\globcount\newcount
	\expandafter\def\csname ver@etex.sty\endcsname{3000/12/31}
\usepackage{autonum}
	% make \Cref work as well as \cref
	% https://tex.stackexchange.com/a/471654/105570
	\makeatletter
	\autonum@generatePatchedReferenceCSL{Cref}
	\makeatother
