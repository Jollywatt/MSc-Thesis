\chapter{The Failure of Fields}







\chapter{Fibre Bundles}
\label{cha:fibre-bundles}

\todo{Treating physical things as fields} would suggest that all values are directly comparable, making expressions like $f(x) + f(y) ∈ A$ geometrically meaningful for different points $x,y ∈ ℳ$.
However, an important lesson from physical theories like general relativity is that it is very often beneficial to distinguish between codomains \emph{at each point in the domain}.

\begin{marginfigure}
	\centering
	\includefigure[0.6\columnwidth]{sphere}
	\caption{
		Vectors in different tangent spaces, and their basis-dependent representation as an $\RR^2$-valued field.
	}
	\label{fig:ball}
\end{marginfigure}

This can be motivated with the simple example of a fluid flowing on a sphere:
The instantaneous fluid velocity at a point is a vector lying in the sphere's tangent plane at that point.
If the fluid flow is given as a field $f : \Sphere^2 → \RR^2$, then any two velocity vectors exist in the ``same'' space, even when \emph{geometrically} they do not (\cref{fig:ball}).
This is more than a purely philosophical point: the fluid flow's representation as a field $f : \Sphere^2 → \RR^2$ is dependent on the choice of basis, i.e., the way in which the single codomain $\RR^2$ is identified with each tangent plane on the sphere.
We would do better with a more geometrical representation of the vector field which is independent of any choice of basis, viewing the fluid velocities at different points as existing in different spaces.

This leads to the formulation of a tangent ``bundle'' $\TT \Sphere^2$, where all the tangent planes of $\Sphere^2$ are collected in a disjoint union forming a manifold.
The vector field on the sphere now becomes a ``section'' of $\TT \Sphere^2$, which is a map $f : \Sphere^2 → \TT \Sphere^2$ satisfying some conditions.
The tangent bundle is a special case of a ``fibre bundle'', which is a manifold consisting of disjoint copies of a space (called the ``fibre'') taken at every point in a base manifold.


\begin{definition}
	\label{def:fibre-bundle}
	A \textdef{fibre bundle} $\fibrebundle[π] F ℱ ℳ$ consists of
	\begin{itemize}
		% \item a \textdef{fibre manifold} $A$;
		\item a \textdef{bulk manifold} $ℱ$;
		\item a \textdef{base manifold} $ℳ$; and
		\item a surjection $π : ℱ → ℳ$, the \textdef{projection}, such that
		\item the inverse image $F_x ≔ π^{-1}(x)$ of a base point $x ∈ ℳ$ is homeomorphic to the \textdef{fibre} $F$.
	\end{itemize}
\end{definition}


\begin{marginfigure}
	\includefigure[\columnwidth]{fibre-bundle}
	\caption{
		(a) A field $f : ℳ → F$, where values at any point can be compared.
		(b) A fibre bundle $\fibrebundle F ℱ ℳ$ with a section $f ∈ \secs(ℱ)$ whose individual fibres $F$ are labelled by base point in $ℳ$.
	}
\end{marginfigure}




The definition above, and all that follow, take place in the category of manifolds.
This means all maps between manifolds \emph{are assumed to be continuous}.
Furthermore, if the qualifier ``smooth'' is present, then the objects exist in the category of \emph{smooth manifolds} in which all maps are smooth (i.e., infinitely differentiable).

Thus, the projection $π : 𝒜 → ℳ$ in \cref{def:fibre-bundle} is required to be continuous; and in a \textdef{smooth fibre bundle}, the projection $π$ is differentiable and $F$, $ℱ$ and $ℳ$ are all smooth manifolds.
