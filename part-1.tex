\part{Special Relativity and Geometric Algebra}

\chapter{Introduction}

The Special Theory of Relativity is a model of \emph{spacetime} --- the geometry in which physical events take place.
Spacetime comprises the Euclidean dimensions of space and time, but only in a way relative to each observer moving through it: there exists no single `universal' ruler or clock.
Instead, two observers in relative motion define different decompositions of spacetime, and their respective clocks and rulers are found to disagree in accordance with Lorentz transformation laws.
The insight of special relativity is that one should focus not on the observer-dependent notions of space and time, but on the Lorentzian geometry of spacetime.

Seven years after Albert Einstein introduced this theory,\sidenote{
	Einstein’s paper \cite{einstein1905electrodynamics} was published in 1905, the so-called \emph{Annus Mirabilis} or ``miracle year'' during which he also published on the photoelectric effect, Brownian motion and the mass-energy equivalence.
	Each of the four papers was a monumental contribution to modern physics.
} he succeeded in formulating a relativistic picture which included gravity.
In this General Theory of Relativity, gravitation is identified with the curvature of spacetime over astronomical distances.
Both theories coincide locally when confined to sufficiently small extents of spacetime, over which the effects of curvature are negligible.

With regard to the Erlangen program,\sidenote{
	Introduced by Felix Klein in 1872 \cite{klein1893erlangen}, the Erlangen program is the characterisation of geometries (Euclidean, hyperbolic, projective, etc.) by their symmetry groups and the properties invariant under those groups.
	E.g., Euclidean geometry studies the invariants of rigid transformations.
} the study of local spacetime geometry amounts to the study of the Poincaré group of its intrinsic symmetries.
These symmetries consist of translations and Lorentz transformations, the latter being the extension of the group of rotations of Euclidean space to the relativistic rotations of spacetime.
The standard matrix representation of the Lorentz group is $\SO^+(1, 3)$, which is the connected component of the orthogonal group
\begin{align}
	\op{O}(1,3) = \set{\lin Λ ∈ \GL(\RR^4) | \lin Λ\trans\lin η\lin Λ = \lin η}
\end{align}
with respect to the bilinear form $η = ±\op{diag}(-1,+1,+1,+1)$.
The rudimentary tools of matrix algebra are sufficient for an analysing the Lorentz group, and are familiar to any physicist.

However, the last century has seen many other mathematical tools be applied to the study of `generalised' rotation groups such as $\SO^+(1,3)$ or the rotation group in $\RR^3$.
Among these tools is the \emph{geometric algebra}, invented\sidenote{
	Clifford algebra was independently discovered by Rudolf Lipschitz two years later \cite{lipschitz1880clifford-alg}. 
	He was the first to use them to the study the orthogonal groups.
} by William Clifford in 1878 \cite{clifford1878grassmann}, which is arguably much better suited to the description of rotations than orthogonal matrix groups, yet still largely unknown to physicists.
It is informative to glean some of the history that led to this (perhaps unfortunate) state of affairs.

\subsubsection{The quest for an optimal formalism for rotations}

Mathematics has seen the invention of a variety of vector formalisms since the 1800s, and the question of which is best suited to physics has a long contentious history.
The vector algebra ``war'' of 1890--1945 saw William Hamilton's prized quaternion algebra $\HH$, hailed as the optimal tool for describing rotations in $\RR^3$, struggle for popularity before being eventually left to gather dust as an old-fashioned curiosity.



\chapter{Preliminary Theory}

\todo{Intro}

Therefore, this section introduces the abstract theory of associative and quotient algebras, which is more generally a part of \emph{ring theory}.\sidenote{
	A \textdef{ring} is a field without the requirement that multiplicative inverses exist or that multiplication commutes; a field is a commutative ring in which non-zero elements are invertible.}
Throughout, $\FF$ denotes the underlying field of some vector space $V$.
Eventually, $\FF$ will always be taken to be $\RR$, but we may begin in generality.
Most definitions in this chapter can be generalised without alteration by replacing the field $\FF$ with a ring; what follows is a brief survey of ring theory, specialised to the case of a field.

\section{Associative Algebras}

\begin{definition}
	\label{def:associative-algebra}
	An \textdef{associative algebra} $A$ is a vector space equipped with a product $⊛ : A × A \to A$ which is associative and bilinear.
\end{definition}
Associativity means $(𝒖 ⊛ 𝒗) ⊛ 𝒘 = 𝒖 ⊛ (𝒗 ⊛ 𝒘)$, while bilinearity means the product is:
\begin{itemize}
	\item compatible with scalars: $(λ𝒖) ⊛ 𝒗 = 𝒖 ⊛ (λ𝒗) = λ(𝒖 ⊛ 𝒗)$ for $λ \in \FF$; and
	\item distributive over addition: $(𝒖 + 𝒗) ⊛ 𝒘 = 𝒖 ⊛ 𝒘 + 𝒗 ⊛ 𝒘$, and similarly for $𝒖 ⊛ (𝒗 + 𝒘)$.
\end{itemize}
This definition can be generalised by relaxing associativity or by letting $\FF$ be a ring.
However, we will use ``algebra'' exclusively to mean an associative algebra over a field (usually $\RR$).

Any ring forms an associative algebra when considered as a one-dimensional vector space.
The complex numbers can be viewed as a real $2$-dimensional algebra by defining $⊛$ to be complex multiplication.



\subsubsection{The tensor algebra}

The most general (associative) algebra containing a given vector space $V$ is the \textdef{tensor algebra over $\TA{V}$}.
The tensor product $⊗$ satisfies exactly the relations of \cref{def:associative-algebra} with no others.
Thus, the tensor algebra associative, bilinear and \emph{free} in the sense that no further information is required in its definition.

As a vector space, the tensor algebra is equal to the infinite direct sum
\begin{align}
	\label{eqn:tensor-algebra-graded-decomposition}
	\TA{V} ≅ \bigoplus_{k=0}^∞ V^{⊗k} ≡ \FF ⊕ V ⊕ (V ⊗ V) ⊕ (V ⊗ V ⊗ V) ⊕ \cdots
\end{align}
where each $\TA[k]{V}$ is the subspace of \textdef{tensors of grade $k$}.



\subsubsection{Quotient algebras}

Owing to the maximal generality of the free tensor algebra, any other associative algebras may be constructed a a \emph{quotient} of $\TA{V}$ with respect to an equivalence relation $\sim$.
In order for a quotient $\TA{V}/{\sim}$ to itself form an algebra, the relation must preserve the associative algebra structure:
\begin{definition}
	\label{def:congruence}
	A \textdef{congruence} on an algebra $A$ is an equivalence relation $\sim$ which is compatible with the algebraic relations, so that if $a \sim a'$ and $b \sim b'$ then $a + b \sim a' + b'$ and $a⊛b \sim a'⊛b'$.
\end{definition}
The quotient of an algebra by a congruence naturally has the structure of an algebra, and so is called a \textdef{quotient algebra}.
\begin{lemma}
	\label{thm:quotient-algebra-by-congruence}
	The \textdef{quotient} $A/{\sim}$ of an algebra $A$ by a congruence $\sim$, consisting of equivalence classes $[a] \in A/{\sim}$ as elements, forms an algebra with the naturally inherited operations $[a] + [b] ≔ [a + b]$ and $[a]⊛[b] ≔ [a⊛b]$.
\end{lemma}
\begin{proof}
	The fact that the operations $+$ and $⊛$ of the quotient are well-defined follows from the structure-preserving properties of the congruence.
	Addition is well-defined if $[a] + [b]$ does not depend on the choice of representatives: if $a' ∈ [a]$ then $[a'] + [b]$ should be $[a] + [b]$.
	By congruence, we have from $a \sim a'$ so that $[a + b] = [a' + b]$ and indeed $[a] + [b] = [a'] + [b]$.
	Likewise for $⊛$.
\end{proof}

Instead of presenting an equivalence relation, it is often easier to define a congruence by specifying the set of elements which are equivalent to zero, from which all other equivalences follow from the algebra axioms.
Such a set of all `zeroed' elements is called an ideal.
\begin{definition}
	A \textdef{(two-sided) ideal} of an algebra $A$ is a subset $I \subseteq A$ which is closed under addition and invariant under multiplication, so that
	\begin{itemize}
		\item if $a, b ∈ I$ then $a + b ∈ I$; and
		\item if $r ∈ A$ and $a ∈ I$ then $r⊛a ∈ I ∋ a⊛r$.
	\end{itemize}
\end{definition}
\begin{lemma}
	An ideal uniquely defines a congruence, and vice versa, by the identification of $I$ as the set of elements equivalent to zero;
	\begin{math}
		a \sim 0 \iff a ∈ I
	.\end{math}
\end{lemma}
\begin{proof}
	The set $I ≔ \set{a | a \sim 0}$ is indeed an ideal because it is closed under addition (for $a, b ∈ I$ we have $\implies a + b \sim 0 + 0 = 0$ so $a + b ∈ I$) and invariant under multiplication (for any $a ∈ I$ and $r ∈ A$, we have $r⊛a \sim r⊛0 = 0 = 0⊛r \sim a⊛r$).
	Conversely, the equivalence defined by $a \sim b ⟺ a - b ∈ I$ is a congruence, since if $a \sim a'$ and $b \sim b'$ then $\sim$ respects addition:
	\begin{align}
		\begin{aligned}
			a - a' &∈ I
		\\	b - b' &∈ I
		\end{aligned}
		\;\Bigg\}
		\implies
		(a + b) - (a' + b') ∈ I
		\iff
		a + b \sim a' + b'
	,\end{align}
	and multiplication:
	\begin{align}
		\begin{aligned}
			(a - a')⊛b &∈ I
		\\	a'⊛(b - b') &∈ I
		\end{aligned}
		\;\Bigg\}
		\implies
		a⊛b - a'⊛b' ∈ I
		\iff
		a⊛b \sim a'⊛b'
	.\end{align}
\end{proof}



\begin{definition}
	The \textdef{dimension} $\dim A$ of a quotient algebra $A = \TA{V}/I$ algebra is its dimension as a vector space.
	The \textdef{base dimension} of $A$ is the dimension of the underlying vector space $V$.
\end{definition}
Algebras may be infinite-dimensional, even with finite base dimension, as in the following example.