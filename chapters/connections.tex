\chapter{Connections on Fibre Bundles}
\label{cha:connections}

We have seen that it is more natural to describe physical fields in the language of fibre bundles rather than simply as maps into a fixed codomain.
However, with a field $f ∈ \secs(ℱ)$ now formulated as a section of a fibre bundle, it no longer makes sense to directly compare values $f|_x$ at different points $x ∈ ℳ$, since each value exists in its own fibre.
But the ability to compare across fibres is desirable, particularly because a notion of derivative requires comparing values across `infinitesimally neighbouring' fibres.
One way to accomplish this (at least for flowable objects) was the Lie derivative of \cref{sec:lieder}.
Another way which is applicable to any bundle is to introduce the additional structure of a \emph{connection}; this then defines the \emph{covariant derivative} of a section.

\begin{marginfigure}
	\includefigure[0.8\columnwidth]{path-dependence}
	\caption{Parallel transport of the northern vector depends on the path taken.}
\end{marginfigure}


A trivial example is the usual connection on (the tangent bundle of) Euclidean space.
There, tangent vectors at a base point may be \emph{parallel transported} (i.e., translated irrotationally) to any other base point in a well-defined, path-independent way.
This defines an isomorphism between every tangent space and tangent space at the origin, forming a connection on $\TT \RR^n$.

We may try to define connections on general fibre bundles in this way --- by choosing an isomorphism from every fibre to a single `reference' fibre.\sidenote{
	This is equivalent to choosing a trivialisation $ℱ → ℳ × F$, or prescribing global coordinates on $ℱ$.
}
But defining a connection like this is needlessly strict, and is of course impossible for non-trivial bundles.
(For example, $\TT \Sphere^2$ is non-trivial; there is no way of smoothly identifying its tangent spaces.)

Instead, it is sufficient to identify fibres \emph{locally}.
In other words, we need only prescribe how values can be compared over infinitesimal paths; from this we can compare any path-connected fibres.
A connection obtained this way is much more general: it accomodates non-trivial bundles and curved connections, where parallel transport may be path-dependent.
(For example, parallel transport on the sphere embedded in $\RR^3$ is path-dependent.)




% \section{Connections on General Fibre Bundles}
\subsubsection{On general fibre bundles: Ehresmann connections}


The most general kind of smooth bundle $ℱ$ is one where the fibres have the minimal structure of a smooth manifold.
We will specify a connection by defining \emph{vertical} and \emph{horizontal motion} within the bulk of the bundle.

A point $p ∈ ℱ$ in the bundle belongs to the fibre $F_{π(p)}$ rooted at the base point $π(p) ∈ ℳ$.
If the point $p$ is moved \emph{within} its fibre, the base point remains fixed and the motion is said to be ``vertical''.
The tangent space $\TT_p F_{π(p)}$ of the fibre (in isolation from the bulk) consists of those displacement vectors which define vertical motion.
Taken together, the vertical tangent spaces of all fibres form the \textdef{vertical bundle}. 
\begin{definition}
	The \textdef{vertical bundle} of a smooth fibre bundle $\fibrebundle F ℱ ℳ$ is a smooth $(\dim F)$-dimensional tangent subbundle $\VV ℱ ⊆ \TT ℱ$ defined by
	\begin{math}
		\VV_p ℱ = \TT_p F_p
	\end{math}
	for each point $p ∈ ℱ$.
\end{definition}


\begin{marginfigure}
	\includefigure{ehresmann-connection}
	\caption{
		Illustration of an Ehresmann connection.
	}
	\label{fig:ehresmann-connection}
\end{marginfigure}
On the other hand, a \textdef{connection} specifies how the value $p ∈ ℱ$ changes when the base point $π(p) ∈ ℳ$ moves, if $p$ is to be considered to move ``horizontally'', i.e., if $p$ is to undergo parallel transport.
\begin{definition}
	A \textdef{horizontal bundle} or \textdef{(Ehresmann) connection} $H$ on a smooth fibre bundle $\fibrebundle F ℱ ℳ$ is a smooth $(\dim ℳ)$-dimensional tangent subbundle $H ⊆ \TT ℱ$ which is complementary to the vertical bundle $V ⊆ \TT ℱ$, in the sense that
	\begin{math}
		\TT_p ℱ = \VV_p ℱ ⊕ H_p
	\end{math}
	for each point $p ∈ ℱ$.
\end{definition}
Note that while the tangent and vertical bundles $\TT ℱ$ and $\VV ℱ$ are canonical constructions, the choice of a horizontal bundle $H$ is not canonical: there may be many distinct horizontal bundles, corresponding to different senses of ``parallel transport''.



The requirement that $H$ be complimentary to $\VV ℱ$ implies $H_p ∩ \VV_p ℱ = \set{\symbf 0}$ at each $p ∈ ℱ$.
This means the restriction of $\dd π : \TT_pℱ \biject \TT_{π(p)}ℳ$ to $H_p ⊆ \TT_pℱ$ is an isomorphism.\sidenote{
	Using the fact that $\ker \dd π = \VV ℱ$, implying $\ker \dd π|_{H_p} = {
	\symbf 0}$.
}
It therefore has an inverse,
\begin{align}
	\label{eqn:lift-of-base-tangent-vectors}
	\dd π|_{H_p}^{-1} : \TT_{π(p)}ℳ \biject H_p
,\end{align}
which acts to ``lift'' tangent vectors from the base into the horizontal subbundle at $p$.
This proves to be a useful construction:
\begin{definition}
	\label{def:connection-map}
	Let $\fibrebundle[π] F ℱ ℳ$ be a fibre bundle with an Ehresmann connection $H ⊆ \TT ℱ$.
	The \textdef{horizontal lift} to the point $p ∈ ℱ$ is the linear map
	\begin{align}
	% 	Γ(p) &: \TT_{π(p)} ℳ → H_p
	% \\	Γ(p) &≔ -\dd π|_{H_p}^{-1}
		Γ(p) ≔ -\dd π|_{H_p}^{-1}
		: \TT_{π(p)} ℳ → H_p
	.\end{align}
	Also define the horizontal lift of a section $f ∈ ℱ$ at $x ∈ ℳ$ by
	\begin{align}
	% 	Γ : \secs(ℱ) &→ \forms[1](ℳ, H)
		Γ(f)|_x &≔ -\dd π|_{H_{f(x)}}^{-1}
	.\end{align}
\end{definition}
\begin{marginfigure}
	\includefigure{connection-map}
	\caption{
		The tangent vector $𝒖$ at $x$ is lifted to the horizontal bulk vector $Γ_𝒖(f)$ at the point $f(x)$.
	}
	\label{fig:connection-map}
\end{marginfigure}
The horizontal lift of a section $f$ is a horizontal-valued $1$-form $Γ(f) ∈ \forms[1](ℳ, H)$ whose action on tangent vectors $𝒖$ we may write as $Γ_𝒖(f) ≔ Γ(f)(𝒖)$.
This device is designed so that tangent vectors $𝒖$ are `lifted' to horizontal bulk vectors $-Γ_𝒖(f)$ located on the section $f$ (see \cref{fig:connection-map}).
`Lifted' means $-Γ_𝒖(f)$ projects onto $𝒖$, so that we have $-\dd π (Γ_𝒖(f)) = 𝒖$.
The minus sign is present to later align with the convention that a plus sign is present in the covariant derivative of a vector section.\sidenote{
	E.g., ``$∇_μ X^a = ∂_μ X^a + Γ_μ{}^a{}_b X^b$''.
} 




\section{Parallel Transportation}

With a connection $H ⊆ \TT ℱ$ defined on a bundle, a bulk value may be moved between fibres so that the motion is always horizontal with respect to the connection.
This is called \textdef{parallel transportation} of the value along a path.

More precisely, a path $γ : [0, 1] → ℳ$ representing the motion of a value $p_0 ∈ ℱ$ from $γ(0) = π(p_0)$ can be \textdef{lifted} to a horizontal path $p_0 : [0, 1] → ℱ$ in the bulk.
This path is `above' $γ$ in the sense that $π(p_γ(λ)) = γ(λ)$, and `horizontal' in the sense that $\dd p_γ(λ) ∈ H_{p_γ(λ)}$ (see \cref{fig:lifted-path}).
% In other words, $p_γ$ is a one-dimensional integral manifold of the connection $H$ restricted to the `wall' $π^{-1}(γ) ⊂ ℱ$.
In other words, $p_γ$ is an integral curve of the connection along $γ$ through $p_0$.

It is useful to describe this path--lifting process as an operator, associating fibre-mappings to each path in $ℳ$.
\begin{definition}
	\label{def:transport-operator}
	If $γ : [0, 1] → ℳ$ is a path, then the \textdef{transport operator} $\trans_γ : F_{γ(0)} → F_{γ(1)}$ is defined by
	\begin{math}
		\trans_γ p = p_γ(1)
	\end{math}
	for any point $p ∈ F_{γ(0)}$ where $p_γ : [0, 1] → ℱ$ is the lifted path satisfying
	\begin{align}
		\label{eqn:transport-operator-path-conds}
		π(p_γ(λ)) = γ(λ)
		\qqtext{and}
		\dd p_γ(λ) ∈ H_{p_γ(λ)}
	\end{align}
	for all $λ ∈ [0, 1]$.
\end{definition}

\begin{marginfigure}
	\includefigure{lifted-path}
	\caption{
		The point $p_0$ parallel transported along a path $γ$, giving the lifted path $p_λ$.
	}
	\label{fig:lifted-path}
\end{marginfigure}

The transport operator is invariant under path reparametrisation, since any path $γ'(λ) = γ(f(λ))$ where $f : [0, 1] → [0, 1]$ is smooth also satisfies \cref{eqn:transport-operator-path-conds} if $γ$ does.
Furthermore, the transport operator respects path concatenation $γ_2 * γ_1$ and inversion,
\begin{align}
	\trans_{γ^{-1}} &= \trans_γ{}^{-1}
,&	\trans_{γ_2*γ_1} &= \trans_{γ_2} \circ \trans_{γ_1}
.\end{align}
% This makes the transport operator a homomorphism from the groupoid of directed paths\sidenote{where the partially-defined group operation is path concatenation} (modulo reparametrisation) into the groupoid of fibre isomorphisms.



Parallel transport along a path involves `integrating' the connection; and conversely, the `derivative' of the transport operator is the horizontal lift, in a way made precise in the following lemma.
\begin{lemma}
	\label{lem:trans-ode}
	The transport operator along a path $γ$ satisfies the ordinary differential equation
	\begin{align}
		\label{eqn:trans-ode}
		\dv{λ} \trans_{γ(λ ← 0)} = -Γ_{\vb{\dot γ}(λ)} \circ \!\trans_{γ(λ ← 0)}
	,\end{align}
	where $γ(λ ← 0)$ denotes the sub-path of $γ$ from $γ(0)$ to $γ(λ)$.
\end{lemma}
\begin{proof}
	If $p ∈ F_{γ(0)}$ then we have
	\begin{math}
		\trans_{γ(λ ← 0)} p = p_γ(λ)
	\end{math}
	where $p_γ$ is the lift of $γ$ through $p$, satisfying the conditions in \cref{def:transport-operator}.
	Differentiating with respect to $λ$,
	\begin{align}
		\label{eqn:trans-ode-working.1}
		\dv{λ} \trans_{γ(λ ← 0)} p = \dd p_γ(λ) ∈ H_{p_γ(λ)}
	,\end{align}
	which is the horizontal by \cref{eqn:transport-operator-path-conds}.
	Additionally, from $π \circ p_γ = γ$ we have $\dd π \circ \dd p_γ = \dd γ$.
	Thus, we see that $\dd p_γ(λ)$ is horizontal lift of $\dd γ(λ)$ to the point $p_γ(λ)$,
	\begin{align}
		\label{eqn:trans-ode-working.2}
		\dd p_γ(λ)
		= \dd π|_{H_{p_γ(λ)}}^{-1}(\dd γ(λ))
		= -Γ_{\vb{\dot γ}(λ)}(p_γ(λ))
	.\end{align}
	Finally, since
	\begin{math}
		p_γ(λ) = \trans_{γ(λ ← 0)} p
	\end{math},
	combining \cref{eqn:trans-ode-working.1,eqn:trans-ode-working.2} we have the result.
\end{proof}

% For a linear bundle (introduced in \cref{sec:vector-bundles}) the composition in \cref{eqn:trans-ode} is just matrix multiplication, and the resulting linear differential equation can be solved explicitly.

Evaluating \cref{lem:trans-ode} at $λ = 0$ yields the following useful result.
\begin{corollary}
	\label{lem:dtrans-is-hlift}
	Let $γ : [0, 1] → ℳ$ be a path and let $p ∈ ℱ_{γ(0)}$.
	\begin{align}
		\dv{λ} \trans_{γ(λ ← 0)} p \, \bigg|_{λ = 0} = -Γ_{\vb{\dot γ}(0)}(p)
	\end{align}
\end{corollary}
An important consequence of this derivative relationship is that, since $\trans_γ ∈ G$ is an element of the group of fibre endomorphisms,\sidenote{
	Technically, $\trans_γ$ can only be called a group element \emph{after} a bundle trivialisation (giving a well-defined identity map between fibres).
} the horizontal lift is Lie algebra--valued, $Γ_𝒖 ∈ \liealg g ≡ T_\op{id} G$.

\section{Covariant Differentiation}

We have seen that a choice of connection $H ⊂ \TT ℱ$ determines which tangent vectors in the bulk of a bundle are horizontal.
This in turn defines the parallel transport operator.
From this we may also define the coordinate-independent \textdef{covariant derivative} as the rate of change of a section with respect to the connection's horizontal.

To decompose vectors into horizontal and vertical components according to $H$, we employ the \textdef{projection} and \textdef{rejection} maps
\begin{align}
	\label{eqn:proj-rej}
	\op{proj}_{H_p} : \TT_p ℱ → H_p
	\qqtext{and}
	\op{rej}_{H_p} : \TT_p ℱ → \VV_p ℱ
\end{align}
defined by $\op{proj}_{H_p} 𝒖 + \op{rej}_{H_p} 𝒖 = 𝒖 ∈ \TT_p ℱ$ and idempotence.


% We expect a submanifold $f$ of $ℱ$ (e.g., a curve or section) to have vanishing covariant derivative if the tangent space is horizontal (i.e., $\dd f$ everywhere lies in $H$).
% In other words, $f$ is covariantly constant if it is an integral manifold of the connection $H ⊆ \TT ℱ$.
% The covariant derivative then measures the failure of $f$ to be covariantly constant; i.e., the rate of change in $f$ relative to the connection's horizontal.

\begin{definition}
	\label{def:covariant-derivative-on-fibre-bundle}
	% The \textdef{covariant derivative} $\df ∇f : \TT_x ℳ → \VV_{f(x)} ℱ$ of a section $f ∈ \secs(ℱ)$ is defined by
	The \textdef{covariant derivative} $\df ∇f ∈ \forms[1](ℳ, \VV ℱ) $ of a section $f ∈ \secs(ℱ)$ is defined by
	\begin{align}
		% ∇f &: \TT_p ℳ → \VV_{f(p)} ℱ
		\label{eqn:covder}
		\df ∇f &= \op{rej}_H{} ∘ \dd f
	.\end{align}
\end{definition}
\Cref{eqn:covder} is a vertical-valued $1$-form, i.e., a linear map
\begin{math}
	\df ∇f|_x : \TT_x ℳ → \VV_{f(x)} ℱ
\end{math}
defined at each $x ∈ ℳ$.
Acting on a vector $𝒖 ∈ \TT_x ℳ$, this reads
\begin{align}
	∇_𝒖f ≔ \df ∇f(𝒖) = \op{rej}_{H_{f(x)}}(\dd f(𝒖)) ∈ \VV_{f(x)} ℱ
.\end{align}
This can be interpreted geometrically as follows.
The true gradient vector $\dd f(𝒖) ∈ \TT_{f(x)}ℱ$ of the section $f$ lies outside the fibre's tangent space $\VV_{f(x)}ℱ ⊆ \TT_{f(x)}ℱ$.
But we do not want to measure horizontal motion --- just the \emph{effective} vertical change of $f(x)$ induced by moving $x$ in the direction of $𝒖$.
Thus, the covariant derivative $∇_𝒖f ∈ \VV_{f(x)}ℱ$ is the vertical projection of $\dd f(𝒖)$ obtained by discarding its horizontal component, where `horizontal' is of course specified by the connection  (see \cref{fig:covariant-derivative}).


\begin{marginfigure}
	\includefigure{covariant-derivative}
	\caption{
		Covariant derivative of $f$ at $x ∈ ℳ$ along $𝒖 ∈ \TT_x ℳ$.
		The vector $-Γ_f(𝒖) = \dd π|_{H_{f(x)}}^{-1}(𝒖)$ indicates horizontal motion under the connection $H$, and $∇_𝒖f$ is the derivative relative to this horizontal.
	}
	\label{fig:covariant-derivative}
\end{marginfigure}

\begin{lemma}
	\label{lem:covariant-derivative-rewritten}
	The covariant derivative as in \cref{def:covariant-derivative-on-fibre-bundle} is equivalent to
	\begin{align}
		∇_𝒖 f = \dd f(𝒖) + Γ_𝒖(f) 
	,\end{align}
	where $\dd f$ is the push-forward of $f ∈ \secs(ℱ)$ and $Γ$ is the horizontal lift as in \cref{def:connection-map}.
\end{lemma}

\begin{proof}
	By the defining property of the projection and rejection \eqref{eqn:proj-rej},
	\begin{align}
		\dd f = \op{rej}_H{} ∘ \dd f + \op{proj}_H{} ∘ \dd f
	\end{align}
	since $\dd f : \TT ℳ → \TT ℱ$ is linear.
	Therefore, rewriting \cref{def:covariant-derivative-on-fibre-bundle},
	\begin{align}
		\df ∇f = \op{rej}_H{} ∘ \dd f
			= \dd f - \op{proj}_H{} ∘ \dd f
	.\end{align}
	Using \cref{eqn:lift-of-base-tangent-vectors}, the projection operator at $p ∈ ℱ$ can be written as
	\begin{align}
		\op{proj}_{H_p} = \dd π|_{H_p}^{-1} ∘ \dd π
	.\end{align}
	Finally, because $f$ is a section, $π ∘ f = \op{id}$ and so
	\begin{math}
		\dd π ∘ \dd f = \op{id}
	\end{math}
	by the chain rule (\cref{lem:differential-chain-rule}).
	Thus, acting on a base vector $𝒖 ∈ \TT_x ℳ$,
	\begin{align}
		∇_𝒖f
			&= \dd f(𝒖) - \dd π|_{H_{f(x)}}^{-1} ∘ \dd π ∘ \dd f (𝒖)
		\\	&= \dd f(𝒖) - \dd π|_{H_{f(x)}}^{-1} (𝒖)
	,\end{align}
	which by \cref{def:connection-map} gives the result.
\end{proof}









\subsubsection{Coordinate representation}

At this point, we may introduce component forms of the above devices for a general fibre bundle.
Choose a (local) trivialisation of $ℱ$ so that $\set{x^A} = \set{x^μ, x^a}$ are (local) coordinates on $ℳ$ and the fibres, respectively.
(Capital Latin indices run over all components, so we may write $(p^A) = (x^μ, x^a)$ for a bulk value $p ∈ ℱ$.)
Vertical motion fixes the base coordinates, but the fibre coordinates $x^a$ are \emph{not} required to be constant under horizontal motion.

% The covariant derivative in \cref{lem:covariant-derivative-rewritten} acting on a section $f : ℳ → ℱ$ in the direction $𝒖 ∈ \TT_x ℳ$ has the full form
% \begin{align}
% 	∇_𝒖f = \dd f(𝒖) + Γ_f(𝒖)
% .\end{align}
% The connection map $Γ_f : \TTℳ → H ⊆ \TT ℱ$ is linear, so it is a $\TT ℱ$-valued $1$-form which we may write as a bitensor (i.e., matrix--valued) section $Γ_f : ℳ → \TT^*ℳ ⊗ \TT ℱ$.
% Without reference to $f$, we have $Γ : ℱ → \TT^*ℳ ⊗ \TT ℱ$ with $Γ_f(x) ≡ Γ(f(x))$.
% % The horizontal projection $\dd π|_{H_{f(x)}}^{-1} : \TT_xℳ → H_{f(x)} ⊆ \TT ℱ$ is a linear operator at each point in $ℱ$, so we may write it as a $\TT ℱ$-valued $1$-form.
Denote the associated coordinate basis of $\TT ℱ$ by $(\∂_A) = (\∂_μ, \∂_a)$.
Recall that $Γ(f) ∈ \forms[1](ℳ, H)$ is a $1$-form, and hence is linear in its tangent vector argument $𝒖 ∈ \secs(\TT ℳ)$.
Thus, we define the components
\begin{align}
	\label{eqn:covariant-derivative-coords.1}
	Γ_μ ≔ Γ_{\∂_μ}
\end{align}
so that
\begin{math}
	Γ_𝒖(f) = u^μ Γ_μ(f)
.\end{math}
Since $Γ_𝒖(f)|_x ∈ H_{f(x)}$ is a (horizontal) vector, we may also define the $2$-component object $Γ_μ{}^A$ by
\begin{align}
	Γ_μ(f) = Γ_μ{}^A(f) \, \∂_A
.\end{align}
Note that horizontal vectors have both fibre \emph{and} base components,
\begin{align}
	Γ_μ{}^A \, \∂_A = Γ_μ{}^ν \, \∂_ν + Γ_μ{}^a \, \∂_a
.\end{align}
Indeed, the same applies to the push-forward
\begin{math}
	\dd f = \dd f^μ \, \∂_μ + \dd f^a \, \∂_a
\end{math}
since $\dd f$ is not vertical (the non-verticality of the usual derivative $\dd f$ is what the covariant derivative attempts to fix).
However, since $∇_μ f ∈ \VV ℱ$ as a whole \emph{is} vertical, the base components $Γ_μ{}^ν$ and $∂_μ f^ν$ must cancel.

This is verified by noting that
\begin{align}
	\label{eqn:covariant-derivative-coords.2}
	\dd π (\dd f (𝒖)) = 𝒖
	\qqtext{and}
	\dd π(-Γ_{f(x)}(𝒖)) ≡ \dd π (\dd π|_{H_{f(x)}}^{-1}(𝒖)) = 𝒖
\end{align}
are equal.
In effect, $\dd π$ projects onto components of the base,
\begin{math}
	\dd π(X^A \∂_A) = X^ν \∂_ν
,\end{math}
and so \cref{eqn:covariant-derivative-coords.2} implies $\dd f^ν(𝒖) = -u^μΓ_μ{}^ν$.
Therefore, in components, the covariant derivative of a section is
\begin{align}
	\label{eqn:covder-comps-general-bundle}
	∇_μ f^a = ∂_μ f^a + Γ_μ{}^a(f)
,\end{align}
with base components of $\dd f(𝒖)$ and $Γ_𝒖(f)$ suppressed.\sidenote{
	In practice, one usually works with a (local) trivialisation in which $f : ℳ → F$ is given as a \emph{field}.
	Then, $\dd f = \dd f^a \, \∂_a$ has no base components anyway, so we take $Γ_μ(f) = Γ_μ{}^a(f) \, \∂_a$.
}
Note that $f$ need not be a vector section of a linear bundle --- \cref{eqn:covder-comps-general-bundle} is general to smooth fibre bundles of any kind.


\section{Connections on Vector Bundles}
\label{sec:vector-bundles}

So far, we have treated connections in the setting of a general smooth fibre bundle.
We now consider connections and their associated covariant derivatives on \emph{vector} bundles $\fibrebundle V 𝒱 ℳ$, with more or less additional structure.

% \subsubsection{On vector bundles}

In general, the transport operator over a path is an invertible map between the start and end fibres.
For a vector bundle, we often require this to be a \emph{linear} map, in which case the connection is said to be linear.
By \cref{lem:trans-ode}, this means the horizontal lift is also linear in its fibre argument,
\begin{math}
	\df Γ(λ^i X_i) = λ^i \df Γ(X_i)
,\end{math}
so we may regard $Γ_𝒖$ as a matrix and $\df Γ$ as a matrix-valued $1$-form, acting on vectors $X ∈ 𝒱$ by matrix multiplication,
\begin{math}
	\df Γ X ≔ \df Γ(X)
.\end{math}

If $\set{\ve_a}$ is a basis for some vector bundle $𝒱$ with a linear connection, then we define $3$-component \textdef{connection coefficients},
\begin{align}
	Γ_μ{}^a{}_b ≔ Γ_μ{}^a \ve_b
	\marginnote{
		Our notation suggest a $1$-form
		$\df Γ = Γ_μ \dx^μ$ of matrices $Γ_μ{} = Γ_μ{}^a{}_b \ve_a ⊗ \ve^b$, but index placement varies in the literature:
		\cite{misner1973gravitation} uses $Γ^a{}_{bμ}$; \cite{carroll2019gr} uses $Γ^a_{μb}$.
	}
.\end{align}
We may write expressions in both basis-free and component forms; 
\begin{align}
	Γ_𝒖 X = u^μ \, Γ_μ{}^a{}_b \, X^b \, \ve_a
.\end{align}

Linearity also allows the covariant derivative to be expressed as the limit of a difference, similar to the usual analytical definition of the derivative of a real function.
\begin{lemma}
	\label{lem:trans-and-covariant-der}
	If $γ : [0, 1] → ℳ$ is a path and $X ∈ \secs_γ(𝒱)$ is a smooth vector section defined on $γ$, then
	\begin{align}
		∇_{\vb{\dot γ}(0)} X|_{γ(0)}
		&= \lim_{ε → 0} \frac{X|_{γ(ε)} - \trans_{γ(ε ← 0)}X|_{γ(0)}}{ε}
	\\	&= \dv{λ}\eval{\qty(X|_{γ(λ)} - \trans_{γ(λ ← 0)}X|_{γ(0)})}_{λ=0}
	.\end{align}
\end{lemma}
\begin{proof}
	Using \cref{lem:dtrans-is-hlift}, the right-hand side is equal to
	\begin{align}
		\dd X(\vb{\dot γ}(0)) + Γ_{\vb{\dot γ}(0)}X
	,\end{align}
	which by \cref{lem:covariant-derivative-rewritten} is equal to $∇_{\vb{\dot γ}(0)} X|_{γ(0)}$.
\end{proof}



\subsubsection{Metric compatibile connections}

A linear connection on a metrical vector bundle $\fibrebundle V 𝒱 ℳ$ is called \textdef{metric compatible} if for any vectors $X, Y ∈ 𝒱$,
\begin{align}
	\ip{\trans X, \trans Y} = \ip{X, Y}
\end{align}
where the transport operators are over some common path. 
\begin{lemma}
	A metric compatible connection satisfies
	\begin{align}
		\ip{\df Γ X, Y} = -\ip{X, \df Γ Y}
		\qqtext{or}
		Γ_{μab} = -Γ_{μba}
	\end{align}
	where $Γ_{μab} = η_{ac}Γ_μ{}^c{}_b$.
\end{lemma}
\begin{proof}
	Consider transport along a path $γ(λ ← 0)$, and abbreviate $T_λ ≔ \trans_{γ(λ ← 0)}$.
	Since $\ip{T_λ X, T_λ Y} = \ip{X, Y}$ is constant with respect to $λ$, its $λ$-derivative vanishes.
	But we also have
	\begin{align}
		0 = \dv{λ} \ip{T_λ X, T_λ Y} \bigg|_{λ = 0}
		% &= \dv{λ} \ip{X, Y} \bigg|_{λ = 0} = 0
		&= \ip{\dv{λ} T_λ X \, \big|_{λ = 0}, Y} + \ip{X, \dv{λ} T_λ Y \, \big|_{λ = 0}}
	\\	&= -\ip{Γ_{\vb{\dot γ}(0)}X, Y} - \ip{X, Γ_{\vb{\dot γ}(0)}Y}
	.\end{align}
	Since $γ$ is arbitrary, we have
	\begin{math}
		\ip{Γ_𝒖 X, Y} + \ip{X, Γ_𝒖 Y} = 0
	\end{math}
	for all $𝒖 ∈ \TT ℳ$.

	Writing this in component form,
	\begin{align}
		η_{ab} \, Γ_μ{}^a{}_c \, X^c \, Y^b
		= -η_{ab} \, X^a \, Γ_μ{}^b{}_c \, Y^c
	\end{align}
	which implies
	\begin{math}
		η_{ab} \, Γ_μ{}^a{}_c
		= -η_{ab} \, Γ_μ{}^b{}_c
	\end{math}
	since $X$ and $Y$ are arbitrary.
\end{proof}

Metric-compatible connections are not unique.
If $n = \dim ℳ$ and $d = \dim V$, then there are $nd^2$ components of $Γ_μ{}^a{}_b$, subject to $nd(d+1)/2$ compatibility equations $Γ_{μab} + Γ_{μba} = 0$, leaving $nd(d+1)/2$ degrees of freedom.



\subsubsection{On algebra bundles}

On vector bundles equipped with an associative product, we often want the linear connection to be constrained so that $\df ∇$ is a derivation;
\begin{align}
	\label{eqn:covder-derivation}
	\df ∇ (A ⊛ B) = (\df ∇ A) ⊛ B + A ⊛ (\df ∇ B)
.\end{align}
This is equivalent to requiring that the transport operator respects multiplication,
\begin{align}
	\label{eqn:trans-respect-prod}
	(\trans X) ⊛ (\trans Y) = \trans (X ⊛ Y)
,\end{align}
similar to the metric compatibility criterion.
% \begin{lemma}
% 	\label{lem:covder-prod}
% 	Let $𝒜$ be an algebra bundle whose product $⊛$ is respected by parallel transportation (as in \cref{eqn:trans-respect-prod}).
% 	Then, $\df ∇$ is a $⊛$-derivation, and
% 	\begin{align}
% 		\label{eqn:covder-prod}
% 		\df ∇ (\etc{X_\i}⊛k) = \df\dd (\etc{X_\i}⊛k) + \sum_{i=1}^k \etcmid{X_\i}{\df Γ X_i}⊛k
% 	\end{align}
% 	for $X_i ∈ \secs(𝒱)$.
% \end{lemma}
\begin{proof}
	We will derive \cref{eqn:covder-derivation} from \cref{eqn:trans-respect-prod}, showing their equivalence.
	Denote $T_λ ≔ \trans_{γ(λ ← 0)}$ for some path $γ$.
	Using \cref{lem:trans-and-covariant-der}, we have
	% \begin{fullwidth}
	\begin{align}
		∇_{\vb{\dot γ}(0)} (\etc{X_\i}⊛k)
		= \df\dd(\etc{X_\i}⊛k)(\vb{\dot γ}(0))
		- \dv{λ} T_λ(\etc{X_\i}⊛k) \bigg|_{λ = 0}
	.\end{align}
	We already know that $\df\dd$ is a derivation.
	For the rightmost term, \cref{eqn:trans-respect-prod}, linearity and associativity imply
	\begin{align}
		-\dv{λ} \etc{T_λ X_i}⊛k \, \bigg|_{λ=0} = -\sum_{i=1}^k \etcmid{X_\i}{\dv{λ} T_λ X_i \big|_{λ=0}}⊛k
	,\end{align}
	which by \cref{lem:dtrans-is-hlift} gives the result, after removing reverence to the arbitrary vector $\vb{\dot γ}(0)$.
\end{proof}

Consequently, a linear connection on a vector bundle $𝒱$ induces a \emph{unique} $⊛$-respecting connection on the algebra bundle generated by $⊛$, since the covariant derivative of a product may be reduced to a product of covariant derivatives of vectors.
For example, for a tensor bundle $\TA{𝒱}$ with a metric compatible connection, we derive the familiar formula for general type-$(p, q)$ tensors, written purely in tems of the connection coefficients for $𝒱$.
\begin{fullwidth}
\begin{align}
	\label{eqn:covder-general-tensor}
	∇_μT^{a_1\cdots a_p}{}_{b_1\cdots b_q}
	= ∂_μT^{a_1\cdots a_p}{}_{b_1\cdots b_q}
	+ \sum_{i = 1}^p Γ_μ{}^{a_i}{}_c T^{a_1\cdots c\cdots a_p}{}_{b_1\cdots b_q}
	- \sum_{j = 1}^q Γ_μ{}^c{}_{b_j} T^{a_1\cdots a_p}{}_{b_1\cdots c\cdots b_q}
\end{align}
\end{fullwidth}
% Note that \cref{eqn:covder-general-tensor} is written only in terms of the connection coefficients $Γ_μ{}^a{}_b$ on $𝒱$.


\subsection{Bivector connections on multivector bundles}

Using the tools of geometric algebra, the covariant derivative associated with a metric--compatible connection may be expressed as a bivector--valued form.
This representation has the advantage that it is independent of the kind of multivector object being differentiated.
(In stark contrast to \cref{eqn:covder-general-tensor} for a general tensor, for example.)

To derive the bivector connection, begin with the covariant derivative of a vector $𝑿 ∈ \GA[1](𝒱, η)$,
\begin{align}
	∇_μ 𝑿 = (∂_μ X^a + Γ_μ{}^a{}_b X^b) \ve_a
.\end{align}
Rewrite the non-derivative term as
\begin{align}
	Γ_μ{}^a{}_b \, \ve_a X^b
	&= Γ_{μab} \, \ve^a (\ve^b \fatdot 𝑿)
\\	&= \frac12 Γ_{μab} \, (\ve^a (\ve^b \fatdot 𝑿) - (𝑿 \fatdot \ve^a) \ve^b )
\end{align}
using the fact that $Γ_{μab} = -Γ_{μba}$ for a metric compatible connection, and that $\ve^a \fatdot 𝑿 = 𝑿 \fatdot \ve^a$ is a scalar commuting with $\ve^b$.
Then, since for vectors the inner product is $𝑿 \fatdot 𝒀 = \frac12(𝑿𝒀 + 𝒀𝑿)$, this is
\begin{align}
	\frac14 Γ_{μab} \, (\ve^a \ve^b 𝑿 + \ve^a 𝑿 \ve^b - 𝑿 \ve^a \ve^b - \ve^a 𝑿 \ve^a)
	&= \frac14 Γ_{μab} \, (\ve^a \ve^b 𝑿 - 𝑿 \ve^a \ve^b)
.\end{align}
In the right-hand side, the scalar parts from the products between $\ve^a$ and $\ve^b$ cancel, leaving a commutator product of the bivector $\ve^a ∧ \ve^b$ with $𝑿$,
\begin{align}
	\frac12 Γ_{μab} \, (\ve^a ∧ \ve^b) × 𝑿 = ω_μ × 𝑿
,\end{align}
where we define the \textdef{connection bivectors} in the basis $\set{\ve_a}$ by
\begin{align}
	\label{eqn:connection-bivectors}
	ω_μ ≔ \frac12 Γ_{μab} \, \ve^a ∧ \ve^b
.\end{align}
Thus, we may write the covariant derivative of $𝑿$ as
\begin{align}
	\label{eqn:covder-connection-bivectors}
	∇_μ 𝑿 = ∂_μ 𝑿 + ω_μ × 𝑿
,\end{align}
and define the \textdef{connection bivector $1$-form} $\df ω$ by
\begin{math}
	\df ω(𝒖) ≡ ω_𝒖 ≔ u^a ω_a
.\end{math}

The connection bivector is especially useful because the form of \cref{eqn:covder-connection-bivectors} is in fact general to \emph{all} multivectors.


\begin{lemma}
	\label{lem:covder-multivector}
	The covariant derivative of any multivector $A ∈ \GA(𝒱, η)$ is
	\begin{align}
		\df ∇ A = \df\dd A + \df ω × A
	.\end{align}
\end{lemma}
\begin{proof}
	The covariant derivative is a derivation if the connection respects the geometric product.
	Therefore, the covariant derivative of a product of $k$-many vectors is
	\begin{align}
		\df ∇(\etc{𝒖_\i}{}k)
		% &= \sum_{i=1}^k \etcmid{𝒖_\i}{(\df ∇ 𝒖_i)}{}k
		&= \sum_{i=1}^k \etcmid{𝒖_\i}{(\df\dd 𝒖_i + \df ω × 𝒖_i)}{}k
	% \\	&= \df\dd(\etc{𝒖_\i}{}k) + \sum_{i=1}^k \etcmid{𝒖_\i}{(\df ω × 𝒖_i)}{}k
	\\	&= \df\dd(\etc{𝒖_\i}{}k) + \df ω × (\etc{𝒖_\i}{}k)
	,\end{align}
	using \cref{eqn:covder-connection-bivectors} and the fact that commutation by a bivector is a derivation (\cref{lem:commutator-derivation}).
	Since all multivectors are linear combinations of products of vectors, the general result follows.
\end{proof}


A rapid alternative derivation of \cref{lem:covder-multivector} starts from the observation that parallel transport along a path may be written as
\begin{align}
	\trans_γ A = R A \rev{R}
,\end{align}
since any transformation continuously connected to the identity which preserves the geometric product belongs to the rotor group, $\op{Spin}^+$ (see \cref{sec:rotors}).
Any such rotor is of the form $R = e^{σ/2}$ for a bivector $σ$, so we have
\begin{align}
	\dv{λ} \trans_{γ(λ ← 0)} A = \frac12 R (σ A - A σ) \rev{R}
\end{align}
where $σ = σ(λ)$ and hence $R$ are functions of the path parameter.
At $λ = 0$, the rotor is trivial, so by \cref{lem:dtrans-is-hlift} we find
\begin{align}
	\dv{λ} \trans_{γ(λ ← 0)} A \, \bigg|_{λ=0} = -Γ_{\vb{\dot γ}(0)}(A) = σ(0) × A
.\end{align}
Thus, the horizontal lift is given by commutation with a specified bivector.
Since this holds for arbitrary multivectors $A$, by \cref{lem:covariant-derivative-rewritten} we have the universally applicable formula for the covariant derivative of a multivector
\begin{align}
	∇_𝒖 A = ∂_𝒖 A + ω_𝒖 × A
\end{align}
where $ω_𝒖$ is the required bivector.





% \section{Covariant Algebraic Derivatives}


% The covariant derivative $∇_𝒖$ is analogous to the directional derivative $∂_𝒖$ of \cref{sec:algder}, but defined for sections on a manifold $F ∈ \secs(𝒜)$ instead of fields on a vector space $F : V → A$.
% In identical vein to \cref{def:algder}, it is useful to define a `total' derivative operator $𝒟$ which is independent of a direction $𝒖 ∈ \TT ℳ$.
% Like the algebraic derivative $\DD$ of \cref{sec:algder}, $𝒟$ is defined whenever an inclusion $ι : \TT^*ℳ → 𝒜$ is given (but $ι$ is usually left implicit) enabling tangent vectors to be multiplied by elements in the algebra.

% \begin{definition}
% 	\label{def:covalgder}
% 	Let $\fibrebundle A 𝒜 ℳ$ be an algebra bundle with product $⊛$ and connection $\df ∇$, equipped with an inclusion $ι : \TT^*ℳ → 𝒜$.
% 	The \textdef{covariant algebraic derivative} of a section $F ∈ \secs(𝒜)$ is
% 	\begin{align}
% 		𝒟 F ≔ ι(\ve^a) ⊛ ∇_{\ve_a} F
% 	\end{align}
% 	(summation on $a$) where $\set{\ve_a} ⊂ \secs(\TT ℳ)$ and $\set{\ve^a} ⊂ \secs(\TT^* ℳ)$ are dual bases of tangent sections.
% \end{definition}


% \subsection{The covariant vector derivative}


% On a geometric algebra bundle $\GA(\TT ℳ, η)$, \cref{def:covalgder} is the \textdef{covariant vector derivative} $\VD$ and has similar properties to the vector derivative $\vd$ on $\GA(V, η)$.
% Given a basis $\set{\ve^a}$ (simultaneously of $\secs(\TT^* ℳ)$ and $\GA[1](ℳ, η)$, so $ι$ may be omitted), we have the operator
% \begin{align}
% 	\VD = \ve^a ∇_a
% ,\end{align}
% sharing the algebraic properties of a grade-$1$ vector.\sidenote{Hence its bold symbol, like $\vd$.}
% In particular, for a $k$-vector section $F ∈ \GA[k](ℳ, η)$, there is a decomposition into $(k ± 1)$-grade parts,
% \begin{align}
% 	\VD F = \VD \lcontr F + \VD ∧ F
% ,\end{align}
% corresponding to the covariant curl and `cocurl'.