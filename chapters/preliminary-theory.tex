\chapter{Preliminary Theory}
\label{cha:preliminary-theory}


Many of the tools we will develop take place in various associative algebras.
As well as the geometric algebra of spacetime, we will encounter tensors, exterior forms, quaternions, and other structures in this category.
Instead of defining each algebra axiomatically as needed, it is easier to develop the general theory of associative algebras, taking special 
This enables the use of the same tools and the same terminology throughout.

Therefore, this section is an overview of the abstract theory of associative algebras, which more generally belongs to \emph{ring theory}.\sidenote{
	A \textdef{ring} is a field without the requiring commutativity nor existence of multiplicative inverses.
}
\emph{Algebras}, \emph{quotients} and \emph{gradings} are defined, as well as \emph{tensors}, \emph{multivectors} and \emph{exterior forms}.
Most definitions in this chapter can be readily generalised by replacing the field $\FF$ with a ring.
The excitable reader may skip this chapter and refer back as needed.


\section{Associative Algebras}

Throughout, $\FF$ denotes the underlying field of some vector space.
(Eventually, $\FF$ will always be taken to be $\RR$, but we may begin in generality.)
\begin{definition}
	\label{def:associative-algebra}
	An \textdef{associative algebra} $A$ is a vector space equipped with a product $⊛ : A × A \to A$ which is associative and bilinear.
\end{definition}
Associativity means $(𝒖 ⊛ 𝒗) ⊛ 𝒘 = 𝒖 ⊛ (𝒗 ⊛ 𝒘)$, while bilinearity means the product is:
\marginnote{
	\emph{Examples.}
	Any ring forms an associative algebra when considered as a one-dimensional vector space.
	The complex numbers can be viewed as a real $2$-dimensional algebra by defining $⊛$ to be complex multiplication;
	\begin{math}
		(x_1, y_1)⊛(x_2, y_2) ≔ (x_1x_2 - y_1y_2, x_1y_2 + y_1x_2)
	.\end{math}
}
\begin{itemize}
	\item compatible with scalars: $(λ𝒖) ⊛ 𝒗 = 𝒖 ⊛ (λ𝒗) = λ(𝒖 ⊛ 𝒗)$ for $λ \in \FF$; and
	\item distributive over addition: $(𝒖 + 𝒗) ⊛ 𝒘 = 𝒖 ⊛ 𝒘 + 𝒗 ⊛ 𝒘$, and similarly for $𝒖 ⊛ (𝒗 + 𝒘)$.
\end{itemize}
This definition can be generalised by relaxing associativity or by letting $\FF$ be a ring.
However, we will use ``algebra'' exclusively to mean an associative algebra over a field (usually $\RR$).




\subsubsection{The free tensor algebra}

The most general (associative) algebra containing a given vector space $V$ is the \textdef{tensor algebra $\TA{V}$}.
The tensor product $⊗$ satisfies exactly the relations of \cref{def:associative-algebra} with no others.
Thus, the tensor algebra associative, bilinear and \emph{free} in the sense that no further information is required in its definition.

As a vector space, the tensor algebra is equal to the infinite direct sum
\begin{align}
	\label{eqn:tensor-algebra-graded-decomposition}
	\TA{V} ≅ \bigoplus_{k=0}^∞ V^{⊗k} ≡ \FF ⊕ V ⊕ (V ⊗ V) ⊕ (V ⊗ V ⊗ V) ⊕ \cdots
\end{align}
where each $\TA[k]{V}$ is the subspace of \textdef{tensors of grade $k$}.

\subsection{Quotient algebras}

Owing to the maximal generality of the free tensor algebra, any other associative algebras may be constructed as a \emph{quotient} of $\TA{V}$.
In order for a quotient $\quot{\TA{V}}{\sim}$ to itself form an algebra, the equivalence relation $\sim$ must preserve the associative algebra structure.
\begin{definition}
	\label{def:congruence}
	A \textdef{congruence} on an algebra $A$ is an equivalence relation $\sim$ which is compatible with the algebraic relations, so that if $a \sim a'$ and $b \sim b'$ then $a + b \sim a' + b'$ and $a⊛b \sim a'⊛b'$.
\end{definition}
The quotient of an algebra by a congruence naturally has the structure of an algebra, and so is called a \textdef{quotient algebra}.
\begin{lemma}
	\label{thm:quotient-algebra-by-congruence}
	The \textdef{quotient} $\quot A {\sim}$ of an algebra $A$ by a congruence $\sim$, consisting of equivalence classes $[a] \in \quot A {\sim}$ as elements, forms an algebra with the naturally inherited operations $[a] + [b] ≔ [a + b]$ and $[a]⊛[b] ≔ [a⊛b]$.
\end{lemma}
\begin{proof}
	The fact that the operations $+$ and $⊛$ of the quotient are well-defined follows from the structure-preserving properties of the congruence.
	Addition is well-defined if $[a] + [b]$ does not depend on the choice of representatives: if $a' ∈ [a]$ then $[a'] + [b]$ should be $[a] + [b]$.
	By congruence, we have from $a \sim a'$ so that $[a + b] = [a' + b]$ and indeed $[a] + [b] = [a'] + [b]$.
	Likewise for $⊛$.
\end{proof}

Instead of presenting an equivalence relation, it is often easier to define a congruence by specifying the set of elements which are equivalent to zero, from which all other equivalences follow from the algebra axioms.
Such a set of all `zeroed' elements is called an ideal.
\begin{definition}
	\label{def:ideal}
	A \textdef{(two-sided) ideal} of an algebra $A$ is a subset $I \subseteq A$ which is closed under addition and invariant under multiplication, so that
	\begin{itemize}
		\item if $a, b ∈ I$ then $a + b ∈ I$; and
		\item if $r ∈ A$ and $a ∈ I$ then $r⊛a ∈ I ∋ a⊛r$.
	\end{itemize}
\end{definition}

We will use the notation $\gen{A}$ to mean the ideal generated by setting $a \sim 0$ for all $a$ in $A$.
For example, $\gen{a} = \spanof{r ⊛ a ⊛ r' | r, r' ∈ A}$ is the ideal consisting of sums and products involving the specified element $a$, and $\gen{𝒖 ⊗ 𝒖 | 𝒖 ∈ V}$, or simply $\gen{𝒖⊗𝒖}$, is the ideal in $\TA{V}$ consisting of sums of terms of the form $a ⊗ 𝒖 ⊗ 𝒖 ⊗ b$ for vectors $𝒖$ and arbitrary $a, b ∈ \TA{V}$.

\begin{lemma}
	\label{lem:congruence-ideal-equiv}
	An ideal uniquely defines a congruence, and vice versa, by the identification of $I$ as the set of zero elements,
	\begin{math}
		a ∈ I \iff a \sim 0
	.\end{math}
\end{lemma}
\begin{proof}
	The set $I ≔ \set{a | a \sim 0}$ is indeed an ideal because it is closed under addition (for $a, b ∈ I$ we have $\implies a + b \sim 0 + 0 = 0$ so $a + b ∈ I$) and invariant under multiplication (for any $a ∈ I$ and $r ∈ A$, we have $r⊛a \sim r⊛0 = 0 = 0⊛r \sim a⊛r$).
	Conversely, let $a \sim a'$ and $b \sim b'$.
	Since $\sim$ respects addition:
	\begin{align}
		\begin{aligned}
			a - a' &∈ I
		\\	b - b' &∈ I
		\end{aligned}
		\;\Bigg\}
		\implies
		(a + b) - (a' + b') ∈ I
		\iff
		a + b \sim a' + b'
	,\end{align}
	and multiplication:
	\begin{align}
		\begin{aligned}
			(a - a')⊛b &∈ I
		\\	a'⊛(b - b') &∈ I
		\end{aligned}
		\;\Bigg\}
		\implies
		a⊛b - a'⊛b' ∈ I
		\iff
		a⊛b \sim a'⊛b'
	,\end{align}
	the equivalence defined by $a \sim b ⟺ a - b ∈ I$ is a congruence.
\end{proof}

The equivalence of ideals and congruences is a general feature of abstract algebra.\sidenote{
	E.g., in group theory, ideals are \emph{normal subgroups} and define congruences, which are equivalence relations satisfying $gag^{-1} \sim \op{id}$ whenever $a \sim \op{id}$.
	% A group modulo a normal subgroup forms a quotient group.
}
Furthermore, both can be given in terms of a homomorphism between algebras,\sidenote{
	A \emph{homomorphism} is a structure-preserving map; in the case of algebras, a linear map $Ψ : A → A'$ which satisfies $Ψ(a⊛b) = Ψ(a)\mathrel{⊛'}Ψ(b)$.
} and this is often the most convenient way to define a quotient.
\begin{theorem}[first isomorphism theorem]
	\label{thm:first-iso}
	If $Ψ : A → B$ is a homomorphism, between algebras, then
	\begin{enumerate}
		\item the relation $a \sim b$ defined by $Ψ(a) = Ψ(b)$ is a congruence;
			\label{item:first-iso:cong}
		\item the kernel $I ≔ \ker Ψ$ is an ideal; and
			\label{item:first-iso:ker}
		\item the quotients $\quot{A}{\sim} ≡ \quot{A}{I} ≅ Ψ(A)$ are all isomorphic.
			\label{item:first-iso:iso}
	\end{enumerate}
\end{theorem}
\begin{proof}
	We assume $A$ and $B$ associative algebras.
	(For a proof in universal algebra, see \cite[§15]{gallian2021abstract-algebra}.)

	To verify \cref{item:first-iso:cong}, suppose that $Ψ(a) = Ψ(a')$ and $Ψ(b) = Ψ(b')$ and note that $Ψ(a + a') = Ψ(b + b')$ by linearity and $Ψ(a⊛b) = Ψ(a'⊛b')$ from $Ψ(a⊛b) = Ψ(a)⊛Ψ(b)$, so the congruence properties of \cref{def:congruence} are satisfied.

	For \cref{item:first-iso:ker}, note that $\ker Ψ$ is a vector subspace, and that $a ∈ \ker Ψ$ implies $a⊛r ∈ \ker Ψ$ for any $r ∈ A$ since $Ψ(a⊛r) = Ψ(a)⊛Ψ(r) = 0$.
	Thus, $\ker Ψ$ is an ideal by \cref{def:ideal}.

	The first equivalence in \cref{item:first-iso:iso} follows from \cref{lem:congruence-ideal-equiv}.
	For an isomorphism $Φ : \quot A {\ker Ψ} → Ω(A)$, pick $Φ([a]) = Ψ(a)$.
	This is well-defined because the choice of representative of the equivalence class $[a]$ does not matter; $a \sim a'$ if and only if $Ψ(a) = Ψ(a')$ by definition of $\sim$, which simultaneously shows that $Φ$ is injective.
	Surjectivity follows since any element of $Ψ(A)$ is of the form $Ψ(a)$ which is the image of $[a]$.
\end{proof}


With the free tensor algebra and \cref{thm:first-iso} in hand, we are able to describe any associative algebra as a quotient of the form $\quot{\TA{V}}{I}$.

\begin{definition}
	The \textdef{dimension} $\dim A$ of a quotient algebra $A = \quot {\TA{V}} I$ is its dimension as a vector space.
	The \textdef{base dimension} of $A$ is the dimension of the underlying vector space $V$.
\end{definition}
Algebras of finite base dimension may be infinite-dimensional, as is the case for the tensor algebra itself (which is a quotient by the trivial ideal).



\subsection{Graded algebras}

Associative algebras may possess another layer of useful structure: a grading.
An example grading for the tensor algebra has already been exhibited in \cref{eqn:tensor-algebra-graded-decomposition}.
Gradings generalise the \emph{degree} or \emph{rank} of tensors or forms, and the notion of \emph{parity} (even/oddness) for functions or polynomials.

Informally, an algebra's grading provides a labelling for some of its elements, such that labels are combined simply (usually by addition) under the algebra's multiplication.

\begin{definition}
	\label{def:grading}
	An algebra $A$ is \textdef{$R$-graded} for $(R, +)$ a monoid\,\sidenote{A \textdef{monoid} is a group without the requirement of inverses; i.e., a set with an associative binary operation for which there is an identity element.} if there exists a decomposition
	\begin{align}
		A = \bigoplus_{k ∈ R} A_{k}
	\end{align}
	such that $A_{i} ⊛ A_{j} ⊆ A_{i + j}$, i.e., $a ∈ A_{i}, b ∈ A_{j} ⟹ a ⊛ b ∈ A_{i + j}$.
\end{definition}
The monoid is usually taken to additive over $\NN$ or $\ZZ$, possibly modulo some integer.
For instance, the tensor algebra $\TA{V}$ is $\NN$-graded, since if $a ∈ \TA[p]{V}$ and $b ∈ \TA[q]{V}$ then $a ⊗ b ∈ \TA[p + q]{V}$.
Indeed, $\TA{V}$ is also $\ZZ$-graded if for $k < 0$ we understand $\TA[k]{V} ≔ \qty{\mathbf{0}}$ to be the trivial vector space.
The tensor algebra is also $\ZZ_p$-graded, where $\ZZ_p ≡ \quot \ZZ {p\ZZ}$ is addition modulo any $p > 0$, since the decomposition
\begin{align}
	\TA{V} = \bigoplus_{k = 0}^{p - 1} Z_k
	\qqtext{where}
	Z_k = \bigoplus_{n = 0}^∞ \TA[k + np]{V}
		= \TA[k]{V} ⊕ \TA[(k + p)]{V} ⊕ \cdots
\end{align}
satisfies $Z_i ⊗ Z_j \subseteq Z_k$ when $k ≡ i + j \mod p$.
In particular, $\TA{V}$ is $\ZZ_2$-graded,\sidenote{
	Algebras which are $\ZZ_2$-graded are sometimes called \emph{superalgebras}, with the prefix `super-' originating from supersymmetry theory.
} and its elements admit a notion of \emph{parity}: elements of $Z_0 = \FF ⊗ \TA[2]{V} ⊗ \cdots$ are even, while elements of $Z_1 = V ⊗ \TA[3]{V} ⊗ \cdots$ are odd, with parity is respected by $⊗$ as it is for the integers.

Just as not all functions $f : \RR → \RR$ are even or odd, not all elements of a $\ZZ_2$-graded algebra are even or odd.
More generally, not all elements of a graded algebra belong to a single graded subspace.


\subsubsection{Graded quotient algebras}

A grading structure may or may not be inherited by a quotient --- in particular, not all quotients of $\TA{V}$ inherit its $\ZZ$-grading.
When reasoning about quotients of graded algebras, the following fact is useful.
\begin{lemma}
	\label{lem:quotients-commute-with-direct-sums}
	Quotients commute with direct sums, so if
	\begin{align}
		A = \bigoplus_{k ∈ R} A_k
		\qqtext{and}
		I = \bigoplus_{k ∈ R} I_k
		\qqtext{then}
		\quot A I = \bigoplus_{k ∈ R} (\quot{A_k}{I_k})
	\end{align}
	where $R$ is some index set.
\end{lemma}
\begin{proof}
	It is sufficient to prove the case for direct sums of length two.
	We then seek an isomorphism
	\begin{math}
		Φ : \quot{(A ⊕ B)}{(I ⊕ J)} → (\quot{A}{I}) ⊕ (\quot{B}{J})
	.\end{math}
	Elements of the domain are equivalence classes of pairs $[(a, b)]$ with respect to the ideal $I ⊕ J$.
	The direct sum ideal $I ⊕ J$ corresponds to the congruence defined by $(a, b) \sim (a', b') \iff a \sim a'$ and $b \sim b'$.
	Therefore, the assignment $Φ = [(a, b)] \mapsto ([a], [b])$ is well-defined.
	Injectivity and surjectivity follow immediately.
\end{proof}
The general non-preservation of gradings motivates strengthening the notion of an ideal:
\begin{definition}
	An ideal $I$ of an $R$-graded algebra $A = \bigoplus_{k ∈ R} A_k$ is \textdef{homogeneous} if $I = \bigoplus_{k ∈ R}I_k$ where $I_k = I \cap A_k$.
\end{definition}
Not all ideals are homogeneous.\sidenote{
	For example, the ideal $I = \gen{\ve_1 + \ve_2 ⊗ \ve_3}$ is distinct from $\bigoplus_{k = 0}^∞ (I \cap \TA[k]{V}) = \gen{\ve_1, \ve_2 ⊗ \ve_3}$ because the former does not contain $\spanof{\ve_1}$, while the latter does.
}
The additional requirement that an ideal be homogeneous ensures that the associated equivalence relation, as well as respecting the basic algebraic relations of \cref{def:congruence}, also preserves the grading structure.
And so, we have a `graded' analogue to \cref{thm:quotient-algebra-by-congruence}:
\begin{theorem}
	\label{thm:quotient-algebra-by-homogeneous-congruence}
	If $A$ is an $R$-graded algebra and $I$ a homogeneous ideal, then the quotient $\quot A I$ is also $R$-graded.
\end{theorem}
\begin{proof}
	By \cref{lem:quotients-commute-with-direct-sums} and the homogeneity of $I$, we have
	\begin{math}
		\quot A I = \bigoplus_{k ∈ R} (\quot{A_k}{I_k})
	.\end{math}
	Elements of $\quot{A_k}{I_k}$ are equivalence classes $[a_k]$ where the representative is of grade $k$.
	Thus, $(\quot{A_p}{I_p}) ⊛ (\quot{A_q}{I_q}) \subseteq \quot{A_{p + q}}{I_{p + q}}$ since $[a_p] ⊛ [a_q] = [a_p ⊛ a_q] = [b]$ for some $b ∈ A_{p + q}$.
	Hence, $\quot A I$ is $R$-graded.
\end{proof}



\section{The Wedge Product: Multivectors}

Perhaps the simplest (yet most useful) nontrivial quotient of the tensor algebra is the \emph{exterior algebra}, first popularised by Hermann Grassmann in 1844.
\begin{definition}
	\label{def:exterior-algebra}
	The \textdef{exterior algebra} over a vector space $V$ is
	\begin{align}
		\EA{V} ≔ \quot{\TA{V}}{ \gen{𝒖⊗𝒖}}
		\marginnote{
			$\gen{𝒖 ⊗ 𝒖} ≡ \gen{𝒖 ⊗ 𝒖 | 𝒖 ∈ V}$ is the ideal defined by $𝒖 ⊗ 𝒖 \sim 0$ for any vectors $𝒖 ∈ V$.
		}
	.\end{align}
	The product in $\EA{V}$ is called the \textdef{wedge product}, denoted $∧$.
\end{definition}
The wedge product is also called the \emph{exterior}, \emph{alternating} or \emph{antisymmetric} product.
The property suggested by its various names is easily seen by expanding the square of a sum:
\begin{align}
	(𝒖 + 𝒗)∧(𝒖 + 𝒗) = 𝒖∧𝒖 + 𝒖∧𝒗 + 𝒗∧𝒖 + 𝒗∧𝒗
.\end{align}
Since all terms of the form $𝒘∧𝒘 = 0$ are definitionally zero, we have
\begin{align}
	𝒖∧𝒗 = -𝒗∧𝒖
\end{align}
for all vectors $𝒖, 𝒗 ∈ V$.
By associativity, it follows that $𝒗_1 ∧ 𝒗_2 ∧ \cdots ∧ 𝒗_k$ vanishes exactly when the $𝒗_i$ are linearly dependent.\sidenote{
	\begin{proof}
		Blades of the form $a = \etc{𝒖_\i}∧k$ vanish when two or more vectors are repeated.
		If $\set{𝒖_i}$ is linearly dependent, then any one $𝒖_i$ can be written in terms of the others, and thus $a$ can be expanded into a sum of such vanishing terms.
	\end{proof}
}

The ideal $\gen{𝒖 ⊗ 𝒖}$ is homogeneous with respect to the $\ZZ$-grading of the parent tensor algebra,\sidenotemark\ and hence $\EA{V}$ is itself $\ZZ$-graded (by \cref{thm:quotient-algebra-by-homogeneous-congruence}).
In particular, the decomposition into fixed-grade subspaces
\begin{align}
	\EA{V} = \bigoplus_{k=0}^{\dim V} \EA[k]{V}
	\qqtext{where}
	\EA[k]{V} = \spanof{𝒗_1 ∧ 𝒗_2 ∧ \cdots ∧ 𝒗_k | 𝒗_i ∈ V}
,\end{align}
is respected by the wedge product, i.e., $(\EA[p]{V}) ∧ (\EA[q]{V}) \subseteq \EA[p + q]{V}$.
\begin{definition}
	An element of $\EA[k]{V}$ is a \textdef{(homogeneous) $k$-vector}.
	An element of $\etc{\EA[k_\i]{V}}⊕n ⊆ \EA{V}$ is an \textdef{(inhomogeneous) $\set{\etc{k_\i},n}$-multivector}.
\end{definition}
\sidenotetext{
	This follows because $\gen{𝒖 ⊗ 𝒖}$ is generated by grade $2$ elements $𝒖 ⊗ 𝒖 ∈ \TA[2]{V}$.
}
All non-zero multivectors are the sum of one or more `irreducible' elements, called \emph{blades}.
\begin{definition}
	A term of the form $\etc{𝒖_\i}∧k$ for $𝒖_i ∈ V$ is a \textdef{$k$-blade}.
\end{definition}
Note that not all $k$-vectors are blades.
For example, the bivector $𝒖_1 ∧ 𝒖_2 + 𝒖_3 ∧ 𝒖_4$ is generally not factorizable into a single $2$-blade.

% \Cref{def:homogeneous-and-multivector,def:blade} carry over directly into $\EA{V}$, so elements of $\EA[k]{V}$ are $k$-vectors, and elements of the form $\etc{𝒖_\i}∧k$ are $k$-blades.


By counting the number of possible linearly independent sets of $k$ vectors in $\dim V$ dimensions, it follows that in base dimension $\dim V = n$,
\begin{align}
	\dim\EA[k]{V} = \binom{n}{k}
,	\qqtext{and hence}
	\dim\EA{V} = 2^{n}
.\end{align}
In particular, note that $\dim \EA[k]{V} = \dim \EA[n - k]{V}$.
Elements of the one-dimensional subspace $\EA[n]{V}$ are called \textdef{pseudoscalars}.\sidenote{
	The prefix `pseudo' means $k \mapsto n - k$.
	Hence, a pseudovector is an $(n - 1)$-vector, etc.
}

Blades have direct geometric interpretations.
The bivector $𝒖 ∧ 𝒗$ is interpreted as the directed planar area spanned by the parallelogram with sides $𝒖$ and $𝒗$.
(Note that blades have no `shape'; only directed magnitude.)
Similarly, higher-grade elements represent directed \paren{hyper}volume elements spanned by parallelepipeds (see \cref{fig:bivector}).
\begin{marginfigure}
	\includefigure[\columnwidth]{bivector}
	\caption{
		Bivectors and trivectors have orientations induced by the order of the wedge product.
	}
	\label{fig:bivector}
\end{marginfigure}
In fact, any $k$-blade may be viewed as a subspace of $V$ with an oriented scalar magnitude:
\begin{definition}
	The \textdef{span} of a non-zero $k$-blade $b = \etc{𝒖_i}∧k$ is the $k$-dimensional subspace
	$\spanof b = \spanof{\etc{𝒖_\i},k}$.
	Define the span of zero to be the trivial subspace.
\end{definition}
Notably, a blade's span is independent of the particular $∧$-decomposition of the blade into vectors.
(E.g., if $\etc{𝒖_\i}∧k = \etc{𝒗_\i}∧k$ are two such decompositions, then $\spanof{𝒖_i} = \spanof{𝒗_i}$.)


\subsection{As antisymmetric tensors}
\label{sec:exterior-algebra-as-antisymmetric}

The exterior algebra may equivalently be viewed as the space of antisymmetric tensors equipped with an antisymmetrising product.
% Consider the (anti)\-sym\-metri\-sation map
Consider the map
\begin{align}
	\label{eqn:(anti)symmetriser}
	\op{Sym}^±(\etc{𝒖_\i}⊗k) = \frac1{k!} \sum_{σ ∈ S_k} (±1)^σ \etc{𝒖_{σ(\i)}}⊗k
\end{align}
where $(-1)^σ$ denotes the sign of the permutation $σ$ in the symmetric group of $k$ elements, $S_k$.
By requiring linearly, $\op{Sym}^± : \TA{V} → \TA{V}$ is defined on all tensors.
A tensor $A$ is called \textdef{symmetric} if $\op{Sym}^+(A) = A$ and \textdef{antisymmetric} if $\op{Sym}^-(A) = A$.

Denote the image $\op{Sym}^-(\TA{V})$ by $S$.
The linear map $\op{Sym}^- : \TA{V} → S$ is \emph{not} an algebra homomorphism with respect to the tensor product on $S$, since, e.g.,
\begin{align}
	\op{Sym}^-(𝒖⊗𝒗) = \frac12(𝒖⊗𝒗 - 𝒗⊗𝒖) ≠ 𝒖 ⊗ 𝒗 = \op{Sym}^-(𝒖)⊗\op{Sym}^-(𝒗)
.\end{align}
However, $\op{Sym}^-$ \emph{is} a homomorphism if we instead equip $S ≡ (S, ∧)$ with the antisymmetrising product $∧ : S × S → S$ defined by
\begin{align}
	\label{eqn:wedge-as-antisymm}
	A ∧ B ≔ \op{Sym}^-(A ⊗ B)
.\end{align}
With this algebra homomorphism, by \cref{thm:first-iso} we have
\begin{align}
	\label{eqn:iso-of-antisym}
	% \quot{\TA{V}}{\ker \op{Sym}^-} \cong \op{Sym}^-(\TA{V})
	S \cong \quot{\TA{V}}{\ker \op{Sym}^-}
.\end{align}
Furthermore, note that the kernel of $\op{Sym}^-$ consists of tensor products of linearly dependent vectors, and sums thereof,\sidenote{
	\begin{proof}
		If $A = \etc{𝒖_\i}⊗k$ where two vectors $𝒖_i = 𝒖_j$ are equal, then $\op{Sym}^-(A) = 0$ since each term in the sum in \cref{eqn:(anti)symmetriser} is paired with an equal and opposite term with $i ↔︎ j$ swapped.
		If $\set{𝒖_i}$ is linearly dependent, any one vector is a sum of the others, so $A$ is a sum of blades with at least two vectors repeated.
	\end{proof}
}
\begin{align}
	\ker \op{Sym}^- = \spanof{\etc{𝒖_\i}⊗k | k ∈ \NN, \text{$\set{𝒖_i}$ linearly dependent}}
,\end{align}
which is exactly the ideal $\gen{𝒖⊗𝒖}$.
Therefore, the right-hand side of \cref{eqn:iso-of-antisym} is identically the exterior algebra of \cref{def:exterior-algebra}.
Hence, we have an algebra isomorphism
\begin{math}
	\op{Sym}^-(\TA{V}) \cong \EA{V}
,\end{math}
where the left-hand side is equipped with the product \eqref{eqn:wedge-as-antisymm}.
This gives an alternative construction of the exterior algebra as the subalgebra of antisymmetric tensors.

\subsubsection{Note on conventions}

The factor of $\frac1{k!}$ present in \cref{eqn:(anti)symmetriser} is not necessary to derive the isomorphism
\begin{math}
	\op{Sym}^-(\TA{V}) \cong \EA{V}
.\end{math}
Indeed, some authors omit the normalisation factor, which has the effect of changing \cref{eqn:wedge-as-antisymm} to\sidenote{
	Written here with $\op{Sym}^-$ \emph{including} the factor $\frac1{k!}$, as in \eqref{eqn:(anti)symmetriser}.
}
\begin{align}
	A ∧ B = \frac{(p + q)!}{p!q!}\op{Sym}^-(A ⊗ B)
\end{align}
for $A$ and $B$ of respective grades $p$ and $q$.
These different normalisations of $∧$ lead to distinct identifications of multivectors in $\EA{V}$ with tensors in $S \subset \TA{V}$.
Both conventions are present in literature, and their differences are clarified in \cref{tbl:wedge-conventions}.
\begin{table}[h]
	\centering
	\setlength{\tabcolsep}{20pt}
	\renewcommand{\arraystretch}{1.5}
	\begin{tabular}{cc}
		\emph{Kobayashi--Nomizu} \cite{kobayashi1963dg}
	&	\emph{Spivak} \cite{spivak1975dg}
	\\	$A ∧ B ≔ \op{Sym}^-(A⊗B)$
	&	$A ∧ B ≔ \frac{(p + q)!}{p!q!} \op{Sym}^-(A⊗B)$
	\\	$𝒖 ∧ 𝒗 ≡ \frac12(𝒖⊗𝒗 - 𝒗⊗𝒖)$
	&	$𝒖 ∧ 𝒗 ≡ 𝒖⊗𝒗 - 𝒗⊗𝒖$
	\end{tabular}
	\caption{
		Different embeddings of $\EA{V}$ into $\TA{V}$.
		We employ the Kobayashi--Nomizu convention as this is coincides with the wedge product of geometric algebra.
		However, the Spivak convention is dominant for exterior differential forms in physics.
	}
	\label{tbl:wedge-conventions}
\end{table}


\subsection{Exterior forms}
\label{sec:exterior-forms}

The exterior algebra is most frequently encountered by physicists as an operation on \emph{exterior (differential) forms}, which are alternating\sidenote{
	An \textdef{alternating} linear map is one which changes sign upon transposition of any pair of arguments.
	% E.g., if $f$ is alternating, then $f(𝒖, 𝒗, 𝒘) = -f(𝒗, 𝒖, 𝒘)$.
} multilinear maps.
We \emph{could} use the exterior algebra $\EA{V^*}$ over the dual space of linear maps $V → \RR$ as a model for exterior forms, though we will not choose to do this, instead defining them separately.

To see why, consider $\EA{V^*}$ as a model for exterior forms.
Any element $f ∈ \EA[k]{V^*}$ has the form $f = f_{\etc{i_\i}{}k} \etc{\ve^{i_\i}}∧k$ for a basis $\set{\ve^i} ⊂ V^*$, and each component acts on $\etc{𝒖_\i}⊗k ∈ \TA[k]{V}$ as
\begin{align}
	(\etc{\ve^{i_\i}}∧k)(\etc{𝒖_\i}⊗k)
	&= \frac1{k!} \sum_{σ ∈ S_k} (-1)^σ\etc{\ve^{i_{σ(\i)}}(𝒖_\i)}{}k
\\	&= \frac1{k!}\det[\ve^{i_m}(𝒖_n)]_{mn}
	\label{eqn:poor-mans-exterior-form}
.\end{align}
However, this differs from the standard definition of exterior forms (as in \cite{spivak1975dg,misner1973gravitation}) in two important ways:
\begin{enumerate}
	\item In \cref{eqn:poor-mans-exterior-form}, the dual vectors $\ve^i ∈ V^*$ are permuted while the order of the arguments $𝒖_i$ are preserved; but for standard exterior forms, the opposite is true.
	This forbids the proper extension of $\EA{V^*}$ to \emph{non-Abelian vector-valued forms}, where the values $\ve^i(𝒖_j)$ may not commute.
	\item More trivially, we insist on the Kobayashi---Nomizu convention of normalisation factor for $\EA{V^*}$; but the Spivak convention for exterior forms is much more common in physics.
\end{enumerate}
For these reasons, we will not define exterior forms in terms of the exterior algebra $\EA{V^*}$, but separately, to better agree with convention.

\begin{definition}
	\label{def:exterior-form}
	For a vector space $V$ over $\FF$, a \textdef{$k$-form} $φ ∈ \forms[k](V)$ is an alternating multilinear map $φ : \TA[k]{V} → \FF$.

	For another vector space $A$, an \textdef{$A$-valued $k$-form} $φ ∈ \forms[k](V, A)$ is such a map with codomain $A$ (instead of $\FF$).
\end{definition}
The evaluation of a form is denoted $φ(\etc{𝒖_\i}⊗k)$ or $φ(\etc{𝒖_\i},k)$, and the wedge product of a $p$-form $φ$ and $q$-form $ϕ$ is defined (in the Spivak convention) as
\begin{align}
	\label{eqn:wedge-of-forms}
	φ ∧ ϕ = \frac{(p + q)!}{p!q!} (φ ⊗ ϕ) \circ \op{Sym}^-
.\end{align}
Explicitly, \cref{eqn:wedge-of-forms} acts to antisymmetrise \emph{arguments}.
To see this, choose a basis $\set{θ^μ}$ of $\forms(V)$, and compare to \cref{eqn:poor-mans-exterior-form},
\begin{align}
 	(\etc{θ^{μ_\i}}∧k)(\etc{𝒖_\i}⊗k)
	&= \sum_{σ ∈ S_k} (-1)^σ \etc{θ^{μ_\i}(𝒖_{σ(\i)})}{}k
% \\	&= \det [θ^{μ_m}(𝒖_n)]_{mn}
.\end{align}

If $φ, ϕ ∈ \forms(V, A)$ are $A$-valued forms, where $A$ is equipped with a bilinear product $⊛ : A × A → A$, then scalar multiplication may be replaced by $⊛$ so that
\begin{align}
	(φ ∧ ϕ)(\etc{𝒖_\i} ⊗ k) = \sum_{σ ∈ S_k} (-1)^σ φ(\etc{𝒖_\i} ⊗ p) ⊛ ϕ(\etc{𝒖_\i} ⊗ q)
.\end{align}
The product $⊛$ need not be commutative nor associative.
In particular, we may have Lie algebra--valued forms.
For example, if $φ, ϕ ∈ \forms[1](V, \liealg g)$, then we may take the Lie bracket $[\phantom{𝒖}, \phantom{𝒖}] : \liealg g × \liealg g → \liealg g$ to be the bilinear product on $\liealg g$, obtaining
\begin{align}
	(φ ∧ ϕ)(𝒖, 𝒗) = [φ(𝒖), ϕ(𝒗)] - [φ(𝒗), ϕ(𝒖)]
.\end{align}
Note that this implies that $φ ∧ φ$ does not necessarily vanish for non-Abelian forms.\sidenote{
	E.g., in the case above, $(φ ∧ φ)(𝒖, 𝒗) = 2[φ(𝒖), φ(𝒗)]$.
}
% \begin{align}
% 	φ ∧ ϕ = \op{Alt}(φ ⊗ ϕ)
% \end{align}
% where the alternating map
% \begin{align}
% 	\op{Alt}(f)(\etc{𝒖_\i}⊗k) = k! f(\op{Sym}^-(\etc{𝒖_\i}⊗k))
% \end{align}
% antisymmetrises arguments to the multilinear function $f$.
 

The two algebras $\EA[V]$ and $\forms(V)$ are isomorphic, but differ in that elements of $\forms(V)$ are identified as alternating maps (using the Spivak convention).
In particular, this means that non-homogeneous exterior forms are often disallowed, since they cannot be identified with multilinear maps --- while non-homogeneous elements of $\EA[V]$ are certainly allowed.\sidenote{
	Non-homogeneous elements play important roles in the geometric algebra, in which $\EA[V]$ is embedded.
}



\section{The Metric: Length and Angle}

The tensor and exterior algebras considered so far are built from a vector space $V$ alone.
Notions of length and angle are central to geometry, but are not intrinsic to a vector space --- this additional structure may be provided by a \emph{metric}.
\begin{definition}
	A \textdef{metric}\,\sidenote{a.k.a. an \emph{inner product}, or \emph{symmetric bilinear form}} is a function $η : V × V \to \FF$, often written $\ip{𝒖, 𝒗} ≡ η(𝒖, 𝒗)$, which satisfies
	\begin{itemize}
		\item symmetry, $\ip{𝒖, 𝒗} = \ip{𝒗, 𝒖}$; and
		\item linearity, $\ip{α𝒖 + β𝒗, 𝒘} = α\ip{𝒖, 𝒘} + β\ip{𝒗, 𝒘}$ for $α, β \in \FF$.
	\end{itemize}
\end{definition}

Linearity in either argument implies linearity in the other by symmetry, so $η$ is bilinear.
A metric is \textdef{non-degenerate} if
\begin{math}
	\ip{𝒖, 𝒗} = 0
\end{math}
for all $𝒖$ implies that $𝒗$ is zero.
With respect to a basis $\set{\ve_i}$ of $V$, the metric components
\begin{math}
	η_{ij} = \ip{\ve_i, \ve_j}
\end{math}
are defined.
Non-degeneracy means that $\det η ≠ 0$ when viewing $η = [η_{ij}]$ as a matrix, and in this case the matrix inverse $η^{ij}$ is also defined and satisfies $η^{ik}η_{kj} = δ^i_j$.
Throughout, we will not have need to consider degenerate metrics,\sidenote{
	Degenerate signatures do find use in computer graphics, especially via \emph{projective geometric algebra} \cite{hestenes1991pga,vince2008ga-graphics}.
} so we assume non-degeneracy.

A vector space $V$ together with a metric $η$ is called an \textdef{inner product space} $(V, η)$.
Alternatively, instead of a metric, an inner product space may be constructed with a quadratic form:
\begin{definition}
	A \textdef{quadratic form} is a function $q : V \to \FF$ satisfying
	\begin{itemize}
		\item $q(λ𝒗) = λ^2q(𝒗)$ for all $λ \in \FF$; and
		\item the requirement that the \textdef{polarization of $q$},
		\begin{align}
			(𝒖, 𝒗) \mapsto q(𝒖 + 𝒗) - q(𝒖) - q(𝒗)
		,\end{align}
		is bilinear.
	\end{itemize}
\end{definition}
To any quadratic form $q$ there is a unique associated bilinear form, which is \emph{compatible} in the sense that $q(𝒖) = \ip{𝒖,𝒖}$.
It is recovered\sidenote{
	Except, of course, if the characteristic of $\FF$ is two.
	We only consider fields of characteristic zero.
} by the \emph{polarization identity}
\begin{align}
	\ip{𝒖, 𝒗} = \frac12\qty\big(q(𝒖 + 𝒗) - q(𝒖) - q(𝒗))
.\end{align}
The prescription of either $η$ or $q$ is therefore equivalent --- but the notion of a metric is more common in physics, whereas the mathematical viewpoint often starts with a quadratic form.


\subsubsection{Covectors and dual bases}

The dual space $V^* ≔ \set{f : V → \FF | \text{$f$ linear}}$ of a vector space consists of \textdef{dual vectors} or \textdef{covectors}, which are linear maps from $V$ into its underlying field.
Summation convention dictates that components of vectors be written superscript, $𝒖 = u^i\ve_i ∈ V$, and covectors subscript, $φ = φ_i\ve^i ∈ V^*$, for bases $\set{\ve_i} \subset V$ and $\set{\ve^i} \subset V^*$.

A metric $η$ on $V$ defines an isomorphism between $V$ and its dual space.
Collectively known as the \textdef{musical isomorphisms}, the maps $\flat : V → V^*$ and its inverse $\sharp : V^* → V$ are defined by
\begin{align}
	𝒖^\flat(𝒗) = \ip{𝒖, 𝒗}
	\qqtext{and}
	\ip{φ^\sharp, 𝒖} = φ(𝒖)
\end{align}
for $𝒖,𝒗 ∈ V$ and $φ ∈ V^*$.
The names become justified when working with a basis: the relations
\begin{align}
	(𝒖^\flat)_i = η_{ij}𝒖^j
	\qqtext{and}
	(φ^\sharp)^i = η^{ij}φ_j
\end{align}
show that $\flat$ \emph{lowers} indices, while $\sharp$ \emph{raises} them.

Given a metric, a choice of basis $\set{\ve_i} ⊂ V$ also defines a \textdef{dual basis $\set{\ve^i} \subset V^*$ of $V$} via
\begin{math}
	\ve^i ≔ η^{ij}\ve_j^\flat
.\end{math}
Note that basis vectors and covectors defined in this way do not exist in the same vector space, but are related by their evaluation on one another by $\ve^i(\ve_j) = δ^i_j$.
In some contexts, we will define a dual basis $\set{\ve^i}$ in $V$ (not in $V^*$), a.k.a.\ a \textdef{reciprocal basis}, and by this we mean
\begin{math}
	\ve^i ≔ η^{ij}\ve_j
.\end{math}
Then, dual and non-dual basis vectors are related via $\ip{\ve^i, \ve_j} = δ^i_j$.

We use both senses of the term ``dual basis''.
In particular, $V^*$ is never needed in the geometric algebra; its role is filled by reciprocal bases.
Often, the distinction can be safely ignored (since, after all, $V \cong V^*$).

%  or $\ve^i = η^{ij}\ve_j$.

\subsection{Metrical exterior algebra}
\label{sec:metrical-exterior-alg}

In an exterior algebra $\EA{V}$ with a metric defined on $V$, there is an induced metric on $k$-vectors defined by
\begin{align}
	\ip{\etc{𝒖_\i}∧k, \etc{𝒗_\i}∧k}
	&= \sum_{σ ∈ S_k} (-1)^σ \etc{\ip{𝒖_\i, 𝒗_{σ(\i)}}}{}k
\\	&= \det [\ip{𝒖_m, 𝒖_n}]_{mn}
.\end{align}
In particular, a metric on $\EA{V}$ defines a magnitude for pseudoscalars.
\begin{definition}
	Let $V$ be an $n$-dimensional vector space with a metric.
	There are two volume elements $\vol ∈ \EA[n]{V}$ of the metrical exterior algebra $\EA{V}$ is a pseudoscalar satisfying $\ip{\vol, \vol} = 1$, each differing by sign.

	A choice of \textdef{orientation} is a choice of volume element $\vol$.
\end{definition}
Given an ordered orthonormal basis $\set{\ve_i}$ with $\ip{\ve_i, \ve_i} = ±1$, the basis is called right-handed if $\etc{\ve_\i}∧n = \vol$ is the chosen volume element, and left-handed otherwise.


\subsubsection{Hodge--dual multivectors}

\marginnote{
	\emph{Hodge duality:} \cite{flanders1963differential,misner1973gravitation}, \cite[§16]{lee2012diffgeo}.

}[-4.1ex]

A useful duality operation can be defined in an exterior algebra $\EA{V}$ with a metric and orientation, which relates the $k$- and $(n - k)$-grade subspaces.
\begin{definition}
	\label{def:hodge-dual}
	Let $\EA{V}$ be a metrical exterior algebra with base dimension $n$ and volume element $\vol$.
	The \textdef{Hodge dual} $\star$ is the unique linear operator satisfying
	\begin{align}
		\label{eqn:hodge-dual}
		A ∧ \star B = \ip{A, B}\vol
	\end{align}
	for any $k$-vectors $A, B ∈ \EA[k]{V}$.
\end{definition}

The Hodge dual $\star : \EA[k]{V} → \EA[n - k]{V}$ defines an isomorphism between pairs of fixed-grade subspaces of the same dimension; in particular, scalars with pseudoscalars via $\star 1 = \vol$.

\begin{lemma}
	The hodge dual of a $p$-vector $A = A^{i_1⋯i_p} \etc{\ve_{i_\i}}∧p$ in $n$ dimensions has components
	\begin{align}
		(\star A)^{j_1⋯j_q} = \frac1{p!} A_{i_1⋯i_p} ε^{i_1⋯i_p j_1⋯j_q}
		\marginnote{
			The \textdef{Levi-Civita symbol} $ε^{i_1⋯i_k}$ is the sign of the permutation $(1, ..., k) ↦ (i_1, ..., i_k)$ if it exists, and zero otherwise.
		}
	\end{align}
	where $q = n - p$ and $A_{i_1⋯i_p} = A^{k_1⋯k_p} \etc{η_{i_\i k_\i}}{}p$.
\end{lemma}
\begin{proof}
	Begin by writing the left-hand side of \cref{eqn:hodge-dual} in component form,
	\begin{align}
		\label{eqn:hodge-dual-lem.1}
		A ∧ \star B
		&= A^{i_1⋯i_p} \, (\star B)^{j_1⋯j_p} \, \etc{\ve_{i_\i}}∧p ∧ \etc{\ve_{j_\i}}∧p
	\\	&= A^{i_1⋯i_p} \, (\star B)^{j_1⋯j_p} \, ε_{i_1⋯i_p j_1⋯j_p} \vol
	.\end{align}
	If the multi-indices $i_k$ and $j_k = i_{σ(i)}$ are permutations of each other, then
	\begin{align}
		\det[η_{i_m j_n}]_{mn} = (-1)^σ \etc{η_{i_\i j_{σ(\i)}}}{}p
	.\end{align}
	Using this, the right-hand side of \cref{eqn:hodge-dual} is
	\begin{align}
		\label{eqn:hodge-dual-lem.2}
		\ip{A,B} \vol
		&= A^{i_1⋯i_p} B^{j_1⋯j_p} \det[ η_{i_m j_n}]_{mn} \vol
	\\	&= A^{i_1⋯i_p} B^{j_1⋯j_p} (-1)^σ \etc{η_{i_\i j_{σ(\i)}}}{}p \vol
	\\	&= A^{i_1⋯i_p} B_{\etc{i_{σ^{-1}(\i)}}{}p} (-1)^σ \vol
		= A^{i_1⋯i_p} B_{i_1⋯i_p} \vol
	\end{align}
	where we absorb the factor $(-1)^σ$ by permuting indices of $B$.
	Equating coefficients of the arbitrary $A^{i_1⋯i_p}$ components, \cref{eqn:hodge-dual-lem.1,eqn:hodge-dual-lem.2} gives
	\begin{align}
		(\star B)^{j_1⋯j_p} ε_{i_1⋯i_p j_1⋯j_p} = B_{i_1⋯i_p}
	\end{align}
	Contracting with $ε_{i_1⋯i_p k_1⋯k_p}$ gives
	\begin{align}
		(\star B)^{k_1⋯k_p} = \frac1{p!} B_{i_1⋯i_p} ε^{i_1⋯i_p k_1⋯k_p}
	\end{align}
	since $ε_{i_1⋯i_p j_1⋯j_p} ε^{i_1⋯i_p k_1⋯k_p} = (-1)^σ p!$ where $σ$ is the permutation sending $j_i \mapsto k_i$.
	The factor of $(-1)^σ$ is absorbed since $(\star B)^{j_1⋯j_p} = (-1)^σ (\star B)^{k_1⋯k_p}$.
	Finally, replacing $A \leftrightarrow B$ is the result.
\end{proof}
