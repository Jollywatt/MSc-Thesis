\chapter{Spacetime as a Manifold}



The investigations of \cref{part:1} were restricted to \emph{flat geometries}.
In particular, special relativity models spacetime as a homogeneous, isotropic Minkowski vector space.
However, the general theory of relativity incorporates gravity as the curvature of space itself.
Thus, spacetime no longer has an intrinsic vector space structure where the `difference between points' has physical meaning.
The mathematical demands of curvature call for the \emph{differential geometry} of manifolds.




Here we only give a pragmatic definition of a manifold as a space which locally looks like $\RR^n$ upon which one can do calculus.
(A rigorous definition in terms of charts and atlases can be found in the first chapter of \cite{lee2012diffgeo}.)
\begin{definition}
	\label{def:manifold}
	A \textdef{manifold $\manif M$ of dimension $n$} is a nice\sidenote{
		Here, a `nice' topological space is:
		\begin{enumerate}[leftmargin=1.3em]
		\item \emph{Hausdorff}, meaning each distinct pair of points have mutually disjoint neighbourhoods (so it is ``not too small''); and
		\item \emph{second-countable}, meaning there exists a countable base (so it is ``not too large'').
		\end{enumerate}
	} topological space which is locally Euclidean, meaning for every $x ∈ \manif M$ there exist neighbourhoods $x ∈ \manif U \subseteq \manif M$ and subsets $U \subseteq \RR^n$ with a homeomorphism (continuous bijection) $\manif U \biject U$ between them.

	A \textdef{smooth manifold} is a manifold with the stricter requirement that $\manif U \biject U$ be a diffeomorphism (differentiable bijection).
\end{definition}


Essentially, \cref{def:manifold} is designed to guarantee that well-behaved local coordinates (corresponding to elements of $U$) always exist.
% Spherical surfaces, closed curves and spaces such as $\SO(n)$ are smooth manifolds, but cubes 
\begin{definition}
	\label{def:coord-chart}
	Let $\manif M$ be an $n$-dimensional manifold.
	A \textdef{(global) coordinate chart $\set{x^i} ≡ \set{\etc{x^\i},n}$} of $ℳ$ is a set of scalar fields $x^i : \manif M → \RR$ such that each point in $ℳ$ is specified uniquely by the coordinate values $(\etc{x^\i},n) ∈ \RR^n$.

	A \textdef{local coordinate chart} about a point $x ∈ \manif M$ is a coordinate chart of a neighbourhood of $x$.
\end{definition}
In other words, the coordinate functions $\set{x^i}$ specify a homeomorphism $ℳ → \RR^n$, or a diffeomorphism in the case of a smooth chart.
We will often call a point $x ∈ \manif M$ by the same symbol as the local coordinates $x^i : \manif M → \RR$ without the index --- but these objects are not interchangeable.

A structure-preserving map between manifolds is a continuous function; and between smooth manifolds, a differentiable function.
For brevity, we assume the definitions that follow take place in the category of manifolds, and \emph{take all maps between manifolds to be continuous.}
Furthermore, if the qualifier ``smooth'' is present, we operate in the category of smooth manifolds and such maps are assumed differentiable.
Thus, the coordinate scalars $x^i$ of \cref{def:coord-chart} are continuous functions, and are differentiable if the manifold is smooth.



\section{Derivatives of Smooth Maps}
\label{sec:differential}

Manifolds themselves do not have inherent vector space structure.
However, being locally Euclidean means there is a real vector space naturally associated to each point:
\begin{definition}
	\label{def:tangent-space-bundle}
	The \textdef{tangent space $\TT_x\manif M$} of a manifold at a point $x ∈ \manif M$ is the vector space of derivations on smooth functions at that point.\sidenote{
		More precisely, each vector $𝒖 ∈ \TT_x\manif M$ is an equivalence class of derivatives evaluated at the point $x$, where different derivations which agree at the point $x$ are identified.
	}
	In any local coordinate chart $\set{x^i}_{i=1}^n$ of $\manif M$ containing $x$, this is
	\begin{align}
		\TT_x\manif M ≅ \spanof{\eval{\pdv{x^i}}_x}_{i=1}^n
	.\end{align}
	The \textdef{tangent bundle $\TT\manif M$} is the disjoint union of all tangent spaces
	\begin{align}
		\TT\manif M = \set{(x, 𝒖) | x ∈ \manif M, 𝒖 ∈ \TT_x\manif M}
	\end{align}
	equipped with an appropriate manifold topology.\sidenote{
		Specifically, the topology of a fibre bundle (see \cref{cha:fibre-bundles}).
	}
\end{definition}


Given a smooth manifold, its tangent bundle comes for free: its construction is canonical and requires no additional data.
Similarly, given a smooth function $f$ between manifolds, its derivative $\dd f$ (i.e., its `tangent') also comes for free.

In the same way that the tangent bundle consists of `directional derivatives of points' in the manifold (i.e., tangent vectors), the differential $\dd f$ encodes the derivative of $f$ at each point in all directions.\sidenote{
	This parallel is precise: $\dd$ and $\TT$ form a functor in category of smooth manifolds, sending
	\begin{math}
		f: ℳ → 𝒩
	\end{math}
	to
	\begin{math}
		\dd f: \TTℳ → \TT𝒩
	.\end{math}
	Some authors use the symbol $\TT$ for both.
}
Intuitively, if $𝒖 ∈ \TT_x\manif M$ is a vector at a point $x ∈ \manif M$, then the vector $\dd f(𝒖) ∈ \TT_{f(x)}\manif N$ is interpreted as the derivative of $f(x) ∈ \manif N$ in the direction $𝒖$.




\begin{definition}
	\label{def:differential}
	The \textdef{differential} or \textdef{push forward} of a map $f : \manif M \to \manif N$ between smooth manifolds is the map $\dd f : \TT\manif M \to \TT\manif N$ defined by
	\begin{align}
		\label{eqn:differential}
		\qty(\dd f(𝒖))(φ)\big|_{f(x)} ≔ 𝒖(φ \circ f)\big|_x
	\end{align}
	for each point $x ∈ \manif M$, vector $𝒖 ∈ \TT_x\manif M$ and smooth function $φ : \manif N → \RR$.
\end{definition}

In the definition above, vectors act on scalar functions as derivations; hence $\dd f(𝒖)$ is defined by its action on an arbitrary scalar field.

Note that $\dd f(𝒖)$ is not always defined everywhere.
If $𝒖|_x ∈ \TT_x\manif M$ is now a family of vectors defined everywhere over $x ∈ ℳ$, then $\dd f(𝒖)|_{f(x)} = \dd f(𝒖|_x)$ is defined only at each $f(x) ∈ \manif N$.
This means that if $f$ fails to be surjective, then $\dd f(𝒖)$ is not defined at those points lying outside the image $f(\manif M) \subset \manif N$.
Likewise, if $f$ fails to be injective at a point $y ∈ \manif N$, then $\dd f(𝒖)$ is \emph{multivalued} at $y$.
Only if $f$ is bijective does $\dd f(𝒖)|_y$ have a single value everywhere.



The meaning of \cref{def:differential} may become clearer when expressed in coordinates.
Suppose $\set{x^i}$ is a local chart of $\manif M$ containing a point $x ∈ \manif M$, and $\set{y^j}$ a chart of $\manif N$ containing $f(x)$.
With the associated coordinate bases
\begin{math}
	\TT_x\manif M = \spanof{\pdv{x^i}}
	\text{ and }
	\TT_{f(x)}\manif N = \spanof{\pdv{y^j}}
,\end{math}
\cref{eqn:differential} takes the full form:
\begin{align}%autonum: was align*
	\qty[\dd f\qty(u^i\pdv{x^i})]^j \eval{\pdv{φ}{y^j}}_{f(x)} &= u^i\eval{\pdv{φ \circ  f}{x^i}}_x
	= u^i\eval{\pdv{y^j \circ  f}{x^i}}_x \eval{\pdv{φ}{y^j}}_{f(x)}
\end{align}%autonum: was align*
The first equality is the definition itself, and the second is an application of the chain rule.
Since $φ$ is an arbitrary smooth function, this holds as an equation of differential operators, and we may remove reference to any particular $φ$ on which the operators act.
\begin{align}
	\label{eqn:differntial-coordinate-form.step1}
	\qty[\dd f\qty(u^i ∂_i)]^j\eval{∂_j}_{f(x)} = u^i \eval{\pdv{f^j}{x^i}}_x \eval{∂_j}_{f(x)}
\end{align}
We have reduced typographical complexity with $∂_i ≔ \pdv{x^i}$ and $∂_j ≔ \pdv{y^j}$, being aware that these are basis vectors of \emph{different} tangent spaces.
We also abbreviate $f^j ≔ y^j \circ  f$ so that $f^j(x)$ is the $j$th coordinate of the point $f(x)$ in the $y^j$ chart.
Thus, the coordinate form of $\dd f$ is precisely the Jacobian matrix,
\begin{align}
	[\dd f(∂_i)]^j = \pdv{f^j}{x^i}
.\end{align}
The point $x$ being arbitrary, we have also suppressed the evaluation signs, with the understanding that the Jacobian maps vectors at $x$ to vectors at $f(x)$.

Turning back to \cref{eqn:differntial-coordinate-form.step1}, the partial derivatives $\pdv*{x^i}$ act on smooth functions $f^j : \manif M → \RR$ to produce smooth functions $\pdv*{f^j}{x^i} : \manif M → \RR$.
However, since we have an intuitive picture of the directional derivative of the {point} $f(x)$ as $x$ is displaced, it is useful to formally extend the notation $\pdv*{x^i}$ so that we may write the partial derivative of a mapping of \emph{points} $f : \manif M → \manif N$.
\begin{marginfigure}
	\centering
	\includefigure[0.8\columnwidth]{derivative-of-point}
	\caption{
		The derivative of the point $x ∈ ℳ$ along the direction of increasing $x^μ$ is a tangent vector $∂_μx ∈ \TT_xℳ$.
		The vector is tangent to the dotted line, along which all coordinates but $x^μ$ are constant. 
	}
	\label{fig:derivative-of-point}
\end{marginfigure}
Semantically, we understand $\pdv{f}{x^i}\big|_x ∈ \TT_{f(x)}\manif N$ to be the infinitesimal displacement vector of the destination point $f(x) ∈ \manif N$ caused by an infinitesimal variation in the $i$th coordinate of the source point $x$.
This is precisely the meaning of the last term in \cref{eqn:differntial-coordinate-form.step1}, so the desired shorthand is
\begin{align}
	\label{eqn:pdv-of-map-notation}
	\pdv{f}{x^i} ≔ \pdv{f^j}{x^i}∂_j
	\qqtext{or, in full,}
	\eval{\pdv{f}{x^i}}_x ≔ \eval{\pdv{y^i \circ  f}{x^i}}_x \eval{\pdv{y^j}}_{f(x)}
.\end{align}
With this, \cref{eqn:differntial-coordinate-form.step1} may be written as
\begin{align}
	\label{eqn:differential-succinct}
	\dd f(𝒖) = u^i\pdv{f}{x^i}
.\end{align}
This condensed form is perhaps too implicit for some purposes, with the notation $\pdv*{f}{x^i}$ doing the work of \cref{eqn:pdv-of-map-notation}.
However, it is nonetheless useful: take for instance the coordinate functions $x^i : \manif M → \RR$ regarded as maps between manifolds.
Then \cref{eqn:differential-succinct} yields the defining property of the coordinate dual basis,
\begin{align}
	\dd x^i(∂_j) = \pdv{x^i}{x^j} = δ^i_j
,\end{align}
where we have identified the one-dimensional vector space $\TT_{x^i}\RR$ with $\RR$ itself.





\begin{lemma}[Chain rule]
	\label{lem:differential-chain-rule}
	If $f\circ g$ is a composition of maps between smooth manifolds, then
	\begin{align}
		\dd (f \circ g) = \dd f \circ \dd g
	.\end{align}
\end{lemma}
\begin{proof}
	Acting on a vector $𝒖$ and applying the forward-pushed vector to a scalar field $φ$, we obtain
	\begin{align}
		(\dd (f \circ g)(𝒖))(φ)
		&= 𝒖(φ \circ f \circ g)
	\\	= 𝒖((φ \circ f) \circ g)
		&= (\dd g (𝒖))(φ \circ f)
		= \dd f(\dd g (𝒖))(φ)
	\end{align}
	by three applications of \cref{def:differential}.
\end{proof}



\section{Fibre Bundles}
\label{cha:fibre-bundles}

\begin{marginfigure}
	\centering
	\includefigure[0.8\columnwidth]{sphere}
	\caption{
		Vectors in different tangent spaces, and their basis-dependent representation as an $\RR^2$-valued field.
	}
	\label{fig:ball}
\end{marginfigure}

In flat geometries, fields were modelled as functions into a fixed vector space.
For example, in flat spacetime $ℳ = \RR^{1+3}$, the electromagnetic bivector $F : ℳ → \EA[2]{\RR^4}$ makes no distinction between the vector space $\EA[2]{\RR^4}$ evaluated at one point in spacetime over another.
This would suggest that all values of a field are directly comparable, making expressions like $F(x) + F(y) ∈ \EA[2]{\RR^4}$ geometrically meaningful for different points $x,y ∈ ℳ$.
However, these kinds of expressions are ill-defined for general smooth manifolds $ℳ$, since they depend on the way tangent spaces are chosen.
Instead, it is beneficial to distinguish between codomains \emph{at each point in the domain}, and treat $F(x)$ and $F(y)$ as belonging to different spaces.

This can be motivated with the simple example of a fluid flowing on a sphere.
The instantaneous fluid velocity at a point is a vector lying in the sphere's tangent plane at that point.
If the fluid flow is given as a map $f : \Sphere^2 → \RR^2$, then any two velocity vectors exist in the ``same'' space, even when \emph{geometrically} they do not (\cref{fig:ball}).
This is more than a purely philosophical point: the fluid flow's representation as a field $f : \Sphere^2 → \RR^2$ is \emph{dependent on the choice of basis}.
That is, $f$ depends on the way in which the single codomain $\RR^2$ is identified with each tangent plane on the sphere, and there is no such canonical choice for the sphere.
We would do better with a more geometrical representation of the vector field which is independent of any choice of basis.
This requires viewing the fluid velocities at different points as existing in different spaces.

From this we construct the tangent \emph{bundle} $\TT \Sphere^2$, where all the tangent planes of $\Sphere^2$ are collected in a disjoint union forming a \emph{bulk}.
The vector field on the sphere now becomes a \emph{section} of $\TT \Sphere^2$, which is a map $f : \Sphere^2 → \TT \Sphere^2$ such that $f(x)$ belongs to the tangent space rooted at $x$.
No longer is the expression $f(x) + f(y)$ well-defined.

The tangent bundle is a special case of a \emph{fibre bundle}, which is a manifold consisting of disjoint copies of a space (called the \emph{fibre}) taken at every point in a base manifold.



\begin{marginfigure}
	\includefigure[\columnwidth]{fibre-bundle}
	\caption{
		(a) A field $f : ℳ → F$, where values at any point can be compared.
		(b) A fibre bundle $\fibrebundle F ℱ ℳ$ with a section $f ∈ \secs(ℱ)$ whose individual fibres $F$ are labelled by base point in $ℳ$.
	}
\end{marginfigure}


\begin{definition}
	\label{def:fibre-bundle}
	A \textdef{fibre bundle} $\fibrebundle[π] F ℱ ℳ$ consists of
	\begin{itemize}
		% \item a \textdef{fibre manifold} $A$;
		\item a \textdef{bulk manifold} $ℱ$;
		\item a \textdef{base manifold} $ℳ$; and
		\item a surjection $π : ℱ → ℳ$, the \textdef{projection}, such that
		\item the inverse image $F_x ≔ π^{-1}(x)$ of a base point $x ∈ ℳ$ is homeomorphic to the \textdef{fibre} $F$.
	\end{itemize}
\end{definition}

\Cref{def:fibre-bundle} takes place in the category of manifolds, so the projection $π : ℱ → ℳ$ is continuous.
In a \textdef{smooth fibre bundle}, the projection $π$ is differentiable and $F$, $ℱ$ and $ℳ$ are all smooth manifolds.

Many different kinds of fibre bundle may be considered by giving $F$ more structure.
For example,
\begin{itemize}
	\item a \textdef{vector bundle} is one where the fibre is a vector space;
	\item a \textdef{principle bundle} is one where the fibre is a group (usually a Lie group); and
	\item an \textdef{algebra bundle} is a vector bundle where each fibre is equipped with a (smoothly varying) algebraic product; and so on.
\end{itemize}


\subsubsection{Trivialisations and coordinates}

The bulk $ℱ$ of a fibre bundle $\fibrebundle F ℱ ℳ$ is itself a manifold (of dimension $\dim ℱ = \dim ℳ + \dim F$) so we may always prescribe local coordinates on $ℱ$.
If we already have coordinates $\set{x^μ}$ on the base $ℳ$ and $\set{x^a}$ on a fibre $F$, then we often want to use the same coordinates $\set{x^μ, x^a}$ to describe the bulk $ℱ$.
This first requires a way of continuously splitting the bulk $ℱ → ℳ × F$ into its base and fibre ``components'', in a way which respects the fibred structure of the bundle.
This splitting is known as a \emph{trivialisation} of the bundle.
\begin{definition}
	A \textdef{trivialisation} of a fibre bundle $\fibrebundle[π] F ℱ ℳ$ is a homeomorphism $φ : ℱ → ℳ × F$ such that
	\begin{math}
		\op{pr}_1 ∘ φ = π
	.\end{math}
\end{definition}
It is not always possible to find a global trivialisation of a fibre bundle, and if it is, the bundle is called a \textdef{trivial fibre bundle} and there may be different possible trivialisations.\sidenote{
	A simple non-trivial fibre bundle is the Möbius strip, viewed as a bundle over the circle $\Sphere^1$ with fibre $[0, 1]$.
	The trivial bundle $\Sphere^1 × [0, 1]$ describes a strip without a twist.
}

However, it is always possible trivialise \emph{locally}.
That is, for any base point $x ∈ ℳ$, there exists a neighbourhood $x ∈ U ⊆ ℳ$ for which the subbundle $\fibrebundle[π] F {π^{-1}(U)} U$ admits a (global) trivialisation.
Hence, it is always possible to assign \emph{local} coordinates $\set{x^μ, x^a}$ to the bulk of a fibre bundle, where $x^μ$ are coordinates on the base and $x^a$ are coordinates on the fibres, such that $x^μ$ do not vary along the fibres.
In other words, local trivialisations are equivalent to local coordinates.







\subsubsection{Sections of fibre bundles}


In the language of fibre bundles, a field $f : ℳ → F$ is replaced by a \emph{section}, which is a ``vertical'' map $f : ℳ → ℱ$ into the bulk $ℱ$ such that $f(x) ∈ F_x$.
\begin{definition}
	A \textdef{section} $f$ of a fibre bundle $\fibrebundle[π] F ℱ ℳ$ is a right-inverse of $π$.
	The space of sections is denoted
	\begin{align}
		\secs(ℱ) = \set{f : ℳ → ℱ | π∘f = \op{id}}
	.\end{align}
\end{definition}
(Again, sections $f ∈ \secs(ℱ)$ are assumed continuous, and \textdef{smooth sections} are sections of smooth fibre bundles for which $f$ is smooth.)


For example, the instantaneous fluid velocity $𝒖$ on a sphere $\Sphere^2$ is a section $𝒖 ∈ \secs(\TT\Sphere^2)$ of the tangent bundle, with a single vector at $x ∈ \Sphere^2$ is denoted $𝒖|_x ∈ \TT_x\Sphere^2$.




\subsection{Exterior differential forms on manifolds}

\Cref{sec:exterior-forms} defined exterior forms $\forms(V, A)$ as alternating multilinear maps from the fixed vector space $V$ into $A$.
Similarly, the differential of a map $\dd f : \TT ℳ → \TT 𝒩$ is an object that takes a vector argument $𝒖 ∈ \TTℳ$ --- just like an exterior $1$-form, except that the entire tangent bundle $\TT ℳ$ is not itself a vector space.

Exterior forms, which are alternating maps from a fixed space $\TA{V}$, can be extended to exterior \emph{differential} forms, which exist on manifolds and define alternating maps from $\TA{(\TT_x ℳ)}$ at each $x ∈ ℳ$.

Although the entire bundle $\TT ℳ$ is not a vector space, the space of vector sections $\secs(\TT ℳ)$ is.
Hence, when viewed as a map $\dd f : \secs(\TT ℳ) → \secs(\TT 𝒩)$ the differential $\dd f ∈ \forms[1](\secs(\TT ℳ), \secs(\TT 𝒩))$ is a $\secs(\TT 𝒩)$-valued exterior $1$-form (by \cref{def:exterior-form}).
This mouthful may be eased by defining the notation
\begin{align}
	\forms(ℳ, \manif E) ≔ \forms(\secs(\TT ℳ), \secs(\manif E))
\end{align}
for some vector bundle $\fibrebundle ℳ {\manif E} V$.
As with exterior forms, the wedge product is defined as in \cref{eqn:wedge-of-forms}.

An element of $\forms[k](ℳ, \manif E)$ is called an \textdef{$\manif E$-valued exterior differential $k$-form}, where `differential' distinguishes it as an object on a manifold.
Scalar-valued exterior differential forms are elements of $\forms[k](ℳ) ≔ \forms[k](ℳ, ℳ × \RR)$, where $\fibrebundle[π] ℳ {ℳ × \RR} \RR$ is the trivial line bundle with projection $π(x, λ) = λ$.


We sometimes use the notation $\df α$ to emphasise that $α$ is an exterior differential form.
\toself{Not necessary for differentials $\dd f$ since is clear.}


\todo{Formula for exterior derivative.}


\clearpage
\section{Tangent Bundles and Lie Derivatives}

\todo{
	Explain that objects built from the tangent bundle admit a notion of derivative independently of a connection --- via the Lie derivative.
	Show the Lie derivative for tensors,
	\begin{align}
		\lie_𝒖 T^{μ_1...μ_p}{}_{ν_1...ν_q}
		= u^λ ∂_λ T^{μ_1...μ_p}{}_{ν_1...ν_q}
		+ \sum_{i=1}^p T^{μ_1...λ...μ_p}{}_{ν_1...ν_q} ∂_λ u^{μ_i}
		+ \sum_{i=1}^q T^{μ_1...μ_p}{}_{ν_1...λ...ν_q} ∂_{ν_i} u^λ
	\end{align}
	and for differential forms via Cartan's magic formula
	\begin{align}
		\lie_𝒖 φ = 𝒖 \lcontr \dd φ + \dd(𝒖 \lcontr φ)
	.\end{align}
	The ``complexity'' of these formulae sets the stage for how amazingly simple everything becomes with geometric calculus...
}

\subsection{The geometric Lie bracket and derivative}

\begin{definition}
	\label{def:geometric-lie-bracket}
	The \textdef{geometric Lie bracket} of two multivectors $A, B ∈ \GA(ℳ, η)$ is
	\begin{align}
		[A, B] ≔ (A \rcontr \vd) ∧ B - (B \rcontr \vd) ∧ A
	.\end{align}
\end{definition}

\begin{theorem}
	\label{thm:geometric-lie-derivative}
	Let $A ∈ \GA(\TT ℳ, η)$ be a multivector and $𝒖 ∈ \TT ℳ$ a tangent vector.
	The Lie derivative of $A$ along $𝒖$ is
	\begin{align}
		\label{eqn:geometric-lie-derivative}
		\lie_𝒖 A = [𝒖, A]
	.\end{align}
\end{theorem}
This is a remarkably elegant result: the multivector $A$ can be of any kind (e.g., a vector, $k$-blade, even an inhomogeneous rotor) and the Lie derivative has the same form.
\todo{Contrast this to the Lie derivative of a general tensor, eq.~(?).}
\begin{proof}
	Since $\lie_𝒖$ is linear, it suffices to prove the case where $A = \etc{𝒂_\i}∧k$ is a $k$-blade.
	Because $\lie_𝒖$ is a derivation, we must have the result that
	\begin{align}
		\label{eqn:geolieder.0}
		\lie_𝒖 (\etc{𝒂_\i}∧k) = \sum_{i=1}^k \etcmid{𝒂_\i}{[𝒖, 𝒂_i]}∧k
	\end{align}
	since $\lie_𝒖 𝒂_i = [𝒖, 𝒂_i]$ is the vector Lie bracket.
	Expanding the right-hand side of \cref{eqn:geometric-lie-derivative}, we have, by \cref{def:geometric-lie-bracket}
	\begin{align}
		[𝒖, A] = 𝒖 \fatdot \vd A - (A \rcontr \vd) ∧ 𝒖
	.\end{align}
	We will expand the two terms on the right-hand side.
	The first is
	\begin{align}
		\label{eqn:geolieder.1}
		𝒖 \fatdot \vd A = 𝒖 \fatdot \vd (\etc{𝒂_\i}∧k)
		= \sum_{i=1}^k \etcmid{𝒂_\i}{𝒖 \fatdot \vd 𝒂_i}∧k
	\end{align}
	since $𝒖 \fatdot \vd \equiv ∂_𝒖$ is a scalar derivation.

	The second is
	\begin{math}
		(A \rcontr \vd) ∧ 𝒖
	.\end{math}
	Recall that contraction by a vector is an anti-derivation \todo{show this}.
	Thus, for some vector $𝒗$,
	\begin{align}
		𝒗 \lcontr A = 𝒗 \lcontr (\etc{𝒂_\i}∧k)
		= \sum_{i=1}^k (-1)^{i-1} \etcmid{𝒂_\i}{𝒗 \fatdot 𝒂_i}∧k
	.\end{align}
	Wedging this with a vector $𝒖$ produces
	\begin{align}
		\label{eqn:geolieder.2}
		𝒖 ∧ (𝒗 \lcontr A)
		= \sum_{i=1}^k \etcmid{𝒂_\i}{(𝒂_i \fatdot 𝒗)𝒖}∧k
	,\end{align}
	where the factor of $(-1)^{i-1}$ is cancelled by anticommuting $𝒖$ to the $i$th position.
	Now, note that $A$, $𝒗 \lcontr A$ and $𝒖 ∧ (𝒗 \lcontr A)$ are of grades $k$, $k - 1$ and $k$, respectively, allowing us to exploit reversion to rewrite as
	\begin{align}
		\label{eqn:geolieder.3}
		𝒖 ∧ (𝒗 \lcontr A)
		= \revsign{k} \, \rev{(𝒗 \lcontr A)} ∧ \rev{𝒖}
		= \revsign{k} \, (\rev{A} \rcontr \rev{𝒗}) ∧ 𝒖
		= (A \rcontr 𝒗) ∧ 𝒖
	.\end{align}
	The notation on the right-hand side lends itself better to the case where $𝒗$ is instead the vector derivative $\vd$ acting on $𝒖$, since $𝒖$ is to its immediate right.
	Thus, \cref{eqn:geolieder.2,eqn:geolieder.3} show that
	\begin{align}
		\label{eqn:geolieder.4}
		(A \rcontr \vd) ∧ 𝒖
		= \sum_{i=1}^k \etcmid{𝒂_\i}{(𝒂_i \fatdot \vd)𝒖}∧k
	.\end{align}
	Combining \cref{eqn:geolieder.1,eqn:geolieder.4}, be obtain
	\begin{samepage}
	\begin{fullwidth}
	\begin{align}
		[𝒖, A] =
		𝒖 \fatdot \vd A - (A \rcontr \vd) ∧ 𝒖
		= \sum_{i=1}^k \etcmid{𝒂_\i}{(𝒖 \fatdot \vd 𝒂_i - 𝒂_i \fatdot \vd 𝒖)}∧k
	\end{align}
	\end{fullwidth}
	\end{samepage}
	whose right-hand side is equal to \cref{eqn:geolieder.0}.
\end{proof}
\todo{Disambiguate $𝒖 \fatdot \vd A = (𝒖 \fatdot \vd) A$ from $𝒖 \fatdot (\vd A)$.}