\chapter{Spacetime as a Manifold}






Here we only give a pragmatic definition of a manifold as a space which locally looks like $\RR^n$ upon which one can do calculus.
(More rigour can be found in the first chapter of \cite{lee2012diffgeo}.)
\begin{definition}
	\label{def:manifold}
	A \textdef{manifold $\manif M$ of dimension $n$} is a nice\sidenote{
		Here, a `nice' topological space is:
		\begin{enumerate}[leftmargin=1.3em]
		\item \emph{Hausdorff}, meaning each distinct pair of points have mutually disjoint neighbourhoods (so it is ``not too small''); and
		\item \emph{second-countable}, meaning there exists a countable base (so it is ``not too large'').
		\end{enumerate}
	} topological space which is locally Euclidean, meaning for every $x ∈ \manif M$ there exist neighbourhoods $x ∈ \manif U \subseteq \manif M$ and subsets $U \subseteq \RR^n$ with a homeomorphism (continuous bijection) $\manif U \biject U$ between them.

	A \textdef{smooth manifold} is a manifold with the stricter requirement that $\manif U \biject U$ be a diffeomorphism (differentiable bijection).
\end{definition}

The definitions that follow take place in the category of manifolds.
Furthermore, if the qualifier ``smooth'' is present, then the objects exist in the category of \emph{smooth manifolds} in which all maps are smooth (i.e., infinitely differentiable).
This means all maps between manifolds are \emph{assumed to be continuous}.

Essentially, \cref{def:manifold} is designed to guarantee that well-behaved local coordinates always exist.
\begin{definition}
	Let $\manif M$ be an $n$-dimensional manifold.
	A \textdef{(global) coordinate chart $\set{x^i} ≡ \set{\etc{x^\i},n}$} of $ℳ$ is a set of scalar fields $x^i : \manif M → \RR$ such that each point in $ℳ$ is specified uniquely by the coordinate values $(x^1, ..., x^n) ∈ \RR^n$.
	A \textdef{local coordinate chart} about a point $x ∈ \manif M$ is a coordinate chart of a neighbourhood of $x$.
\end{definition}
We will often call a point $x ∈ \manif M$ by the same symbol as the local coordinates $x^i : \manif M → \RR$ without the index --- but these objects are not interchangeable.



\chapter{Fibre Bundles}
\label{cha:fibre-bundles}

\todo{Treating physical things as fields} would suggest that all values are directly comparable, making expressions like $f(x) + f(y) ∈ A$ geometrically meaningful for different points $x,y ∈ ℳ$.
However, an important lesson from physical theories like general relativity is that it is very often beneficial to distinguish between codomains \emph{at each point in the domain}.

\begin{marginfigure}
	\centering
	\includefigure[0.8\columnwidth]{sphere}
	\caption{
		Vectors in different tangent spaces, and their basis-dependent representation as an $\RR^2$-valued field.
	}
	\label{fig:ball}
\end{marginfigure}

This can be motivated with the simple example of a fluid flowing on a sphere:
The instantaneous fluid velocity at a point is a vector lying in the sphere's tangent plane at that point.
If the fluid flow is given as a field $f : \Sphere^2 → \RR^2$, then any two velocity vectors exist in the ``same'' space, even when \emph{geometrically} they do not (\cref{fig:ball}).
This is more than a purely philosophical point: the fluid flow's representation as a field $f : \Sphere^2 → \RR^2$ is dependent on the choice of basis, i.e., the way in which the single codomain $\RR^2$ is identified with each tangent plane on the sphere.
We would do better with a more geometrical representation of the vector field which is independent of any choice of basis, viewing the fluid velocities at different points as existing in different spaces.

This leads to the formulation of a tangent \emph{bundle} $\TT \Sphere^2$, where all the tangent planes of $\Sphere^2$ are collected in a disjoint union forming a \emph{bulk manifold}.
The vector field on the sphere now becomes a \emph{section} of $\TT \Sphere^2$, which is a map $f : \Sphere^2 → \TT \Sphere^2$ satisfying some conditions.
The tangent bundle is a special case of a \emph{fibre bundle}, which is a manifold consisting of disjoint copies of a space (called the \emph{fibre}) taken at every point in a base manifold.


\begin{definition}
	\label{def:fibre-bundle}
	A \textdef{fibre bundle} $\fibrebundle[π] F ℱ ℳ$ consists of
	\begin{itemize}
		% \item a \textdef{fibre manifold} $A$;
		\item a \textdef{bulk manifold} $ℱ$;
		\item a \textdef{base manifold} $ℳ$; and
		\item a surjection $π : ℱ → ℳ$, the \textdef{projection}, such that
		\item the inverse image $F_x ≔ π^{-1}(x)$ of a base point $x ∈ ℳ$ is homeomorphic to the \textdef{fibre} $F$.
	\end{itemize}
\end{definition}

\Cref{def:fibre-bundle} takes place in the category of manifolds, so the projection $π : ℱ → ℳ$ is continuous.
In a \textdef{smooth fibre bundle}, the projection $π$ is differentiable and $F$, $ℱ$ and $ℳ$ are all smooth manifolds.

\begin{marginfigure}
	\includefigure[\columnwidth]{fibre-bundle}
	\caption{
		(a) A field $f : ℳ → F$, where values at any point can be compared.
		(b) A fibre bundle $\fibrebundle F ℱ ℳ$ with a section $f ∈ \secs(ℱ)$ whose individual fibres $F$ are labelled by base point in $ℳ$.
	}
\end{marginfigure}






\subsubsection{Trivialisations and coordinates}

The bulk $ℱ$ of a fibre bundle $\fibrebundle F ℱ ℳ$ is itself a manifold (of dimension $\dim ℱ = \dim ℳ + \dim F$) so we may always prescribe local coordinates on $ℱ$.
If we already have coordinates $\set{x^μ}$ on the base $ℳ$ and $\set{x^a}$ on a fibre $F$, then we often want to use the same coordinates $\set{x^μ, x^a}$ to describe the bulk $ℱ$.
This first requires a way of continuously splitting the bulk $ℱ → ℳ × F$ into its base and fibre ``components'', in a way which respects the fibred structure of the bundle.
This splitting is known as a \emph{trivialisation} of the bundle.
\begin{definition}
	A \textdef{trivialisation} of a fibre bundle $\fibrebundle[π] F ℱ ℳ$ is a homeomorphism $φ : ℱ → ℳ × F$ such that
	\begin{math}
		\op{pr}_1 ∘ φ = π
	.\end{math}
\end{definition}
It is not always possible to find a trivialisation of a fibre bundle, and if it is, the bundle is called a \textdef{trivial fibre bundle} and there may be different possible trivialisations.\sidenote{
	A simple non-trivial fibre bundle is the Möbius strip, viewed as a bundle over the circle $\Sphere^1$ with fibre $[0, 1]$.
	The trivial bundle $\Sphere^1 × [0, 1]$ describes a strip without a twist.
}

However, it is always possible trivialise \emph{locally}.
That is, for any base point $x ∈ ℳ$, there exists a neighbourhood $x ∈ U ⊆ ℳ$ for which the subbundle $\fibrebundle[π] F {π^{-1}(U)} U$ admits a trivialisation.
Hence, it is always possible to assign \emph{local} coordinates $\set{x^μ, x^a}$ to the bulk of a fibre bundle, where $x^μ$ are coordinates on the base and $x^a$ are coordinates on the fibres, such that $x^μ$ do not vary along the fibres.








\subsubsection{Sections of fibre bundles}


In the language of fibre bundles, a field $f : ℳ → F$ becomes a \emph{section}, which is a ``vertical'' map $f : ℳ → ℱ$ into the bulk $ℱ$ such that $f(x) ∈ F_x$.
\begin{definition}
	A \textdef{section} $f$ of a fibre bundle $\fibrebundle[π] F ℱ ℳ$ is a right-inverse of $π$.
	The space of sections is denoted
	\begin{align}
		\secs(ℱ) = \set{f : ℳ → ℱ | π∘f = \op{id}}
	.\end{align}
\end{definition}
(Again, sections $f ∈ \secs(ℱ)$ are assumed continuous, and \textdef{smooth sections} are sections of smooth fibre bundles for which $f$ is smooth.)


For example, the instantaneous fluid velocity $𝒖$ on a sphere $\Sphere^2$ is a section $𝒖 ∈ \secs(\TT\Sphere^2)$ of the tangent bundle, with a single vector at $x ∈ \Sphere^2$ is denoted $𝒖|_x ∈ \TT_x\Sphere^2$.