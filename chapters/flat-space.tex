\chapter{Calculus in Flat Geometries}

So far, we have been concerned with special relativity at a single point in spacetime.
We move now toward the description of \emph{fields} --- quantities extending across regions of spacetime.
The first step in this direction is the calculus of \emph{flat spacetime}.
In a flat geometry, we may assume that
\begin{itemize}
	\item points in spacetime are elements of a vector space, with differences of points being physically meaningful; and that
	\item fields are parametric functions of a single vector argument representing a location in spacetime.
\end{itemize}
We reserve the word \textdef{field} to mean a map with a fixed vector space codomain.
For instance, the electromagnetic bivector field in flat space $F : \RR^4 → \EA[2]{\RR^4}$ is a function between fixed vector spaces.
In particular, the value of a vector field $F : V → A$ at different points in spacetime can be added; $F(x) + F(y) ∈ A$.

These assumptions are acceptable in special relativity, but in arbitrary regions of spacetime and in the presence of gravity, curvature prevents spacetime from admitting a meaningful vector space structure, and it becomes unphysical to compare field values at different points.
(Consideration of curvature leads to differential geometry and comprises \cref{part:2}.)


\section{Differentiation}


The directional derivative of a vector field $F : V → A$ in the direction $𝒖 ∈ V$ and evaluated at the point $𝒙 ∈ V$ is
\begin{align}
	% ∂_𝒖 F(x) = \lim_{ε → 0}\frac{F(x + ε𝒖) - F(x)}{ε}
	∂_𝒖 F(𝒙) = \dv{ε} \eval{F(𝒙 + ε𝒖)}_{ε = 0}
	= \lim_{ε → 0}\frac{F(𝒙 + ε𝒖) - F(𝒙)}{ε}
.\end{align}
The directional derivative is linear in $𝒖$, since by a change of variables,
\begin{math}
	∂_{u_a\ve^a}
	= \sum_a \dv{ε} \eval{F(x + εu_a\ve^a)}_{ε = 0}
	= \sum_a u_a \dv{\bar ε} \eval{F(x + \bar ε\ve^a)}_{\bar ε = 0}
	= \sum_a u_a ∂_{\ve^a}
.\end{math}
In the presence of a dual basis $\set{\ve^a} \subset V^*$, we use the notation $∂_a ≔ ∂_{\ve^a}$ for brevity, so long as it is understood that this is not a partial derivative with respect to a scalar coordinate, $\pdv{x^a}$.
Of course, it may be viewed as such by setting $f(\etc{x^\i},n) = f(x^i\ve_i)$ so that
\begin{align}
	∂_{\ve^a}f(x^i\ve_i) = \pdv{x^a}f(\etc{x^\i},n)
,\end{align}
though this is a basis-dependent definition.


Suppose $F : V → A$ is some algebra--valued field.
It is useful to define a kind of ``total'' derivative $\DD F$ which does not depend on a direction of differentiation $𝒖$, but instead encompasses, in a sense, all derivatives in a single object $\DD F : V → A$.
The motivation for this is that it encompasses as special cases the soon-to-be-defined exterior derivative (of exterior algebra) and vector derivative (of geometric algebra).
This derivative will be defined when there is a canonical inclusion $ι : V^* → A$ of dual vectors into the algebra $A$, which is automatic if $A$ is a quotient of $\TA{(V^*)}$.

\begin{definition}
	\label{def:algder}
	Let $F : V → A$ be a field with values in an algebra $A$ with product $⊛$, equipped with an inclusion $ι : V^* → A$.
	The \textdef{algebraic derivative} of $F$ is
	\begin{align}
		\label{eqn:algder}
		\DD F ≔ ι(\ve^a) ⊛ ∂_{\ve^a} F
	\end{align}
	(summation on $a$) where $\set{\ve_a} ⊂ V$ and $\set{\ve^a} ⊂ V^*$ are dual bases.
\end{definition}

To understand this definition, consider the simple case of the free tensor algebra $F : V → \TA{(V^*)}$.
We leave the canonical inclusion $ι : V^* → \TA{(V^*)}$ implicit.
Given a basis $\set{\ve^a} ⊂ V^*$, the algebraic derivative is
\begin{math}
	\DD F = \ve^a ⊗ ∂_a F
,\end{math}
which simply encodes the partial derivatives of a $k$-vector $F$ in a $(k + 1)$-grade object.
In component language,
\begin{math}
	(\DD F)_{a\etc{a_\i}{}k} = ∂_aF_{\etc{a_\i}{}k}
.\end{math}

\subsection{The exterior derivative}

Consider a vector field $F : V → \EA{V^*}$ with values in the (dual) exterior algebra.
In this case \cref{eqn:algder} is the \textdef{exterior derivative}
\begin{align}
	\dd F = \dx^a ∧ \pdv{F}{x^a}
\end{align}
where $\set{\dx^a} ⊂ V^*$ form a basis of $\EA{V^*}$.
If $F : V → \EA[k]{V^*}$ is a $k$-vector field, then
\begin{math}
	\dd F = ∂_a F_{\etc{a_\i}{}k} \dx^a ∧ \etc{\dx^{a_\i}}∧k
\end{math}
is a $(k + 1)$-vector.

Using the equivalence of $\EA{V^*}$ with the subspace of antisymmetric tensors (see \cref{sec:exterior-algebra-as-antisymmetric}), the exterior derivative is seen to be the totally anti-symmetrised partial derivative.
In components,
\begin{math}
	(\dd F)_{\etc{a_\i}{}k} = ∂_{[a_1}F_{\etc[2]{a_\i}{}k]}
.\end{math}

The treatment of exterior forms is identical.
An exterior form field $φ : V → \forms[k](V, U)$ is called a $U$-valued \textdef{exterior differential $k$-form}, with exterior derivative defined via its action on vectors
\begin{align}
	(\dd φ)(𝒖, \etc{𝒖_\i},{k})
	&= (\dx^a ∧ ∂_a φ)(𝒖, \etc{𝒖_\i},k)
% \\	&= \sum_{σ ∈ S_{k + 1}} (-1)^σ ∂_{𝒖_{σ(0)}} φ(\etc{𝒖_{σ(\i)}}{}k)
\\	&= \sum_{i = 0}^k (-1)^i ∂_{𝒖_{i}} φ(𝒖_0, ..., \widehat{𝒖_i}, ..., 𝒖_k)
\end{align}
in the Spivak convention (see \cref{sec:exterior-forms}).
Note that the partial derivative acts on the position dependence of $φ$ only --- the vectors $𝒖_i ∈ V$ are fixed input vectors.
This changes when generalising from vector fields of alternating maps to forms defined on a \emph{manifold}, where correction terms are needed to account for partial derivatives of input vectors (discussed in \cref{part:2}).


\subsection{The vector derivative}

The algebraic derivative in the tensor and exterior algebras are somewhat uninteresting, because they are easily expressible in component form (e.g., $∂_aF_{\etc{a_\i}{}k}$ or $∂_{[a}F_{\etc{a_\i}{}k]}$).
This is not possible in the geometric algebra, however, because $\GA(V, η)$ is not $\ZZ$-graded, and we would face the problem of notating inhomogeneous objects with a variable number of indices.
The algebraic derivative is, however, still geometrically significant and useful in this case.

In $\GA(V, η)$, the algebraic derivative is called the \textdef{vector derivative}, which we denote $\vd$.
Explicitly, if $F : V → \GA(V, η)$ is a multivector field, then in \cref{eqn:algder} we take $⊛$ to be with the geometric product and take the inclusion to be\sidenote{
	We could just as well consider fields $V → \GA(V^*, η)$, avoiding the need for the isomorphism $\sharp : V^* → V$.
	But the metric is already defined, so we prefer multivectors $\GA(V, η)$ to `dual-multivectors'.
}
\begin{math}
	V^* ∋ 𝒖 ↦ ι(𝒖^\sharp) ∈ \GA(V, η)
.\end{math}
Here, we use the metric to relate $V^* → V$ and the canonical inclusion $ι : V ≡ \GA[1](V, η) → \GA(V, η)$.
The vector derivative is then
\begin{align}
	\vd F = \ve^a \, ∂_{\ve^a} F
\end{align}
(summation on $a$) where $\set{\ve_a} ⊂ V$ and $\set{\ve^a} ⊂ V$ are dual bases, and juxtaposition denotes the geometric product.
If $F$ is a homogeneous $k$-vector, then we may write its components as
\begin{math}
	F = F_{\etc{a_\i}{}k} \etc{\ve^{a_\i}}∧k
\end{math}
and hence
\begin{align}
	\vd F = ∂_{\ve^a}F_{\etc{a_\i}{}k} \, \ve^a(\etc{\ve^{a_\i}}∧k)
.\end{align}
Note that these terms are not $(k + 1)$-blades, but geometric products of vectors $\ve^a$ with $k$-blades --- in general, $(k ± 1)$-multivectors.

We may regard the vector derivative itself as an operator-valued vector,
\begin{align}
	\vd = \ve^a ∂_a
,\end{align}
reflecting the fact that $\vd$ behaves algebraically like a vector.
For instance, the derivative of a vector $𝒖$ has scalar and bivector parts,
\begin{math}
	\vd 𝒖 = \vd \fatdot 𝒖 + \vd ∧ 𝒖
,\end{math}
just like the geometric product of two vectors, $𝒖𝒗 = 𝒖 \fatdot 𝒗 + 𝒖 ∧ 𝒗$.
For a general multivector $F$, then, we have
\begin{align}
	\vd F = \vd \lcontr F + \vd ∧ F
.\end{align}


\section{Case Study: Maxwell's Equations}


Expressed in the standard vector calculus of $\RR^3$, Maxwell's equations for the electric $\vb E$ and magnetic $\vb B$ fields in the presence of a source are
\begin{align}
	∇ \cdot \vb E &= \frac{ρ}{ε_0} &&\text{(Gauß' law)}
\\	∇ \cdot \vb B &= 0 &&\text{(Absence of magnetic monopoles)}
\\[1ex]	∇ × \vb E &= -∂_t \vb B &&\text{(Faraday's law)}
\\[1ex]	∇ × \vb B &= μ_0(\vb J + ε_0∂_t \vb E) &&\text{(Ampère's law)}
\end{align}
where $ρ$ is the scalar charge density and $\vb J$ the current density.
The constants $ε_0$ and $μ_0$ are the vacuum permittivity and permeability, respectively, related to the speed of light $c$ by $ε_0μ_0c^2 = 1$.



\begin{margintable}
	\footnotesize
	\begin{tabular}{cl}
		\\
		\multicolumn{2}{c}{\emph{Non-relativistic}} \\
		quantity & dimension \\
		$\vb E$ & $MQ^{-1}LT^{-2}$ \\
		$\vb B$ & $MQ^{-1}T^{-1}$ \\
		$ρ$ & $QL^{-3}$ \\
		$\vb J$ & $QT^{-1}L^{-2}$ \\
		$μ_0$ & $MQ^{-2}L$ \\
		$ε_0$ & $M^{-1}Q^2L^{-3}T^2$ \\
		$∇$, $∂_t$ & $L^{-1}$, $T^{-1}$ \\
		$c$ & $LT^{-1}$ \\
		\\
		\multicolumn{2}{c}{\emph{Relativistic}} \\
		quantity & dimension \\
		$F$ & $MQ^{-1}S^{-1}$ \\
		$J$ & $QS^{-3}$ \\
		$μ_0$, $ε_0^{-1}$ & $MQ^{-2}S$ \\
		$∂$ & $S^{-1}$ \\
		$c$ & $1$ \\
	\end{tabular}
	\caption{
		Dimensions of physical quantities in Maxwell's equations.
		$M$ is mass, $Q$ is electric charge, $T$ is duration and $L$ is length.
		In the relativistic formulation, $T$ and $L$ are unified and replaced by \emph{spacetime interval} $S$.
	}
\end{margintable}

\subsection{With tensor calculus}

These can be expressed relativistically as eight scalar equations,
\setlength{\fboxsep}{1.4ex}
\begin{empheq}[box=\fbox]{align}
	\label{eqn:maxwells-tensor-form}
	∂_μF^{μν} &= μ_0J^ν
,&	∂_μG^{μν} &= 0
\end{empheq}
where $F^{μν} = -F^{νμ}$ is the Faraday tensor and $G^{μν}$ its Hodge dual, both encoding the electric and magnetic fields via
\begin{align}
	\label{eqn:components-of-electromagnetic-tensor}
	F^{i0} &= \frac{E^i}{c}
,&	F^{ij} &= -ε^{ijk}B_k
,&	G^{μν} &= \frac12 ε^{μν}{}_{ρσ} F^{ρσ}
% ,&	G^{μν} &= \frac12 η^{μρ}η^{νσ}ε_{ρσλυ} F^{λυ}
,\end{align}
and where $J^μ$ encodes both the static charge density $J^0 = cρ$ and current density $J^i = \vb J$.
The left of eqs.~\eqref{eqn:maxwells-tensor-form} is the \emph{source equation}, while the right is the \emph{second Bianchi identity}.
These equations assume the metric signature $\sig{+---}$, where the equivalent equations under $\sig{-+++}$ are obtained by a change of sign $F^{μν} \mapsto -F^{μν}$.



\begin{proof}
	We show how the relativistic equations \eqref{eqn:maxwells-tensor-form} reduce to the non-relativistic vector calculus equivalents.
	The $0$-component of the source equation is
	$∂_μ F^{μ0} = ∂_iE^i/c = μ_0J^0 = μ_0cρ$ implying $∇ · \vb E = ρ/ε_0$ (Gauß' law).
	The $i$-components are
	\begin{align}
		∂_0F^{0i} + ∂_jF^{ji} &= \frac1c ∂_t \qty(-\frac{E^i}{c}) - ∂_j ε^{jik} B_k = μ_0 J^i
\\		\qqtext{or} ∂_jε^{ijk}B_k &= μ_0 J^i + μ_0ε_0∂_tE^i
	,\end{align}
	which is equivalent to Ampère's law.
	The $0$-component of the Bianchi identity $∂_μG^{μ0} = 0$ is
	\begin{align}
		% \frac12 ε^{i0}{}_{μν}∂_iF^{μν}
		\frac12 ε^i{}_{jk}∂_iF^{jk}
		= -\frac12 ε^i{}_{jk}ε^{jkl}∂_iB_l
		= -∂_iB^i = 0
	,\end{align}
	which using the identity $ε_{ijk}ε^{jkl} = 2δ^l_i$ is $∇ · \vb B = 0$.
	Finally, the $i$-component gives
	\begin{align}
		0 = ∂_μG^{μi} &= \frac12ε^{μi}{}_{ρσ}∂_μF^{ρσ}
		= \frac12ε^{0i}{}_{jk}∂_0F^{jk} + ε^{ji}{}_{k0}∂_jF^{k0}
	\\	&= -\frac14ε^i{}_{jk}ε^{jkl}∂_0B_l - \frac1{2c} ε^{ijk}∂_jE_k
		= -\frac1{2c}\qty(∂_tB^i  + ε^{ijk}∂_jE_k)
	\end{align}
	yielding Faraday's law $∇ × \vb E = -∂_t \vb B$.
\end{proof}



\subsection{With exterior calculus}

It is easy to translate from the language of exterior calculus to tensor calculus, and hence vice versa, by identifying the former as the subalgebra of totally antisymmetric tensors (as in \cref{sec:exterior-algebra-as-antisymmetric}).
We will employ the Spivak convention, which in particular identifies $2$-forms via
\begin{align}
	\ve^μ ∧ \ve^ν ≡ \ve^μ ⊗ \ve^ν - \ve^ν ⊗ \ve^μ
\end{align}
where $\ve^μ$ are spacetime basis vectors (having physical dimensions of spacetime interval, $S$).
We then the electromagnetic bivector as
\begin{math}
	\mathcal{F} = \frac12 F_{μν} \ve^μ ∧ \ve^ν
\end{math}
(omitting the $\frac12$ in the Kobayashi–Nomizu convention).


Since the charge density $J \sim QS^{-3}$ has dimensions of charge per spacetime $3$-volume, it is natural to interpret it as a \emph{trivector}
\begin{align}
	\mathcal{J} = J^{μνλ} \, \ve_μ ∧ \ve_ν ∧ \ve_λ ≔ J^μ \star \ve_μ = \frac1{3!}ε_{μνλα}J^α \ve^μ ∧ \ve^ν ∧ \ve^λ
\end{align}
so that the coefficients $J^{μνλ} \sim Q$ have dimensions of charge.\sidenote{
	Note that dual vectors $\ve_μ$ have dimension $S^{-1}$.
}

The relativistic Maxwell equations are then
\begin{empheq}[box=\fbox]{align}
	\dd \star \mathcal{F} &= μ_0 \mathcal{J}
,&	\dd \mathcal{F} &= 0
.\end{empheq}
\begin{proof}
	The first equation written in component form is
	\begin{align}
		\frac14 ε_{μνρσ}∂_λF^{ρσ} &= \frac1{3!}ε_{λμνα}μ_0J^α
	,\end{align}
	which, by contracting with $ε^{μνλβ}$ and using the identities
	\begin{math}
		ε^{μνλβ}ε_{μνρσ} = 2(δ^λ_ρδ^β_σ - δ^λ_σδ^β_ρ)
		\text{ and }
		ε^{μνλβ}ε_{λμνα} = 3!δ^β_σ
	,\end{math}
	reduces to
	\begin{align}
		\frac12(∂_λF^{λβ} - ∂_λF^{βλ}) &= μ_0J^β
	\end{align}
	or $∂_μF^{μν} = μ_0J^ν$, the source equation.
	The Bianchi identity can be rewritten as
	\begin{align}
		∂_μG^{μν}
		= \frac12ε^{μν}{}_{ρσ}∂_μF^{ρσ}
		= -\frac12ε^{ν[μρσ]}∂_μF_{ρσ}
		= -\frac12ε^{νμρσ}∂_{[μ}F_{ρσ]}
		= 0
	,\end{align}
	implying $\dd \mathcal{F} = 0$.
\end{proof}


\subsection{With geometric calculus}

Using the spacetime algebra $\GA(1,3)$ with vector basis $\set{\vg_μ}$ as introduced in \cref{cha:sta}, the electromagnetic bivector is\sidenote{
	This coincides with the electromagnetic bivector $2$-form $\mathcal{F}$ in the Kobayashi–Nomizu convention, because the wedge product in geometric algebra is naturally normalised (see \cref{tbl:wedge-conventions}).
}
\begin{align}
	\label{eqn:ga-em-bivector}
	F = F^{μν} \vg_μ\vg_ν
\end{align}
and the current density is
\begin{align}
	\vb J = J^μ \vg_μ
.\end{align}
Maxwell's equations are equivalent to the \emph{single} multivector equation
\begin{empheq}[box=\fbox]{align}
	\label{eqn:maxwell-sta}
	\vd F = μ_0\vb J
.\end{empheq}
\begin{proof}
	The multivector equation $\vd F = μ_0\vb J$ separates into a vector part 
	\begin{math}
		{\vd \cdot F} = μ_0\vb J
	\end{math}
	and a trivector part
	\begin{math}
		\vd ∧ F = 0
	.\end{math}
	In terms of components, the vector part is
	\begin{align}
		\vd \cdot F = ∂_λF^{μν} \vg^λ \cdot (\vg_μ\vg_ν) = μ_0J^ν\vg_ν
	.\end{align}
	The only non-zero components are those for which $μ \ne ν$.
	If $λ$, $μ$ and $ν$ are all distinct, then $\vg^λ \cdot (\vg_μ\vg_ν) = \grade[1]{\vg^λ\vg_μ\vg_ν} = 0$.
	There are then two cases, $λ = μ$ and $λ = ν$, which respectively simplify
	\begin{align}
		\vg^μ \cdot (\vg_μ\vg_ν) &= \grade[1]{\vg^μ\vg_μ\vg_ν} = \vg_ν
	,\\	\vg^ν \cdot (\vg_μ\vg_ν) &= \grade[1]{\vg^ν\vg_μ\vg_ν} = -\vg_μ
	,\end{align}
	so that
	\begin{align}
		\vd \cdot F = \qty(∂_μF^{μν}\vg_ν - ∂_νF^{μν}\vg_μ) = ∂_μF^{μν}\vg_ν
	.\end{align}
	This recovers the source equation $2∂_μF^{μν} = μ_0J^ν$.

	It is clear that the trivector part
	\begin{align}
		\vd ∧ F = ∂_λF^{μν} \vg^λ ∧ (\vg_μ\vg_ν) = ∂_λF_{μν} \vg^λ ∧ \vg^μ ∧ \vg^ν = 0
	\end{align}
	is equivalent to the exterior algebraic Bianchi identity $\dd \mathcal{F} = 0$.
\end{proof}


\subsubsection{In terms of electric and magnetic fields}



It is worth showing how the relativistic Maxwell equation \eqref{eqn:maxwell-sta} splits into a frame-dependent description in the geometric algebra framework.
As in \cref{sec:spacetime-split}, we use the notation $\vec u$ to indicate relative vectors; i.e., timelike bivectors
of the spacetime algebra $\GA(1,3)$ which are simultaneously grade-$1$ vectors in the observer's algebra $\GA(3)$.

From \cref{eqn:ga-em-bivector,eqn:components-of-electromagnetic-tensor}, the electromagnetic bivector is expressed in the $\vg_0$-frame as\sidenote{
	We assume $\sig{+---}$ for concreteness; for $\sig{-+++}$ replace $F \mapsto -F$.
}
\begin{align}
	\label{eqn:F-is-E-iB}
	F = \frac1c \vec E + \vol \vec B
,\end{align}
where
\begin{math}
	\vec E = E^i\vs_i = E^i \vg_i\vg_0
\end{math}
and
\begin{align}
	\vol\vec B = B_i \vol \vs^i = \frac12 B_i ε^{ijk}\vs_j\vs_k
	= \frac12 B_i ε^{ijk}\vg_j\vg_k
.\end{align}
\Cref{eqn:F-is-E-iB} should be compared with the Riemann-Silberstein vector \cite{silberstein} which has the form $\vec F_\CC = \vec E + ic\vec B$.

Similarly, the current density spacetime vector $J$ may be viewed under the space/time split by (left) multiplying by the frame velocity $\vg_0$,
\begin{align}
	\vg_0 \vb J = cρ - \vec J
,\end{align}
where $J^0 = cρ$ and $\vec J = J^i \vs_i$.
Similarly for the vector derivative, we have
\begin{align}
	\vg_0 \, \vd = \frac1c \pdv{t} + \vec ∇
\end{align}
in either signature.

Putting these together, we split \cref{eqn:maxwell-sta} within the $\vg_0$-frame by left-multiplying by $\vg_0$;
\begin{align}
	\vg_0 \, \vd F &= \vg_0 μ_0 \vb J
\\=	\qty(\frac1c \pdv{t} + \vec ∇)\qty(\frac1c \vec E + \vol \vec B) &= μ_0\qty(cρ - \vec J)
.\end{align}
By expanding and equating grades, we obtain four equations,
\begin{align}
	\frac1c \vec ∇ · \vec E &= μ_0 cρ
&	&\text{(scalar)}
\\	\frac1{c^2} \pdv{\vec E}{t} + \vol (\vec ∇ ∧ \vec B) &= -μ_0 \vec J
&	&\text{(vector)}
\\	\frac1c \vec ∇ ∧ \vec E + \frac\vol c \pdv{\vec B}{t} &= 0
&	&\text{(bivector)}
\\	\vol (\vec ∇ · \vec B) &= 0
&	&\text{(pseudoscalar)}
\end{align}
Note that the cross product relates to the bivector curl in $\GA(3)$ by
\begin{align}
	𝒖 ∧ 𝒗 = \vol(𝒖 × 𝒗)
	\qqtext{so that}
	∇ × 𝑿 = -\vol(\vec ∇ ∧ \vec X)
.\end{align}
Hence, by adjusting by factors of $c$ and $\vol$ (and using $μ_0ε_0c^2 = 1$), the above equations reduce immediately to Gauß's law, Ampère's law, Faraday's law and the magnetic monopole equation, respectively.

This was done assuming $η = \op{diag}\sig{+---}$.
In the $\sig{-+++}$ signature, $\vg_0 \vb J = -cρ + \vec J$ differs by an overall sign, complementing $F \mapsto -F$.


\section{Integration}


\subsection{Stokes' theorem for exterior calculus}

\begin{theorem}[Stokes' theorem in $\RR^n$]
	\label{thm:flat-stokes}
	If $R ⊆ \RR^n$ is a compact $k$-dimensional hypersurface with boundary $∂R$, then a smooth differential form $ω ∈ \forms[k - 1](R)$ satisfies
	\begin{align}
		\label{eqn:stokes}
		\int_R \dd ω = \int_{∂R} ω
	.\end{align}
\end{theorem}
\begin{proof}
	Since $R$ is a $k$-dimensional region with boundary, every point $x ∈ R$ has a neighbourhood diffeomorphic to a neighbourhood of the origin in either $\RR^k$ or $H^k ≔ [0, ∞) ⊕ \RR^{k - 1}$, depending on whether $x$ is an interior point or a boundary point, respectively.

	\begin{marginfigure}
		\centering
		\includefigure[\columnwidth]{stokes-theorem}
		\caption{
			Neighbourhoods in $R$ are diffeomorphic either to interior balls or boundary half-balls.
		}
		\label{fig:stokes-theorem}
	\end{marginfigure}

	Let $\set{U_i}$ be a cover of $R$ consisting of such neighbourhoods.
	Since $R$ is compact, we may assume $\bigcup_{i=1}^N \set{U_i} = R$ to be a finite covering.
	Thus, we have finitely maps $h_i : U_i → X$ where $X$ is either $\RR^k$ or the half-space $H^k$, where $U_i \cong h_i(U_i)$ are diffeomorphic (see \cref{fig:stokes-theorem}).

	Finally, let $\set{ϕ_i : R → [0, 1]}$ be a partition of unity subordinate to $\set{U_i}$, so that $\set{x ∈ R | ϕ_i(x) > 0} ⊆ U_i$ and $ω = \sum_{i=1}^N ϕ_iω$.
	We need only prove the equality \eqref{eqn:stokes} for each $ω_i ≔ ϕ_iω$, and the full result follows be linearity.
	
	The form $h_i^*ω_i ∈ \forms[k - 1](X)$ can be written with respect to canonical coordinates of $X$ as
	\begin{align}
		h_i^*ω_i = \sum_{j=1}^k f_j (-1)^{j - 1} \dd x^{1\cdots\hat{j}\cdots k}
	\end{align}
	using the multi-index notation $\dx^{\etc{i_\i}{}k} ≡ \etc{\dx^{i_\i}}∧k$, where the hat denotes an omitted term.
	The factor of $(-1)^{j - 1}$ gives the $(k - 1)$-form the boundary orientation induced by the volume form $\dx^{\etc\i{}k}$ for convenience.
	Since pullbacks commute with $\dd$,
	\begin{align}
		h^*\ddω_i = \dd(h_i^*ω_i) = \sum_{j=1}^k \pdv{f_j}{x^j} \dd x^{1\cdots n}
	.\end{align}
	There are then two cases to consider.
	\begin{itemize}
		\item \emph{Interior case.}
		If $h_i : U_i → \RR^k$, then the right-hand side of \cref{eqn:stokes} vanishes because $ω_i$ is zero outside the neighbourhood $U_i ⊂ R$ which nowhere meets the boundary $∂R$.
		\begin{align}
			\int_{∂R} ω_i = \int_{∂U_i} ω_i = \int_∅ ω_i = 0
		\end{align}
		The left-hand side evaluates to
		\begin{align}
			\int_R \dd ω_i
			&= \int_X \dd (h_i^*ω_i)
			= \int_{\RR^k} \sum_{j=1}^k \pdv{f_j}{x^j} \dd x^{1\cdots n}
		\\	&= \underbrace{\etc{\int_{-∞}^{+∞}}{}{}}_k \sum_{j=1}^k \pdv{f_j}{x^j} \etc{dx^\i}{}{k}
		\\	&= \underbrace{\etc{\int_{-∞}^{+∞}}{}{}}_{k - 1} \sum_{j=1}^k \eval{f_j}_{x^j=-∞}^{+∞} (-1)^{j-1} dx^1\cdots\widehat{dx^j}\cdots dx^k
			= 0
		,\end{align}
		which vanishes because $h_i^*ω_i$, and hence the $f_j$, vanish outside the neighbourhood $h_i(U_i) ⊂ \RR^k$.

	\item \emph{Boundary case.}
	If $h_i : U_i → H^k$, then the boundary $∂U_i ⊂ ∂R$ is mapped onto the hyperplane $∂H^k = \set{(0, \etc[2]{x^\i},k) | x^j ∈ \RR}$.
	Thus, $dx^1 = 0$ on this boundary, and the right-hand side of \cref{eqn:stokes} becomes
	\begin{align}
		\int_{∂R}ω_i
		&= \int_{∂U_i} h_i^*ω_i
		= -\int_{\RR^{k - 1}} f_1 \etc[2]{dx^\i}{}k
	\\	&= -\underbrace{\etc{\int_{-∞}^{+∞}}{}{}}_{k - 1} f_1(0, \etc[2]{x^\i},k) \etc[2]{dx^\i}{}k
	.\end{align}
	The factor of $-1$ comes from the induced orientation of the boundary $∂H^k$, which is outward-facing, so in the \emph{negative} $x^1$ direction.
	For the left-hand side of \cref{eqn:stokes},
	\begin{align}
		\int_R \dd ω_i
		&= \int_{H^k} h_i^*\ddω_i
		= \int_0^∞ \etc{\int_{-∞}^{+∞}}{}{} \sum_{j=1}^k \pdv{f_j}{x^j} \etc{dx^\i}{}k
	\intertext{
		All terms $\pdv{f_j}{x^j}dx^j$ in the sum for $j > 1$ integrate to boundary terms $x_j → ±∞$ where $f_j$ vanishes.
		This leaves the single term from the integration of $dx^1$,
	}
		&= -\etc{\int_{-∞}^{+∞}}{}{} \eval{f_1}_{x^1=0}^∞ \etc[2]{dx^\i}k
	.\end{align}

	\end{itemize}

	Thus, we have equality for all $ω_i$, so
	\begin{align}
		\int_R \dd ω = \sum_{i=1}^N \int_R \dd ω_i = \sum_{i=1}^N \int_{∂R} ω_i = \int_{∂R} ω
	\end{align}
	by linearity.
\end{proof}

\subsection{Fundamental theorem of geometric calculus}

\begin{theorem}
	Let $f(𝒙)$ be a multivector field.
	The vector derivative is
	\begin{align}
		\vd f(𝒙) = \lim_{|ℛ| → 0} \frac1{|ℛ|\vol} \oint_{∂ℛ} \bd S f
	,\end{align}
	where $ℛ$ is a volume containing $𝒙$ with boundary $∂ℛ$ and volume $|ℛ| = \int_ℛ \bd V$.
	The limit is taken as the volume $ℛ$ shrinks to the point $𝒙$.
\end{theorem}
Note that the integrand $\bd S f$ is the geometric product between the hypersurface element and the field.
\begin{proof}
	It will suffice to prove the case where $ℛ$ is a box shape; arbitrary regions can be approximated via tessellation in the limit of vanishing voxel size.

	Let $B_ε = \set{x^i\ve_i | x^i ∈ [-ε, +ε]}$ denote the $n$-dimensional cube of diameter $2ε$ centred at the origin.
	If the surface $∂B_ε$ is oriented outward, then the face in the $+\ve^k$ direction is orientated like the $(n-1)$-blade $\vol \ve^k = (-1)^{n-k}\etcskip{\ve_\i}∧kn$.
	Upon this face the infinitesimal surface element is
	\begin{align}
		% \bd^{(k)}x = \vol \ve^k dx^1\cdots\widehat{dx^k}\cdots dx^n
		\bd^{(k)}x = \vol \ve^k \etcskip{dx^\i}{}{k}{n}
	,\end{align}
	while the opposing face has surface element $-\bd^{(k)}x$.

	Consider the integral of $f$ over the surface $∂B_ε$, split into a sum of $n$ surface integrals over each pair of opposing faces.
	The $k$th pair are the surfaces $\set{x^i\ve_i ± ε\ve_k | x^i ∈ [-ε, +ε], i ≠ k}$ where $i$ sums over axes \emph{other} than $k$.
	Hence, we have
	\begin{align}
		\oint_{∂B_ε} \bd S f = \sum_{k=1}^n \int_{[-ε, +ε]^{n-1}} \bd^{(k)}x \, \qty(f(x^i\ve_i + ε\ve_k) - f(x^i\ve_i - ε\ve_k))
	,	\quad(i ≠ k)
	.\end{align}
	By series expanding $f$ in each $x^i$, and then in $ε$, obtain
	\begin{align}
		f(x^i\ve_i ± ε\ve_k) &= f(±ε\ve_k) + x^i ∂_{\ve^i} (f(0) ± ε∂_{\ve^k} f(0))
	.\end{align}
	Since $|x^i| ≤ ε$, the last term is $𝒪(ε^2)$, and difference in the integrand is hence
	\begin{align}
		f(x^i\ve_i + ε\ve_k) - f(x^i\ve_i - ε\ve_k)
		&= f(ε\ve_k) - f(-ε\ve_k) + 𝒪(ε^2)
	\\	&= 2ε∂_{\ve^k}f(0) + 𝒪(ε^2)
	.\end{align}
	Therefore, after pulling constants outside the integrals, we have
	\begin{align}
		\oint_{∂B_ε} \bd S f
		&≈ \sum_{k=1}^n 2ε \, ∂_{\ve^k} f(0) \int_{[-ε, +ε]^{n-1}} \bd^{(k)}x
	\end{align}
	to order $𝒪(ε^2)$.
	The integrands each evaluate to the area $(2ε)^{n-1}$, giving
	\begin{align}
		\oint_{∂B_ε} \bd S f
		≈ (2ε)^n \vol\ve^k ∂_{\ve^k} f(0)
		= |B_ε|\vol \, \vd f(0)
	,\end{align}
	to order $𝒪(ε^2)$, which becomes exact in the limit,
	\begin{align}
		\label{eqn:vd-int-form}
		\vd f(0) = \lim_{ε → 0} \frac1{|B_ε|\vol} \oint_{∂B_ε} \bd S f
	.\end{align}

	By translation, $f(𝒙) ↦ f'(𝒙) = f(𝒙 - 𝒖)$, we obtain the integral form of $\vd f(𝒖)$ evaluated at an arbitrary position $𝒖$.
\end{proof}

\begin{theorem}
	For an $n$-dimensional region $ℛ$ with boundary $∂ℛ$, and a multivector field $f(𝒙)$,
	\begin{align}
		\int_ℛ \bd V \, \vd f = \oint_{∂ℛ} \bd S f
	,\end{align}
	where $\bd V$ denotes an $n$-blade volume element, and $\bd S$ an $(n - 1)$-blade surface element, and where juxtoposition is the geometric product. 
\end{theorem}
\begin{proof}
	An arbitray volume $ℛ$ with boundary $∂ℛ$ can be approximated as tessellated boxes of arbitrily small size.\sidenote{
		It is not neccesary that the surface area of the approximation converge to $|∂ℛ|$.
	}
	Suppose $ℛ$ is approximated by a regular lattice of $N$ boxes of radius $ε$.
	Consider the sum of $\vd f$ over the lattice points, weighted by volume.
	From \cref{eqn:vd-int-form} this can be written in terms a sum of surface integrals,
	\begin{align}
		\sum_{i=1}^N |B_i|\vol \, \vd f(𝒙_i) = \sum_{i=1}^N \oint_{∂B_i} \bd S f
	.\end{align}
	Note that interior faces of the boxes come in oppositely-oriented pairs, so that surface integrals over interior faces cancel.
	Therefore, the result is obtained in the continuous limit $N → ∞$.
\end{proof}

\todo{Comment on how this generalizes Stokes' theorem.}