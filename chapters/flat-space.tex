\chapter{Calculus in Flat Geometries}

So far, we have been concerned with special relativity at a single point in spacetime.
The next step is a description of \emph{fields}, which are quantities extending across regions of spacetime.
The simplest step in this direction is the calculus of \emph{flat spacetime}.
In a flat geometry, we assume that
\begin{itemize}
	\item points in spacetime are elements of a vector space, with differences of points being physically meaningful; and that
	\item fields are parametric functions of a single vector argument representing a location in spacetime.
\end{itemize}
For instance, the electromagnetic bivector in flat space $F : \RR^4 → \EA[2]{\RR^4}$ is a function between fixed vector spaces.
These assumptions are acceptable in special relativity, but in arbitrary regions of spacetime and in the presence of gravity, curvature prevents spacetime from admitting a meaningful vector space structure.
(Consideration of curvature leads to differential geometry, and comprises \cref{part:2}.)


\todo{
	\begin{enumerate}
		\item Exterior derivative and vector derivative
		\item Stokes' theorem and the GA equivalent
		\item Maxwell's equations
	\end{enumerate}
}

\section{Differentiation}

We reserve the word \textdef{field} to mean a map with a fixed vector space codomain.
% \begin{definition}
% 	A \textdef{vector field} is a mapping into a fixed vector space.
% \end{definition}
In particular, the value of a vector field $F : V → A$ at different points in spacetime can be added; $F(x) + F(y) ∈ A$.

The directional derivative of a vector field $F : V → A$ in the direction $𝒖 ∈ V$ is
\begin{align}
	% ∂_𝒖 F(x) = \lim_{ε → 0}\frac{F(x + ε𝒖) - F(x)}{ε}
	∂_𝒖 F(x) = \dv{ε} \eval{F(x + ε𝒖)}_{ε = 0}
	= \lim_{ε → 0}\frac{F(x + ε𝒖) - F(x)}{ε}
\end{align}
where the point $x ∈ V$ is also a vector.
The directional derivative is linear in $𝒖$, since by a change of variables,
\begin{math}
	∂_{u^a\ve_a}
	= \sum_a \dv{ε} \eval{F(x + εu^a\ve_a)}_{ε = 0}
	= \sum_a u^a \dv{ε'} \eval{F(x + ε'\ve_a)}_{ε' = 0}
	= u^a ∂_{\ve_a}
.\end{math}


Suppose $F : V → A$ is some algebra--valued field.
It is useful to define a kind of ``total'' derivative $\DD F$ which does not depend on a direction of differentiation $𝒖$, but instead encompasses, in a sense, all derivatives in a single object $\DD F : V → A$.
The motivation for this construction is that it encompasses the soon-to-be-defined exterior derivative (of exterior algebra) and vector derivative (of geometric algebra) as special cases.
This derivative will be defined when there is a canonical inclusion $ι : V^* → A$ of dual vectors into the algebra $A$, which is automatic if $A$ is a quotient of $\TA{(V^*)}$.

\begin{definition}
	\label{def:algder}
	Let $F : V → A$ be a field with values in an algebra $A$ with product $⊛$, equipped with an inclusion $ι : V^* → A$.
	The \textdef{algebraic derivative} of $F$ is
	\begin{align}
		\label{eqn:algder}
		\DD F ≔ ι(\ve^a) ⊛ ∂_{\ve_a} F
	\end{align}
	(summation on $a$) where $\set{\ve_a} ⊂ V$ and $\set{\ve^a} ⊂ V^*$ are dual bases.
\end{definition}

To understand this definition, consider the simple case of the free tensor algebra $F : V → \TA{(V^*)}$.
We leave the inclusion $ι : V^* → \TA{(V^*)}$ implicit.
Given a basis $\set{\ve^a} ⊂ V^*$, the algebraic derivative is
\begin{math}
	\DD F = \ve^a ⊗ ∂_a F
\end{math}
simply encodes the partial derivatives of a $k$-vector $F$ in a $(k + 1)$-grade object.
In component language,
\begin{math}
	(\DD F)_{a\etc{a_\i}{}k} = ∂_aF_{\etc{a_\i}{}k}
.\end{math}

\subsection{The Exterior Derivative}

Consider a vector field $F : V → \EA{V^*}$ with values in the (dual) exterior algebra.
In this case the algebraic derivative \eqref{eqn:algder} reduces to the \textdef{exterior derivative}
\begin{align}
	\dd F = \dx^a ∧ \pdv{F}{x^a}
\end{align}
where $\set{\dx^a} ⊂ V^*$ also form a dual basis of $\EA{V^*}$.
If $F : V → \EA[k]{V^*}$ is a $k$-vector field, then
\begin{math}
	\dd F = ∂_a F_{\etc{a_\i}{}k} \dx^a ∧ \etc{\dx^{a_\i}}∧k
\end{math}
is a $(k + 1)$-vector.

Using the equivalence of $\EA{V^*}$ with the subspace of antisymmetric tensors (see \cref{sec:exterior-algebra-as-antisymmetric}), the exterior derivative is seen to be the totally anti-symmetrised partial derivative.
In components,
\begin{math}
	(\dd F)_{\etc{a_\i}{}k} = ∂_{[a_1}F_{\etc[2]{a_\i}{}k]}
.\end{math}

The treatment of exterior forms is identical.
An exterior form field $φ : V → \forms[k](V, U)$ is called a $U$-valued \textdef{exterior differential $k$-form}, with exterior derivative defined via its action on vectors
\begin{align}
	(\dd φ)(𝒖, \etc{𝒖_\i},{k})
	&= (\dx^a ∧ ∂_a φ)(𝒖, \etc{𝒖_\i},k)
% \\	&= \sum_{σ ∈ S_{k + 1}} (-1)^σ ∂_{𝒖_{σ(0)}} φ(\etc{𝒖_{σ(\i)}}{}k)
\\	&= \sum_{i = 0}^k (-1)^i ∂_{𝒖_{i}} φ(𝒖_0, ..., \widehat{𝒖_i}, ..., 𝒖_k)
\end{align}
in the Spivak convention (see \cref{sec:exterior-forms}).
Note that the partial derivative acts on the position dependence of $φ$ only --- the vectors $𝒖_i ∈ V$ are fixed input vectors.
This changes when generalising from vector fields of alternating maps $V → \forms(V, U)$ to objects defined on a \emph{manifold} $\forms(ℳ, U)$, where correction terms are needed to account for partial derivatives of input vector sections (discussed in \cref{part:2}).


\subsection{The Vector Derivative}

The algebraic derivative in the tensor and exterior algebras are somewhat uninteresting, because they are easily expressible in component form (as $∂_aF_{\etc{a_\i}{}k}$ or $∂_{[a}F_{\etc{a_\i}{}k]}$).
This is not possible in the geometric algebra, however, because $\GA(V, η)$ is not $\ZZ$-graded, and we face the problem of notating inhomogeneous objects with a variable number of indices.

In $\GA(V, η)$, the algebraic derivative is called the \textdef{vector derivative}, denoted $\vd$.
Explicitly, if $F : V → \GA(V, η)$ is a multivector field, then we take \cref{eqn:algder} with the geometric product and the inclusion\sidenote{
	We could just as well consider fields $V → \GA(V^*, η)$, avoiding the need for the isomorphism $\sharp : V^* → V$.
	But the metric is already defined, so we prefer multivectors $\GA(V, η)$ to `dual-multivectors'.
}
\begin{align}
	V^* ∋ 𝒖 \mapsto ι(𝒖^\sharp) ∈ \GA(V, η)
.\end{align}
Here, we use the metric to relate $V^* → V$ and the canonical inclusion $ι : V ≡ \GA[1](V, η) → \GA(V, η)$.
The vector derivative is then
\begin{align}
	\vd F = \ve^a \, ∂_{\ve_a} F
\end{align}
(summation on $a$) where $\set{\ve_a} ⊂ V$ and $\set{\ve^a} ⊂ V$ are dual bases, and juxtaposition denotes the geometric product.
If $F$ is a homogeneous $k$-vector, then we may write its components as
\begin{math}
	F = F_{\etc{a_\i}{}k} \etc{\ve^{a_\i}}∧k
\end{math}
and hence
\begin{align}
	\vd F = ∂_{\ve_a}F_{\etc{a_\i}{}k} \, \ve^a(\etc{\ve^{a_\i}}∧k)
.\end{align}
Note that these terms are not $(k + 1)$-blades, but geometric products of vectors $\ve^a$ with $k$-blades --- in general, $(k ± 1)$-multivectors.

We may regard the vector derivative itself as an operator-valued vector,
\begin{align}
	\vd = \ve^a ∂_a
,\end{align}
reflecting the fact that $\vd$ behaves algebraically like a vector.
For instance, the derivative of a vector $𝒖$ has scalar and bivector parts,
\begin{math}
	\vd 𝒖 = \vd \fatdot 𝒖 + \vd ∧ 𝒖
,\end{math}
just like the geometric product of two vectors, $𝒖𝒗 = 𝒖 \fatdot 𝒗 + 𝒖 ∧ 𝒗$.
For a general multivector $F$, then, we have
\begin{align}
	\vd F = \vd \lcontr F + \vd ∧ F
.\end{align}


\subsection{Case Study: Maxwell's Equations}


\section{Integration}


\subsection{Stokes' Theorem for Exterior Calculus}

\begin{theorem}[Stokes' theorem in $\RR^n$]
	\label{thm:flat-stokes}
	If $R ⊆ \RR^n$ is a compact $k$-dimensional hypersurface with boundary $∂R$, then a smooth differential form $ω ∈ \forms[k - 1](R)$ satisfies
	\begin{align}
		\label{eqn:stokes}
		\int_R \dd ω = \int_{∂R} ω
	.\end{align}
\end{theorem}
\begin{proof}
	Since $R$ is a $k$-dimensional region with boundary, every point $x ∈ R$ has a neighbourhood diffeomorphic to a neighbourhood of the origin in either $\RR^k$ or $H^k ≔ [0, ∞) ⊕ \RR^{k - 1}$, depending on whether $x$ is an interior point or a boundary point, respectively.

	\begin{marginfigure}
		\centering
		\includefigure[\columnwidth]{stokes-theorem}
		\caption{
			Neighbourhoods in $R$ are diffeomorphic either to interior balls or boundary half-balls.
		}
		\label{fig:stokes-theorem}
	\end{marginfigure}

	Let $\set{U_i}$ be a cover of $R$ consisting of such neighbourhoods.
	Since $R$ is compact, we may assume $\bigcup_{i=1}^N \set{U_i} = R$ to be a finite covering.
	Thus, we have finitely maps $h_i : U_i → X$ where $X$ is either $\RR^k$ or the half-space $H^k$, where $U_i \cong h_i(U_i)$ are diffeomorphic (see \cref{fig:stokes-theorem}).

	Finally, let $\set{ϕ_i : R → [0, 1]}$ be a partition of unity subordinate to $\set{U_i}$, so that $\set{x ∈ R | ϕ_i(x) > 0} ⊆ U_i$ and $ω = \sum_{i=1}^N ϕ_iω$.
	We need only prove the equality \eqref{eqn:stokes} for each $ω_i ≔ ϕ_iω$, and the full result follows be linearity.
	
	The form $h_i^*ω_i ∈ \forms[k - 1](X)$ can be written with respect to canonical coordinates of $X$ as
	\begin{align}
		h_i^*ω_i = \sum_{j=1}^k f_j (-1)^{j - 1} \dd x^{1\cdots\hat{j}\cdots k}
	\end{align}
	using the multi-index notation $\dx^{\etc{i_\i}{}k} ≡ \etc{\dx^{i_\i}}∧k$, where the hat denotes an omitted term.
	The factor of $(-1)^{j - 1}$ gives the $(k - 1)$-form the boundary orientation induced by the volume form $\dx^{\etc\i{}k}$ for convenience.
	Since pullbacks commute with $\dd$,
	\begin{align}
		h^*\ddω_i = \dd(h_i^*ω_i) = \sum_{j=1}^k \pdv{f_j}{x^j} \dd x^{1\cdots n}
	.\end{align}
	There are then two cases to consider.
	\begin{itemize}
		\item \emph{Interior case.}
		If $h_i : U_i → \RR^k$, then the right-hand side of \cref{eqn:stokes} vanishes because $ω_i$ is zero outside the neighbourhood $U_i ⊂ R$ which nowhere meets the boundary $∂R$.
		\begin{align}
			\int_{∂R} ω_i = \int_{∂U_i} ω_i = \int_∅ ω_i = 0
		\end{align}
		The left-hand side evaluates to
		\begin{align}
			\int_R \dd ω_i
			&= \int_X \dd (h_i^*ω_i)
			= \int_{\RR^k} \sum_{j=1}^k \pdv{f_j}{x^j} \dd x^{1\cdots n}
		\\	&= \underbrace{\etc{\int_{-∞}^{+∞}}{}{}}_k \sum_{j=1}^k \pdv{f_j}{x^j} \etc{dx^\i}{}{k}
		\\	&= \underbrace{\etc{\int_{-∞}^{+∞}}{}{}}_{k - 1} \sum_{j=1}^k \eval{f_j}_{x^j=-∞}^{+∞} (-1)^{j-1} dx^1\cdots\widehat{dx^j}\cdots dx^k
			= 0
		,\end{align}
		which vanishes because $h_i^*ω_i$, and hence the $f_j$, vanish outside the neighbourhood $h_i(U_i) ⊂ \RR^k$.

	\item \emph{Boundary case.}
	If $h_i : U_i → H^k$, then the boundary $∂U_i ⊂ ∂R$ is mapped onto the hyperplane $∂H^k = \set{(0, \etc[2]{x^\i},k) | x^j ∈ \RR}$.
	Thus, $dx^1 = 0$ on this boundary, and the right-hand side of \cref{eqn:stokes} becomes
	\begin{align}
		\int_{∂R}ω_i
		&= \int_{∂U_i} h_i^*ω_i
		= -\int_{\RR^{k - 1}} f_1 \etc[2]{dx^\i}{}k
	\\	&= -\underbrace{\etc{\int_{-∞}^{+∞}}{}{}}_{k - 1} f_1(0, \etc[2]{x^\i},k) \etc[2]{dx^\i}{}k
	.\end{align}
	The factor of $-1$ comes from the induced orientation of the boundary $∂H^k$, which is outward-facing, so in the \emph{negative} $x^1$ direction.
	For the left-hand side of \cref{eqn:stokes},
	\begin{align}
		\int_R \dd ω_i
		&= \int_{H^k} h_i^*\ddω_i
		= \int_0^∞ \etc{\int_{-∞}^{+∞}}{}{} \sum_{j=1}^k \pdv{f_j}{x^j} \etc{dx^\i}{}k
	\intertext{
		All terms $\pdv{f_j}{x^j}dx^j$ in the sum for $j > 1$ integrate to boundary terms $x_j → ±∞$ where $f_j$ vanishes.
		This leaves the single term from the integration of $dx^1$,
	}
		&= -\etc{\int_{-∞}^{+∞}}{}{} \eval{f_1}_{x^1=0}^∞ \etc[2]{dx^\i}k
	.\end{align}

	\end{itemize}

	Thus, we have equality for all $ω_i$, so
	\begin{align}
		\int_R \dd ω = \sum_{i=1}^N \int_R \dd ω_i = \sum_{i=1}^N \int_{∂R} ω_i = \int_{∂R} ω
	\end{align}
	by linearity.
\end{proof}

\subsection{Fundamental Theorem of Geometric Calculus}