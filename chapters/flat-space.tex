\chapter{Calculus in Flat Geometries}

So far, we have been concerned with special relativity at a single point in spacetime.
We move now toward the description of \emph{fields} --- quantities extending across spacetime.
The first step in this direction is the calculus of \emph{flat spacetime}.
In a flat geometry, we may assume that
\begin{itemize}
	\item points in spacetime form a vector space, with differences of points being physically--meaningful displacement vectors; and that
	\item fields are parametric functions of a point in spacetime.
\end{itemize}
We reserve the word \textdef{field} to mean a map with a \emph{fixed} codomain.
For instance, the electromagnetic bivector \emph{field} in flat space $F : \RR^4 → \EA[2]{\RR^4}$ is a function between vector spaces, and values of $F$ at different points in spacetime belong to the same space, making expressions like $F(x) + F(y) ∈ A$ are well-defined.

These assumptions are acceptable in special relativity, but in arbitrary regions of spacetime and in the presence of gravity, curvature prevents spacetime from admitting a meaningful vector space structure.
It is then \emph{un-physical} to compare field values at different points in spacetime.
(Curvature leads to differential geometry and comprises \cref{part:2}.)

This chapter defines differentiation of fields, introducing the \emph{exterior} and \emph{vector derivatives} as instances of the `algebraic derivative' in the exterior and geometric algebras, respectively.
These devices combine derivative information with the geometrical structure inherent in the respective algebras.
To demonstrate their utility, Maxwell's equations of electromagnetism are exhibited in both algebras.


\section{Differentiation of Fields}
\label{sec:algder}

The derivative of a vector field $F : V → A$ in the direction $𝒖 ∈ V$ at $𝒙 ∈ V$ may be defined in the usual way,
\begin{align}
	% ∂_𝒖 F(x) = \lim_{ε → 0}\frac{F(x + ε𝒖) - F(x)}{ε}
	∂_𝒖 F(𝒙) = \dv{ε} \eval{F(𝒙 + ε𝒖)}_{ε = 0}
	= \lim_{ε → 0}\frac{F(𝒙 + ε𝒖) - F(𝒙)}{ε}
.\end{align}
The directional derivative is linear in both its argument and direction.\sidenote{
	By a change of variables,
	\begin{math}
		∂_{u^a\ve_a}
		= \dv{ε} \eval{F(x + εu^a\ve_a)}_{ε = 0}
		= u^a \dv{\bar ε} \eval{F(x + \bar ε\ve_a)}_{\bar ε = 0}
		= u^a ∂_{\ve_a}
	\end{math}
	(summation on $a$).
}
We define the notation $∂_a ≔ ∂_{\ve_a}$ for brevity, so long as it is understood that this is not a partial derivative with respect to a scalar coordinate, $\pdv{x^a}$.
Of course, it may be viewed as such by setting $f(\etc{x^\i},n) = f(x^i\ve_i)$ so that
\begin{align}
	∂_{\ve_a}f(x^i\ve_i) = \pdv{x^a}f(\etc{x^\i},n)
,\end{align}
though this is a basis-dependent definition.


Suppose $F : V → A$ is some algebra--valued field.
It is useful to define a kind of ``total'' derivative $\DD F$ which does not depend on a direction $𝒖$, but instead encompasses, in a sense, all directional derivatives in a single object $\DD F : V → A$.
The motivation for this is that the soon-to-be-defined exterior derivative (of exterior algebra) and vector derivative (of geometric algebra) are realised as special cases of such a construction.
The derivative $\DD$ will be defined given an inclusion $ι : V^* → A$ of dual vectors into the algebra.

\begin{definition}
	\label{def:algder}
	Let $F : V → A$ be a field with values in an algebra $A$ with product $⊛$, equipped with an inclusion $ι : V^* → A$.
	The \textdef{algebraic derivative} of $F$ is
	\begin{align}
		\label{eqn:algder}
		\DD F ≔ ι(\ve^a) ⊛ ∂_{\ve_a} F
	\end{align}
	(summation on $a$) where $\set{\ve_a} ⊂ V$ and $\set{\ve^a} ⊂ V^*$ are dual bases.
\end{definition}

To understand this definition, consider the simple case of the free tensor algebra $F : V → \TA{(V^*)}$.
We leave the canonical inclusion $ι : V^* → \TA{(V^*)}$ implicit.
Given a basis $\set{\ve^a} ⊂ V^*$, the algebraic derivative is
\begin{math}
	\DD F = \ve^a ⊗ ∂_a F
,\end{math}
which simply encodes the partial derivatives of a $k$-vector $F$ in a $(k + 1)$-grade object.
In component language,
\begin{math}
	(\DD F)_{a\etc{a_\i}{}k} = ∂_aF_{\etc{a_\i}{}k}
.\end{math}
\Cref{def:algder} becomes more interesting when the algebra's product $⊛$ carries more structure.

\subsection{The exterior derivative}
\label{sec:exterior-derivative-forms}

Consider a vector field $F : V → \EA{V^*}$ with values in the (dual) exterior algebra.
The algebraic derivative in this case is called the \textdef{exterior derivative $\dd$}, and \cref{eqn:algder} takes the form
\begin{align}
	\dd F = \ve^a ∧ ∂_a F
,\end{align}
where $\set{\ve^a} ⊂ V^*$ also form a basis of $\EA{V^*}$ (so $ι : V^* → \EA{V^*}$ may be omitted).
More explicitly, if $F$ is a $k$-vector field, then
\begin{math}
	\dd F = ∂_a F_{\etc{a_\i}{}k} \ve^a ∧ \etc{\ve^{a_\i}}∧k
\end{math}
is a $(k + 1)$-vector.

Viewing $\EA{V^*}$ as the subspace of antisymmetric tensors (see \cref{sec:exterior-algebra-as-antisymmetric}), the exterior derivative is the totally anti-symmetrised partial derivative.
\marginnote{
	$[\cdots]$ denotes anti\hyp symmetrisation over the enclosed indices.
	$A_{a[b_1\cdots b_k]} = \frac1{k!}\sum_{σ ∈ S_k} (-1)^σ A_{a[b_{σ(1)}\cdots b_{σ(k)}]}$
}
In components,
\begin{math}
	(\dd F)_{\etc{a_\i}{}k} = ∂_{[a_1}F_{\etc[2]{a_\i}{}k]}
.\end{math}

The treatment of exterior forms is identical.
On an exterior form field $φ : V → \forms[k](V, U)$, the exterior derivative i formally defined by its action on vectors,
\begin{align}
	(\dd φ)(𝒖_0, \etc{𝒖_\i},{k})
	&= (\ve^a ∧ ∂_a φ)(𝒖_0, \etc{𝒖_\i},k)
\\	&= \frac1{k!} \sum_{σ ∈ S_{k + 1}} (-1)^σ \ve^a(𝒖_{σ(0)}) \, ∂_a φ(\etc{𝒖_{σ(\i)}}{}k)
\\	&= \sum_{i = 0}^k (-1)^i \, ∂_{𝒖_{i}} φ(𝒖_0, ..., \widehat{𝒖_i}, ..., 𝒖_k)
,\end{align}
under the Spivak convention (see \cref{sec:normalisation-conventions}).
Note that the directional derivative acts on the position dependence of $φ$ only --- the vectors $𝒖_i ∈ V$ are \emph{fixed} input vectors to the field $\dd φ$.
This changes when generalising to forms defined on a \emph{manifold}, where correction terms are needed to account for partial derivatives of input vectors (discussed in \cref{sec:forms-on-manifolds}).


\subsection{The vector derivative}

The algebraic derivative in the tensor and exterior algebras are somewhat uninteresting because they are easily expressible in component form (e.g., $∂_aF_{\etc{a_\i}{}k}$ or $∂_{[a}F_{\etc{a_\i}{}k]}$).
This is not possible in the geometric algebra, however, because $\GA(V, η)$ is not $\ZZ$-graded, and we would face the problem of notating inhomogeneous objects with a variable number of indices.
The algebraic derivative is, however, still geometrically significant and extremely useful in geometric algebra.

In $\GA(V, η)$, the algebraic derivative is called the \textdef{vector derivative}, denoted $\vd$.
Explicitly, if $F : V → \GA(V, η)$ is a multivector field, then in \cref{eqn:algder} $⊛$ is the geometric product and we take inclusion\sidenote{
	We could just as well consider fields $V → \GA(V^*, η)$, avoiding the need for $\sharp : V^* → V$.
	But the metric is already defined, and we prefer to think about multivectors over `dual multivectors'.
}
\begin{align}
	V^* ∋ 𝒖 ↦ ι(𝒖^\sharp) ∈ \GA(V, η)
.\end{align}
Here, we use the canonical inclusion $ι : V ≡ \GA[1](V, η) → \GA(V, η)$ and the metric to relate $V^* → V$.
The vector derivative then reads
\begin{align}
	\vd F = \ve^a \, ∂_{\ve_a} F
\end{align}
(summation on $a$) where $\set{\ve_a} ⊂ V$ and $\set{\ve^a} ⊂ V^*$ are dual bases, and juxtaposition denotes the geometric product.
If $F$ is a homogeneous $k$-vector, then we may write its components as
\begin{math}
	F = F_{\etc{a_\i}{}k} \etc{\ve^{a_\i}}∧k
\end{math}
and hence
\begin{align}
	\vd F = ∂_{\ve^a}F_{\etc{a_\i}{}k} \, \ve^a(\etc{\ve^{a_\i}}∧k)
.\end{align}
Note that these terms are not $(k + 1)$-blades, but geometric products of vectors $\ve^a$ with $k$-blades --- in general, $(k ± 1)$\hyp multivectors.

We may regard the vector derivative itself as an operator-valued vector,
\begin{align}
	\vd = \ve^a ∂_a
,\end{align}
reflecting the fact that $\vd$ behaves algebraically like a vector.
For instance, the derivative of a vector $𝒖$ has scalar and bivector parts,
\begin{math}
	\vd 𝒖 = \vd \fatdot 𝒖 + \vd ∧ 𝒖
,\end{math}
just like the geometric product of two vectors, $𝒖𝒗 = 𝒖 \fatdot 𝒗 + 𝒖 ∧ 𝒗$.
For a general multivector $F$, then, we have
\begin{align}
	\vd F = \vd \lcontr F + \vd ∧ F
.\end{align}
The $(k + 1)$-grade part $\vd ∧ F$ is the \emph{curl} of $F$, and coincides with the exterior derivative $\dd F$.
The $(k - 1)$-grade part involves the metric, and can be related to the `interior' derivative ${\star}{\dd}{\star} A$ via Hodge duality.\sidenote{
	Observe that
	\begin{math}
		\vd \lcontr A = \grade[k - 1]{\vd \vol^{-1}\vol A} = ±\vol\grade[1 + n - k]{\vd (\vol A)} = ± \vol \, \vd ∧ (\vol A)
	;\end{math}
	also see \cref{sec:geoprod-in-ea}.
}
Indeed, using \cref{eqn:geoprod-in-ea}, the vector derivative may be emulated in the exterior algebra by the combination
\begin{align}
	\vd F ≡ \star^{-1} \dd \star F + \dd F
,\end{align}
although it is easier to treat it as a vector in the geometric algebra.

\section{Case Study: Maxwell's Equations}


Expressed in the standard vector calculus of $\RR^3$, Maxwell's equations for the electric $\vb E$ and magnetic $\vb B$ fields in the presence of a source are
\begin{align}
	∇ \cdot \vb E &= \frac{ρ}{ε_0} &&\text{(Gauß' law)}
\\	∇ \cdot \vb B &= 0 &&\text{(Absence of magnetic monopoles)}
\\[1ex]	∇ × \vb E &= -∂_t \vb B &&\text{(Faraday's law)}
\\[1ex]	∇ × \vb B &= μ_0(\vb J + ε_0∂_t \vb E) &&\text{(Ampère's law)}
\end{align}
where $ρ$ is the scalar charge density and $\vb J$ the current density.
The constants $ε_0$ and $μ_0$ are the vacuum permittivity and permeability, respectively, related to the speed of light by $ε_0μ_0c^2 = 1$.




\pagebreak
\subsection{With tensor calculus}

\begin{margintable}
	\footnotesize
	\begin{tabular}{cl}
		\\
		\multicolumn{2}{c}{\emph{Non-relativistic}} \\
		quantity & dimension \\
		$\vb E$ & $MQ^{-1}LT^{-2}$ \\
		$\vb B$ & $MQ^{-1}T^{-1}$ \\
		$ρ$ & $QL^{-3}$ \\
		$\vb J$ & $QT^{-1}L^{-2}$ \\
		$μ_0$ & $MQ^{-2}L$ \\
		$ε_0$ & $M^{-1}Q^2L^{-3}T^2$ \\
		$∇$, $∂_t$ & $L^{-1}$, $T^{-1}$ \\
		$c$ & $LT^{-1}$ \\
		\\
		\multicolumn{2}{c}{\emph{Relativistic}} \\
		quantity & dimension \\
		$F$ & $MQ^{-1}S^{-1}$ \\
		$J$ & $QS^{-3}$ \\
		$μ_0$, $ε_0^{-1}$ & $MQ^{-2}S$ \\
		$∂$ & $S^{-1}$ \\
		$c$ & $1$ \\
	\end{tabular}
	\caption{
		Dimensions of physical quantities in Maxwell's equations.
		$M$ is mass, $Q$ is electric charge, $T$ is duration and $L$ is length.
		In the relativistic formulation, $T$ and $L$ are unified and replaced by \emph{spacetime interval} $S$.
	}
\end{margintable}

The above can be expressed relativistically as eight scalar equations,
\setlength{\fboxsep}{1.4ex}
\begin{empheq}[box=\fbox]{align}
	\label{eqn:maxwells-tensor-form}
	∂_μF^{μν} &= μ_0J^ν
,&	∂_μG^{μν} &= 0
\end{empheq}
where $F^{μν} = -F^{νμ}$ is the Faraday tensor and $G^{μν}$ its Hodge dual, both encoding the electric and magnetic fields via
\begin{align}
	\label{eqn:components-of-electromagnetic-tensor}
	F^{i0} &= \frac{E^i}{c}
,&	F^{ij} &= -ε^{ijk}B_k
,&	G^{μν} &= \frac12 ε^{μν}{}_{ρσ} F^{ρσ}
% ,&	G^{μν} &= \frac12 η^{μρ}η^{νσ}ε_{ρσλυ} F^{λυ}
,\end{align}
and where $J^μ$ encodes both the static charge density $J^0 = cρ$ and current density $J^i = \vb J$.
The left of eqs.~\eqref{eqn:maxwells-tensor-form} is the \emph{source equation}, while the right is the \emph{second Bianchi identity}.
These equations assume the metric signature $\sig{+---}$, where the equivalent equations under $\sig{-+++}$ are obtained by a change of sign $F^{μν} \mapsto -F^{μν}$.



\begin{proof}
	We show how the relativistic equations \eqref{eqn:maxwells-tensor-form} reduce to the non-relativistic vector calculus equivalents.
	The $0$-component of the source equation is
	$∂_μ F^{μ0} = ∂_iE^i/c = μ_0J^0 = μ_0cρ$ implying $∇ · \vb E = ρ/ε_0$ (Gauß' law).
	The $i$-components are
	\begin{align}
		∂_0F^{0i} + ∂_jF^{ji} &= \frac1c ∂_t \qty(-\frac{E^i}{c}) - ∂_j ε^{jik} B_k = μ_0 J^i
\\		\qqtext{or} ∂_jε^{ijk}B_k &= μ_0 J^i + μ_0ε_0∂_tE^i
	,\end{align}
	which is equivalent to Ampère's law.
	The $0$-component of the Bianchi identity $∂_μG^{μ0} = 0$ is
	\begin{align}
		% \frac12 ε^{i0}{}_{μν}∂_iF^{μν}
		\frac12 ε^i{}_{jk}∂_iF^{jk}
		= -\frac12 ε^i{}_{jk}ε^{jkl}∂_iB_l
		= -∂_iB^i = 0
	,\end{align}
	which using the identity $ε_{ijk}ε^{jkl} = 2δ^l_i$ is $∇ · \vb B = 0$.
	Finally, the $i$-component gives
	\begin{align}
		0 = ∂_μG^{μi} &= \frac12ε^{μi}{}_{ρσ}∂_μF^{ρσ}
		= \frac12ε^{0i}{}_{jk}∂_0F^{jk} + ε^{ji}{}_{k0}∂_jF^{k0}
	\\	&= -\frac14ε^i{}_{jk}ε^{jkl}∂_0B_l - \frac1{2c} ε^{ijk}∂_jE_k
		= -\frac1{2c}\qty(∂_tB^i  + ε^{ijk}∂_jE_k)
	\end{align}
	yielding Faraday's law $∇ × \vb E = -∂_t \vb B$.
\end{proof}



\subsection{With exterior calculus}

It is easy to translate between exterior calculus and tensor calculus by identifying the former as the subalgebra of totally antisymmetric tensors (as in \cref{sec:exterior-algebra-as-antisymmetric}).
We will employ the Spivak convention, which in particular identifies $2$-forms with tensors via
\begin{math}
	\ve^μ ∧ \ve^ν ≡ \ve^μ ⊗ \ve^ν - {\ve^ν ⊗ \ve^μ}
\end{math}
where $\ve^μ$ are spacetime basis vectors (having physical dimensions of spacetime interval, $S$).
We then identify the electromagnetic bivector as
\begin{math}
	\mathcal{F} = \frac12 F_{μν} \ve^μ ∧ \ve^ν
\end{math}
(the $\frac12$ is omitted in the Kobayashi–Nomizu convention).


Since the charge density $J \sim QS^{-3}$ has dimensions of charge per spacetime $3$-volume, it is natural to interpret it as a \emph{trivector}
\begin{align}
	\mathcal{J} = J^{μνλ} \, \ve_μ ∧ \ve_ν ∧ \ve_λ ≔ J^μ \star \ve_μ = \frac1{3!}ε_{μνλα}J^α \ve^μ ∧ \ve^ν ∧ \ve^λ
\end{align}
so that the coefficients $J^{μνλ} \sim Q$ have dimensions of charge.\sidenote{
	Note that dual vectors $\ve_μ$ have dimension $S^{-1}$.
}

The relativistic Maxwell equations are then
\begin{empheq}[box=\fbox]{align}
	\dd \star \mathcal{F} &= μ_0 \mathcal{J}
,&	\dd \mathcal{F} &= 0
.\end{empheq}
\begin{proof}
	The first equation written in component form is
	\begin{align}
		\frac14 ε_{μνρσ}∂_λF^{ρσ} &= \frac1{3!}ε_{λμνα}μ_0J^α
	,\end{align}
	which, by contracting with $ε^{μνλβ}$ and using the identities
	\begin{math}
		ε^{μνλβ}ε_{μνρσ} = 2(δ^λ_ρδ^β_σ - δ^λ_σδ^β_ρ)
		\text{ and }
		ε^{μνλβ}ε_{λμνα} = 3!δ^β_σ
	,\end{math}
	reduces to
	\begin{align}
		\frac12(∂_λF^{λβ} - ∂_λF^{βλ}) &= μ_0J^β
	\end{align}
	or $∂_μF^{μν} = μ_0J^ν$, the source equation.
	The Bianchi identity can be rewritten as
	\begin{align}
		∂_μG^{μν}
		= \frac12ε^{μν}{}_{ρσ}∂_μF^{ρσ}
		= -\frac12ε^{ν[μρσ]}∂_μF_{ρσ}
		= -\frac12ε^{νμρσ}∂_{[μ}F_{ρσ]}
		= 0
	,\end{align}
	implying $\dd \mathcal{F} = 0$.
\end{proof}


\subsection{With geometric calculus}

Using the spacetime algebra $\GA(1,3)$ with vector basis $\set{\vg_μ}$ as introduced in \cref{cha:sta}, the electromagnetic bivector is\sidenote{
	This coincides with the electromagnetic bivector $2$-form $\mathcal{F}$ in the Kobayashi–Nomizu convention, because the wedge product in geometric algebra is naturally normalised (see \cref{tbl:wedge-conventions}).
}
\begin{align}
	\label{eqn:ga-em-bivector}
	F = F^{μν} \vg_μ\vg_ν
\end{align}
and the current density is
\begin{align}
	\vb J = J^μ \vg_μ
.\end{align}
Maxwell's equations are equivalent to the \emph{single} multivector equation
\begin{empheq}[box=\fbox]{align}
	\label{eqn:maxwell-sta}
	\vd F = μ_0\vb J
.\end{empheq}
\begin{proof}
	The multivector equation $\vd F = μ_0\vb J$ separates into a vector part 
	\begin{math}
		{\vd \cdot F} = μ_0\vb J
	\end{math}
	and a trivector part
	\begin{math}
		\vd ∧ F = 0
	.\end{math}
	In terms of components, the left-hand side of the vector part is
	\begin{align}
		\vd \cdot F = ∂_λF^{μν} \vg^λ \cdot (\vg_μ\vg_ν)
	,\end{align}
	whose only non-zero components are those for which $μ \ne ν$.
	If $λ$, $μ$ and $ν$ are all distinct, then $\vg^λ \cdot (\vg_μ\vg_ν) = \grade[1]{\vg^λ\vg_μ\vg_ν} = 0$.
	There are then two cases, $λ = μ$ and $λ = ν$, which respectively simplify to
	\begin{align}
		\vg^μ \cdot (\vg_μ\vg_ν) &= \grade[1]{\vg^μ\vg_μ\vg_ν} = \vg_ν
	,\\	\vg^ν \cdot (\vg_μ\vg_ν) &= \grade[1]{\vg^ν\vg_μ\vg_ν} = -\vg_μ
	,\end{align}
	so that
	\begin{align}
		\vd \cdot F = \qty(∂_μF^{μν}\vg_ν - ∂_νF^{μν}\vg_μ) = ∂_μF^{μν}\vg_ν
	.\end{align}
	Equality with the right-hand side $μ_0J^ν\vg_ν$ recovers the source equation.

	It is clear that the trivector part
	\begin{align}
		\vd ∧ F = ∂_λF^{μν} \vg^λ ∧ (\vg_μ\vg_ν) = ∂_λF_{μν} \vg^λ ∧ \vg^μ ∧ \vg^ν = 0
	\end{align}
	is equivalent to the exterior algebraic Bianchi identity $\dd \mathcal{F} = 0$.
\end{proof}


\subsubsection{In terms of electric and magnetic fields}



It is worth showing how the relativistic Maxwell equation \eqref{eqn:maxwell-sta} splits into a frame-dependent description in the geometric algebra framework.
As in \cref{sec:spacetime-split}, we use the notation $\vec u$ to indicate relative vectors; i.e., timelike bivectors
of the spacetime algebra $\GA(1,3)$ which are simultaneously grade-$1$ vectors in the observer's algebra $\GA(3)$.

From \cref{eqn:ga-em-bivector,eqn:components-of-electromagnetic-tensor}, the electromagnetic bivector is expressed in the $\vg_0$-frame as\sidenote{
	We assume $\sig{+---}$ for concreteness; for $\sig{-+++}$ replace $F \mapsto -F$.
}
\begin{align}
	\label{eqn:F-is-E-iB}
	F = \frac1c \vec E + \vol \vec B
,\end{align}
where
\begin{math}
	\vec E = E^i\vs_i = E^i \vg_i\vg_0
\end{math}
and
\begin{align}
	\vol\vec B = B_i \vol \vs^i = \frac12 B_i ε^{ijk}\vs_j\vs_k
	= \frac12 B_i ε^{ijk}\vg_j\vg_k
.\end{align}
\Cref{eqn:F-is-E-iB} should be compared with the Riemann-Silberstein vector \cite{silberstein} which has the form $\vec F_\CC = \vec E + ic\vec B$.

The current density spacetime vector $J$ may be viewed under the space\slash time split by (left) multiplying by the frame velocity $\vg_0$,
\begin{align}
	\vg_0 \vb J = cρ - \vec J
,\end{align}
where $J^0 = cρ$ and $\vec J = J^i \vs_i$.
Similarly for the vector derivative, we have
\begin{align}
	\vg_0 \, \vd = \frac1c \pdv{t} + \vec ∇
\end{align}
in either signature.

Putting these together, the $\vg_0$-frame equation
\begin{math}
	\vg_0 \, \vd F =  μ_0 \vg_0 \vb J
\end{math}
is
\begin{align}
	\qty(\frac1c \pdv{t} + \vec ∇)\qty(\frac1c \vec E + \vol \vec B) &= μ_0\qty(cρ - \vec J)
.\end{align}
By expanding and equating grades, we instantly obtain four equations:
\begin{align}
	\frac1c \vec ∇ · \vec E &= μ_0 cρ
&	&\text{(scalar)}
\\	\frac1{c^2} \pdv{\vec E}{t} + \vol (\vec ∇ ∧ \vec B) &= -μ_0 \vec J
&	&\text{(vector)}
\\	\frac1c \vec ∇ ∧ \vec E + \frac\vol c \pdv{\vec B}{t} &= 0
&	&\text{(bivector)}
\\	\vol (\vec ∇ · \vec B) &= 0
&	&\text{(pseudoscalar)}
\end{align}
Note that the cross product relates to the bivector curl in $\GA(3)$ by
\begin{align}
	𝒖 ∧ 𝒗 = \vol(𝒖 × 𝒗)
	\qqtext{so that}
	∇ × 𝑿 = -\vol(\vec ∇ ∧ \vec X)
.\end{align}
Hence, by adjusting by factors of $c$ and $\vol$ (and using $μ_0ε_0c^2 = 1$), the above equations reduce immediately to Gauß's law, Ampère's law, Faraday's law and the magnetic monopole equation, respectively.

The calculations in this section were performed assuming the metric $η = \op{diag}\sig{+---}$.
In the $\sig{-+++}$ signature, $\vg_0 \vb J = -cρ + \vec J$ differs by an overall sign, which is absorbed by the change of sign $F \mapsto -F$.



\subsubsection{Circularly polarised plane wave solutions}

In a vacuum, Maxwell's equation
\begin{align}
	\label{eqn:maxwell-vacuum-split}
	\qty(\frac1c \pdv{t} + \vec ∇)F = 0
\end{align}
admits plane wave solutions
\begin{align}
	\label{eqn:plane-wave}
	F_± = F_0 e^{±\vol(ωt - \vec k \cdot \vec x)}
,\end{align}
where $ω > 0$ is the frequency and $\vec k$ the wave vector.
It should be emphasised that, in the geometric algebra, \cref{eqn:plane-wave} is a \emph{real} multivector --- we are not invoking the unit imaginary $i$, and do not implicitly take the real part of $F_±$ at the end of calculations.
Instead, the `complex plane' is replaced with something geometrical: the $\vec E$-$\vec B$ plane.
Indeed, from the geometric meaning inherent in the algebra, the solution \eqref{eqn:plane-wave} necessarily describes \emph{circularly polarised light}, with the $\vec E$ and $\vec B$ vectors rotating within the plane normal to the propagation direction \cite{hestenes1971sta}.

This can be established by substituting the plane wave \cref{eqn:plane-wave} into \cref{eqn:maxwell-vacuum-split} to get
\begin{align}
	±\vol \qty(\fracωc - \vec k) F = 0
.\end{align}
The condition $(ω/c - \vec k)F = 0$ encodes several geometrical relationships.
Firstly, by multiplying on the left by $(ω/c + \vec k)$, we see that
\begin{align}
	\qty(\frac{ω^2}{c^2} - k^2) F = 0
\end{align}
which, since $F ≠ 0$ gives the expected dispersion relation $ω = c\|\vec k\|$.
Hence, by dividing by the magnitude of $\vec k$, we have
\begin{math}
	(1 - \hat{k})F = 0
\end{math}
where $\hat{k}^2 = 1$.
Reintroducing the unknown electric and magnetic field vectors, this implies
\begin{align}
	(1 - \hat{k})(\vec E + \vol \vec B)
	= \underbrace{\vec E }_1 + \underbrace{\vol \vec B}_2 - \underbrace{\hat{k} \vec E}_{0,2} - \underbrace{\hat{k} \vol \vec B}_{1,3}
	= 0
,\end{align}
where the grades of terms as multivectors in $\GA(3)$ are indicated.
Taking only the even or odd parts yields the condition
\begin{align}
	\hat{k}\vec E = \vol \vec B
,\end{align}
which implies two things: firstly, by multiplying both sides by their reverse, we see that $\|\vec E\| = \|\vec B\|$; secondly, by dividing on the right by the vector $\vec B$ to obtain
\begin{align}
	\hat{k} \hat E \hat B = \vol
,\end{align}
we conclude that $(\hat{k}, \hat E, \hat B)$ forms a right-handed orthonormal frame.

Finally, to see the time dependence, evaluate the solution on the $\vec k \fatdot \vec x = 0$ plane,
\begin{math}
	F_+(t) = F_0 e^{-\vol ωt}
\end{math}
and expand noting that $\vol \vec B_0 = \hat{k} \vec E_0 = -\vec E_0 \hat{k}$.
\begin{align}
	\vec E(t) + \vol \vec B(t)
	&= (\vec E_0 + \vol \vec B_0) (\cos ωt + \vol \sin ωt)
\\	&= (\vec E_0 + \vol \vec B_0) \cos ωt + (\vol \vec E_0 - \vec B_0) \sin ωt
% \\	&= \mp \vec E_0 (1 - \hat {k}) (\cos ωt + \vol \sin ωt)
\end{align}
Taking only the vector part of this equation yields
\begin{align}
	\vec E(t) = \vec E_0 \cos ωt - \vec B_0 \sin ωt
.\end{align}
Thus, looking toward the approaching plane wave $F_+(t)$ moving in the $\hat{k}$ direction, the $\vec E(t)$ and hence $\vec B(t)$ vectors are rotating clockwise; for $F_-(t)$, anticlockwise.