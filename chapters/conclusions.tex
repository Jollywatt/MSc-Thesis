\chapter{Conclusions}

The focal result of \cref{part:1} was the discovery of a relatively simple \x{BCH} formula for Lorentz transformations (indeed, for proper orthogonal transformations in any space of dimension at most four) \cite{wilson2021ga-bch}.
The key to this was the rotor formalism, where transformations $\lin Λ(𝒖) = R𝒖\rev{R}$ are represented by rotors in the double covering spin group.
This allows for a more elegant `arithmetic of rotations' via the geometric algebra.
In the case of $(1 + 3)$-dimensional spacetime, the algebra's linear representation by complex $2 × 2$ matrices makes the formula easy to use implement numerically.

The \x{BCH} formula is also useful algebraically; in \cref{sec:wigner} it was used to directly derive the Wigner angle of the rotation resulting from the composition of Lorentz boosts.
This is facilitated by the space\hyp time split, whereby Lorentz boosts are generated by spacelike vectors and rotations by spacelike bivectors --- objects with clear geometric meaning which are easy to work with.

% The scope is then expanded to include fields in flat space,and the vector derivative $\vd$ is related the standard exterior derivative and is dual by $\vd = \star^{-1} ∘ \dd ∘ \star + \dd$.
%  exhibiting Maxwell's equations in both algebras.

Expanding the scope to include curved spaces in \cref{part:2}, the geometric algebra is used to rewrite the Lie and covariant derivatives in invariant, basis-free ways.
Like Cartan's formula \eqref{eqn:magic-formula} for differential forms, the Lie derivative admits a formula $\lie_𝒖 A = [𝒖, A] = ∂_𝒖 A + (A \lcontr \vd) ∧ 𝒖$ for multivectors of any grade.
Similarly, the covariant derivative of a multivector has the invariant form $\df ∇ A = \df\dd A + \df ω × A$ when expressed in terms of the connection bivector $1$-form.

Finally, the curvature $2$-form is exposited in two interesting ways: as an obstruction to integrability, and as the surface-ordered integrand appearing in \cref{thm:nast}.
Sections on a manifold can be integrated over manifolds by parallel transporting values to a common fibre --- but in the presence of curvature, this is only possible along $1$-dimensional curves.
The path-dependence of parallel transport induced by curvature means a `surface ordering' is needed to integrate sections over surfaces.
A special case of this the Stokes-like theorem for curvature $2$-forms, adapted from \cite{bralic1980nast}, which relates the curvature over a surface to the holonomy around its boundary.
