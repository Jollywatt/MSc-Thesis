\chapter{Composition of Rotors in terms of their Generators}
\label{cha:bch}

In studying proper orthogonal transformations, it is often easier to represent them in terms of their generators $σ_i ∈ \GA(p,q)$ which belong to the Lie algebra $\liealg{so}(p,q)$.
A fundamental question is how such transformations compose in terms of these generators: ``given $σ_1$ and $σ_2$, what is $σ_3$ such that $e^{σ_1}e^{σ_2} = e^{σ_3}$?''
This is of theoretical interest and is useful practically when representing transformations in terms of their generators is cheaper.
One may use the \x{BCH full}\sidenote{
	Often simply Baker--Campbell--Hausdorff and permutations thereof.
} (\x{BCH}) formula $\bch{σ_1}{σ_2} ≔ \log(e^{σ_1}e^{σ_2})$ which is well studied in general Lie theory \cite{achilles2012bch-early}.
However, the general \x{BCH} formula
\begin{align}
	\bch{a}{b} = a + b + \frac12[a, b] + \frac{1}{12}[a, [a, b]] + \frac{1}{12}[[a, b], b] + \cdots
	\label{eqn:general-BCH-formula}
\end{align}
involves an infinite series of nested commutators and may not obviously admit a useful closed form.

In the case of Lorentz transformations $\SO^+(1,3)$, some closed-form expressions for \cref{eqn:general-BCH-formula} have been found using a $2$-form representation of $\liealg{so}(1,3)$ \cite{coll2002sr-generator-composition,coll1990sr-generator-exponentiation}, but the expressions are complicated and do not clearly reduce to well-known formulae in, for example, the special cases of pure rotations or pure boosts.
The rotor formalism of geometric algebra leads to an elegant closed form of \cref{eqn:general-BCH-formula} which, in the case of Lorentzian spacetime, is inexpensive to compute.




\section{A Geometric \x{BCH} Formula}
\label{sec:bch-derivation}

Suppose $σ ∈ \GA[2](p, q)$ is a bivector in a geometric algebra of dimension $p + q ≤ 4$.
By their definitions as formal power series, we have
\begin{math}
	e^{σ} = \cosh σ + \sinh σ
,\end{math}
where `$\cosh$' involves even powers of $σ$ and `$\sinh$' odd powers.
For convenience, define the linear projections onto \textdef{self-reverse} and \textdef{anti-self-reverse} parts respectively as
\begin{align}
	\srev{A} &≔ \frac12\qty(A + \rev{A})
&	&\text{and}
&	\arev{A} &≔ \frac12\qty(A - \rev{A})
	\label{eqn:rev-notation}
.\end{align}
Since any bivector obeys $\rev{σ} = -σ$, it follows that $\rev{(e^σ)} = e^{-σ} = \cosh σ - \sinh σ$.
Using the notation \eqref{eqn:rev-notation}, the self-reverse and anti-self-reverse projections of $e^σ$ are $\srev{e^{σ}} = \cosh σ$ and $\arev{e^{σ}} = \sinh σ$, respectively.
Furthermore, these two projections commute, and so
\begin{align}
	{\arev{e^{σ}}}{\srev{e^{σ}}}^{-1}
	= {\srev{e^{σ}}}^{-1}{\arev{e^{σ}}}
	= \frac{\arev{e^{σ}}}{\srev{e^{σ}}}
	= \tanh σ
\end{align}
which leads to an expression for the logarithm of any rotor $\rotor R = ± e^σ$.
\begin{align}
	σ = \log(\rotor R) = \arctanh\qty(\frac{\arev{\rotor R}}{\srev{\rotor R}})
	\label{eqn:log-rotor}
\end{align}
Note that the overall sign of the rotor is not recovered, and $\log(+\rotor R) = \log(-\rotor R)$ according to \cref{eqn:log-rotor}.
However, this does not affect the Lorentz transformation $\lin R ∈ \SO^+(p, q)$, since it is defined by $R(𝒖) = \rotor R𝒖\rev{\rotor R}$.
The exact sign can be recovered by considering the relative signs of $\arev{\rotor R}$ and $\srev{\rotor R}$, as in \cite[§5.3]{lasenby2011ga-practical}.


From \cref{eqn:log-rotor} we may derive a \x{BCH} formula by substituting $\rotor R = e^{σ_1}e^{σ_2}$ for any two bivectors $σ_i ∈ \GA[2](p, q)$.
Using the shorthand $\Co{i} ≔ \cosh σ_i$ and $\Si{i} ≔ \sinh σ_i$, the composite rotor is
\begin{align}
	\rotor R = e^{σ_1}e^{σ_2}
	= (\Co1 + \Si1)(\Co2 + \Si2)
	= \Co1\Co2 + \Si1\Co2 + \Co1\Si1 + \Si1\Si2
.\end{align}
For $p + q < 4$, any even function of a bivector (such as $\Co{i}$) is a scalar, and for $p + q = 4$, is a $\qty{0,4}$\hyp multivector $α + β\vol$.
In either case, the $\Co{i}$ commute with even multivectors, so $[\Co{i}, \Co{j}] = [\Co{i}, \Si{j}] = 0$.
Therefore, the self-reverse and anti-self-reverse parts are
\begin{align}
	\srev{\rotor R} &= \Co1\Co2 + \frac12\qty{\Si1, \Si2}
&	&\text{and}
&	\arev{\rotor R} &= \Si1\Co2 + \Co1\Si2 + \frac12\qty[\Si1, \Si2]
	\label{eqn:arev-and-srev-parts}
.\end{align}
Hence, from \cref{eqn:log-rotor} we obtain an explicit \x{BCH} formula.

\begin{theorem}[rotor \x{BCH} formula]
	\label{thm:geometric-BCH}
	If $σ_1, σ_2 ∈ \GA[2](p, q)$ are bivectors in $p + q ≤ 4$ dimensions, then $e^{σ_1}e^{σ_2} = ±e^{\bch{σ_1}{σ_2}}$ where
	\begin{align}
		\label{eqn:geometric-BCH}
		\bch{σ_1}{σ_2}
		% \coloneqq \log(e^{σ_1}e^{σ_2})
		= \arctanh\qty(\frac{
			\Ta1 + \Ta2 + \frac12\qty[\Ta1, \Ta2]
		}{
			1 + \frac12\qty{\Ta1, \Ta2}
		})
	\end{align}
	where we abbreviate $\Ta{i} \coloneqq \tanh σ_i$.
	Note that this satisfies the rotor equation with an overall ambiguity in sign.
\end{theorem}

We may wish to express \cref{eqn:geometric-BCH} in terms of geometrically significant products instead of \paren{anti}commutators.
A bivector product is generally a $\qty{0,2,4}$\hyp multivector
\begin{align}
	ab
	&= \grade[0]{ab} + \grade[2]{ab} + \grade[4]{ab}
\\	&= a \fatdot b + a × b + a ∧ b
	\label{eqn:bivector-products}
.\end{align}
where $a×b = \grade[2]{ab} = \frac12[a, b]$ is the commutator product.
We may then write \cref{eqn:geometric-BCH} so that the grade of each term is explicit:
\begin{align}
	\bch{σ_1}{σ_2} = \arctanh\qty(\frac{
		\Ta1 + \Ta2 + \Ta1 × \Ta2
	}{
		1 + \Ta1 \fatdot \Ta2 + \Ta1 ∧ \Ta2 
	})
	\label{eqn:geometric-BCH-products}
\end{align}
The numerator is a bivector, while the denominator contains scalar ($\Ta1·\Ta2$) and $4$-vector ($\Ta1\wedge\Ta2$) terms.







\subsection{Zassenhaus-type formulae}

It is interesting to generalise the \x{BCH} formula \eqref{eqn:general-BCH-formula} to three rotors
\begin{math}
	e^{σ_1}e^{σ_2}e^{σ_3} = e^σ
\end{math}
in an algebra $\GA(p, q)$ with $p + q ≤ 4$.
A solution to this rotor equation is
\begin{align}
	σ = \log(±e^σ) = \arctanh\qty(\frac{\arev{e^{σ_1}e^{σ_2}e^{σ_3}}}{\srev{e^{σ_1}e^{σ_2}e^{σ_3}}})
,\end{align}
by \cref{eqn:log-rotor}.


We will find it convenient to define the \textdef{anticommutator product} $A\wedot B ≔ \frac12\qty{A, B}$ to complement the commutator product $A×B$.
The symbol ``$\wedot$'' is motivated by the fact that, for bivectors, we have
\begin{math}
	σ\wedot ρ = σ\cdot ρ + σ∧ρ
\end{math}
and thus
\begin{align}
	\label{eqn:(anti)com-prods}
	σ\wedot ρ &≔ \frac12(σρ + ρσ) = \srev{σρ}
,&	σ×ρ &≔ \frac12(σρ - ρσ) = \arev{σρ}
.\end{align}





Because $e^{σ_1}e^{σ_2}e^{σ_3} ∈ \GA[+](p, q)$ is an even multivector, the anti-self-reverse projection is exactly the bivector part,
\sidenotetext{
	Recall $\rev{A} = \revsign{k}A$ for a $k$-vector $A$ where $(\etc{\revsign{\i}}{}4) = ({+}\,{-}\,{-}\,{+})$.
}
\begin{math}
	\arev{e^{σ_1}e^{σ_2}e^{σ_3}} = \grade[2]{e^{σ_1}e^{σ_2}e^{σ_3}}
,\end{math} and the self-reverse projection is the $\set{0,4}$\hyp multivector part.\sidenotemark\
Decomposing $e^{σ_i} = \Co i + \Si i$, we find $2^3$ terms which separate into%
\begin{fullwidth}
	\begin{align}
		\arev{e^{σ_1}e^{σ_2}e^{σ_3}} &=
		% \grade[2]{e^{σ_1}e^{σ_2}e^{σ_3}} =
		\Si1\Co2\Co3 +
		\Co1\Si2\Co3 +
		\Co1\Co2\Si3 +
		(\Co1\Si2 + \Si1\Co2)×\Si3 +
		(\Si1×\Si2)\Co3 +
		\arev{\Si1\Si2\Si3} 
	,\\	\srev{e^{σ_1}e^{σ_2}e^{σ_3}} &=
		% \grade[0,4]{e^{σ_1}e^{σ_2}e^{σ_3}} =
		\Co1\Co2\Co3 +
		(\Co1\Si2 + \Si1\Co2) \wedot \Si3 +
		(\Si1\wedot\Si2) \Co3 +
		\srev{\Si1\Si2\Si3}
	.\end{align}
\end{fullwidth}
The $\set{0,4}$\hyp multivectors $\Co i$ commute with the bivectors $\Si i$, and products of $\Co i$ and $\Si j$ are themselves bivectors.
Therefore, terms containing one $\Si i$ factor are bivectors, and terms containing two $\Si i$ factors, such as $\Si1\Si2\Co3$, are products of bivectors, or $\set{0, 2, 4}$\hyp multivectors.
These terms are split into bivectors $(\Si1×\Si2)\Co3$ and $\set{0,4}$\hyp multivectors $(\Si1\wedot\Si2)\Co3$.


Cancelling factors of $\Co1\Co2\Co3$, we then have
\begin{align}
	\label{eqn:ternary-bch.1}
	\frac{\arev{e^{σ_1}e^{σ_2}e^{σ_3}}}{\srev{e^{σ_1}e^{σ_2}e^{σ_3}}} =
	\frac{
		\Ta1 + \Ta2 + \Ta3 + (\Ta1 + \Ta2)×\Ta3 + \Ta1 × \Ta2 + \arev{\Ta1\Ta2\Ta3}
	}{
		1 + (\Ta1 + \Ta2)\wedot \Ta3 + \Ta1 \wedot \Ta2 + \srev{\Ta1\Ta2\Ta3}
	}
\end{align}
where $\Ta i ≔ \tanh σ_i$.
This fraction is well-defined since the $\set{0, 4}$\hyp multivector denominator commutes with the numerator.

The next lemma is used to rewrite the rightmost terms with \paren{anti}commutator products \eqref{eqn:(anti)com-prods}.
\begin{lemma}
	For any bivectors $σ,ρ,ω ∈ \GA[2](p, q)$ where $p + q ≤ 4$,
	\begin{align}
		\arev{σρω} &= (σ\wedotρ)\wedotω + (σ×ρ)×ω
	,&	\srev{σρω} &= (σ×ρ)\wedotω
	.\end{align}
\end{lemma}
\begin{proof}
	Observe that $\arev{σρω} = \grade[2]{σρω}$ since $σρω$ is a $\set{0, 2, 4}$\hyp multivector, of which only the bivector part is anti-self-reverse.
	Using associativity and linearity,
	\begin{align}
		\grade[2]{σρω} = \grade[2]{(σ\wedotρ)ω} + \grade[2]{(σ×ρ)ω} = (σ\wedotρ)ω + (σ×ρ)×ω
	.\end{align}
	The product $(σ\wedotρ)ω = (σ\wedot ρ)\wedot ω$ is between a $\set{0,4}$\hyp multivector and a bivector, which may only contain bivector components.
	The product $(σ×ρ)ω$ is between two bivectors, having bivector part $(σ×ρ)×ω$.

	Similarly, note that
	\begin{align}
		\srev{σρω} = \grade[0,4]{(σ\wedotρ)ω} + \srev{(σ×ρ)ω} = (σ×ρ)\wedotω
	,\end{align}
	where the first term vanishes since $(σ\wedotρ)ω$ is a bivector.
\end{proof}

This allows us to collect the terms in \cref{eqn:ternary-bch.1} as
{\begin{align}
	\frac{\arev{e^{σ_1}e^{σ_2}e^{σ_3}}}{\srev{e^{σ_1}e^{σ_2}e^{σ_3}}}
	% &= \frac{
	% 	(\Ta1 + \Ta2 + \Ta1 × \Ta2) + \Ta3 + (\Ta1 + \Ta2 + \Ta1 × \Ta2)×\Ta3 + (\Ta1\wedot\Ta2)\wedot\Ta3
	% }{
	% 	1 + (\Ta1 + \Ta2 + \Ta1×\Ta2)\wedot \Ta3 + \Ta1 \wedot \Ta2
	% }
	&= \frac{
		\Ta{12} + \Ta3 + \Ta{12}×\Ta3 + (\Ta1\wedot\Ta2)\wedot\Ta3
	}{
		1 + \Ta{12}\wedot \Ta3 + \Ta1 \wedot \Ta2
	}
\end{align}}
where $\Ta{12} ≔ \Ta1 + \Ta2 + \Ta1×\Ta2$.
This leads us to the following result.
\begin{lemma}
	For bivectors $σ_i ∈ \GA[2](p, q)$ with $p + q ≤ 4$,
	\begin{align}
		e^{σ_1 + σ_2} &= e^{σ_1}e^{σ_2}e^ρ
	\end{align}
	where
	\begin{align}
		ρ &= \arctanh\qty(\frac{
			F - R - R×F + S\wedot F
		}{
			1 - R\wedot F + S
		})
	,\\	F &= \tanh(σ_1 + σ_2)
	,\\	R &= \tanh(σ_1)×\tanh(σ_2) + \tanh(σ_1) + \tanh(σ_2)
	,\\	S &= \tanh(σ_1)\wedot\tanh(σ_2)
	.\end{align}
\end{lemma}

\todo{First order corrections? Don’t know where to go with this.}









\subsection{In low dimensions: Rodrigues' rotation formula}

It is illustrative to see how the \x{BCH} formula \eqref{eqn:geometric-BCH} reduces in low-dimensional special cases.
Indeed, in two dimensions, all bivectors are scalar multiples of $\vol = \ve_1\ve_2$, and we recover the trivial case $e^ae^b = e^{a+b}$. %$\bch{a}{b} = a + b$.
Specifically, in the Euclidean $\GA(2)$ plane (or anti-Euclidean $\GA(0,2)$ plane) we have $\vol^2 = -1$, and \cref{eqn:geometric-BCH} simplifies by way of the tangent angle addition identity
\begin{align}
	\arctan(\frac{\tan θ_1 + \tan θ_1}{1 - \tan θ_1 \tan θ_2}) = θ_1 + θ_2
.\end{align}
This identity encodes how angles add when given as the gradients of lines; $m = \tan θ$.

Similarly, in the hyperbolic plane $\GA(1,1)$ with basis $\set{\ve_+, \ve_-}, \ve_±^2 = ±1$, the pseudoscalar $\vol = \ve_+\ve_-$ generates \emph{hyperbolic} rotations $e^{\vol ξ} = \cosh ξ + \vol\sinh ξ$ owing to the fact that $\vol^2 = -\ve_+^2\ve_-^2 = +1$.
Then, \cref{eqn:geometric-BCH} simplifies by the hyperbolic angle addition identity
\begin{align}
	\arctanh\qty(\frac{\tanh ξ_1 + \tanh ξ_1}{1 + \tanh ξ_1 \tanh ξ_2}) = ξ_1 + ξ_2
\end{align}
which encodes how collinear rapidities add when given as relativistic velocities; $β = \tanh ξ$.

Less trivially, a rotation in $\RR^3$ by $θ$ may be represented by its \textdef{Rodrigues vector}\sidenote{
	Olinde Rodrigues originated the formula in 1840 \cite[pp. 406]{rodrigues1840rotations}.
} $\vb r = \vb{\hat{r}}\tan\fracθ2$ pointing along the axis of rotation.
The composition of two rotations is then succinctly encoded in Rodrigues' composition formula
\begin{align}
	\vb r_{12} = \frac{\vb r_1 + \vb r_2 - \vb r_1 × \vb r_2}{1 - \vb r_1 · \vb r_2}
	\label{eqn:rodrigues-formula}
\end{align}
involving the standard vector dot and cross products.

We can easily derive \cref{eqn:rodrigues-formula} as a special case of \cref{eqn:geometric-BCH-products} as follows:
Let $σ_1, σ_2 ∈ \GA[2](3)$ be two bivectors defining the rotors $e^{σ_1}$ and $e^{σ_2}$ in three dimensions.
In $\GA(3)$, the only $4$-vector is trivial, so $σ_1 ∧ σ_2 = 0$ and for the composite rotor $e^{σ_3} \coloneqq e^{σ_1}e^{σ_2}$ we have
\begin{align}
	σ_3 = \bch{σ_1}{σ_2} = \arctanh\qty(\frac{
	\tanh σ_1 + \tanh σ_2 + \tanh σ_1 \times \tanh σ_2
	}{
	1 + \tanh σ_1 · \tanh σ_2
	})
\end{align}
where $a × b$ is the commutator product of bivectors as in \cref{eqn:bivector-products}, not the vector cross product.
Observe that Euclidean bivectors $σ_i ∈ \GA[2](3)$ have negative square (e.g., $(\ve_1\ve_2)^2 = -\ve_1^2\ve_2^2 = -1$) and relate to their dual normal vectors by $\vb{u}_i$ by $σ_i = \vb{u}_i\vol$.
Therefore, by rewriting
\begin{math}
	\tanh σ_i
	= \tanh (\vb{u}_i\vol)
	= (\tan \vb{u}_i)\vol
,\end{math}
we obtain the formula in terms of plain vectors and the vector cross product.
\begin{align}
	\vb{u}_{12} = (\bch{\vb{u}_1\vol}{\vb{u}_2\vol})\vol^{-1}
	= \arctan\qty(\frac{
	\tan \vb{u}_1 + \tan \vb{u}_2 - \tan \vb{u}_1 \times \tan \vb{u}_2
	}{
	1 - \tan \vb{u}_1 · \tan \vb{u}_2
	})
\end{align}
Indeed, a bivector $σ_i = \vb{u}_i\vol$ generates an $\RR^3$ rotation through an angle $θ = 2\|\vb{u}_i\|$ via the double-sided transformation law
\begin{math}
	a \mapsto e^{\vb{u}\vol}ae^{-\vb{u}\vol}
.\end{math}
Hence, $\tan \vb{u}_i = \vb{\hat{v}}_i\tan\fracθ2 ≡ \vb{r}_i$ are exactly the half-angle Rodrigues vectors, and we recover \cref{eqn:rodrigues-formula}.

The necessity of the half-angle in the Rodrigues vectors reflects the fact that they actually generate \emph{rotors}, not direct rotations, and hence belong to the underlying spin representation of $\SO^+(3)$ --- a fact made clearer in the context of geometric algebra.









\subsection{In higher dimensions}

In fewer than four dimensions, the $4$-vector $\Ta1\wedge\Ta2 = 0$ appearing in the geometric \x{BCH} formula is trivial, and so \cref{eqn:geometric-BCH} involves only bivector addition and scalar multiplication.
In four dimensions, there is one linearly independent $4$-vector --- the pseudoscalar --- which necessarily commutes with all even multivectors.
However, in more than four dimensions, $4$-vectors do \emph{not} necessarily commute with bivectors, and the assumptions underlying \cref{eqn:arev-and-srev-parts} and hence the main result \eqref{eqn:geometric-BCH} fail.


On the face of it, the \x{BCH} formula \eqref{eqn:geometric-BCH} in the four-dimensional case appears deceptively simple --- it hides complexity in the calculation of the trigonometric functions of arbitrary bivectors,
\begin{fullwidth}
	\begin{align}
		\tanh σ_i &= σ - \frac13σ^3 + \frac{2}{15}σ^5 + \cdots
	&	&\text{and}
	&	\arctanh σ_i &= σ +\frac13σ^3 + \frac15σ^5 + \cdots
		\label{eqn:trig-power-series}
	.\end{align}
\end{fullwidth}

In fewer dimensions, $σ^2$ is a scalar, and so these power series are as easy to compute as their real equivalents.\sidenote{
	If $σ^2 = N_σ^2 ∈ \RR$, then we have simply $\tanh σ = (\tanh N_σ)N_σ^{-1}σ$.
}
But in four dimensions, $σ^2$ is in general a $\qty{0,4}$\hyp multivector (by lemma~\ref{lem:grades-of-square}) and the power series \eqref{eqn:trig-power-series} are more complicated.
However, if $σ^2 \ne 0$ has a square root $N_σ = α + β\vol$ in the scalar--pseudoscalar plane, then one has $σ = N_σ\hat{σ} = \hat{σ}N_σ$ where $\hat{σ} \coloneqq σ/N_σ$ so that $\hat{σ}^2 = 1$.
With a bivector $σ = N_σ\hat{σ}$ expressed in this form, the valuation of a formal power series $f(z) = \sum_{n=1}^\infty f_n z^n$ simplifies to
\begin{align}
\label{eqn:normalised-power-series}
	\text{($f$ even)}&
&	f(σ) &= \sum_{n = 1}^\infty f_{2n} σ^{2n}
	= \sum_{n = 1}^\infty f_{2n} N_σ^{2n}
	= f(N_σ)
,\\	\text{($f$ odd)}&
&	f(σ) &= \sum_{n = 1}^\infty f_{2n + 1} σ^{2n + 1}
	= \sum_{n = 1}^\infty f_{2n} N_σ^{2n + 1} \hat{σ}
	= f(N_σ)\hat{σ}
.\end{align}
This is especially useful in the case of Minkowski spacetime $\GA(1,3)$ because the scalar--pseudoscalar plane is isomorphic to $\CC$ and square roots always exist (see \cref{sec:invariant-bivector-decomposition}).
From now on, we focus on the special case of Minkowski spacetime, and consider practical and theoretical applications.








\section{\x{BCH} Composition in Spacetime}

Because the geometric \x{BCH} formula is constructed from sums and products of bivectors, it involves only even spacetime multivectors.
Therefore, in numerical applications, it is not necessary to represent the full STA, but only the even subalgebra $\GA[+](1,3) \cong \GA(3)$.

The algebra of physical space $\GA(3)$ admits a faithful complex linear representation by the Pauli spin matrices (see \cref{sec:common-alg-isos}).
The real dimension of both $\CC^{2×2}$ and $\GA(3)$ is eight, so there is no redundancy in the Pauli representation, making it suitable for computer implementations.



An even $\GA[+](1,3)$ multivector --- or equivalently, a general $\GA(3)$ multivector --- may be parametrised by four complex scalars $q^μ = \Re(q^μ) + i\Im(q^μ) ∈ \CC$ as
\begin{align}
	A = \Re(q^0) + \Re(q^i)\vs_i + \Im(q^i)\vol\vs_i + \Im(q^0)\vol
,\end{align}
where the $\vs_i$ may be read both as spacetime bivectors $\vs_i \equiv \vg_0\vg_i ∈ \GA[+](1,3)$ or as basis vectors of $\GA(3)$ under a space\slash time split.
The Pauli matrices $σ_i ∈ \CC^{2×2}$ form a linear representation of $\GA(3)$ by the association $\vs_i \equiv σ_i$.
Explicitly, identifying
\begin{align}
	\vs_1 &\equiv \mqty[
		 0&+1\\
		+1& 0\\
	]
&	\vs_2 &\equiv \mqty[
		 0&-i\\
		+i& 0\\
	]
&	\vs_3 &\equiv \mqty[
		+1& 0\\
		 0&-1\\
	]
\end{align}
along with $1 \equiv I$ and $\vol \equiv iI$ where $I$ is the $2×2$ identity matrix, we obtain a representation of the multivector $A$ by a $2 × 2$ complex matrix:
\begin{align}
	\lin A \equiv \mqty[
		q^0 + q^3 & q^1 - iq^2 \\
		q^1 + iq^2 & q^0 - q^3
	]
	\label{eqn:multivector-as-matrix}
.\end{align}






A proper Lorentz transformation $\lin Λ ∈ \SO^+(1,3)$ is determined in the $K$ frame by a vector rapidity $\vb ξ ∈ \RR^3$ and axis--angle vector $\vb θ ∈ \RR^3$.
The standard $4×4$ matrix representation of $\lin Λ$ is then obtained as the exponential of the generator
\begin{align}
	\mqty[
		 0 &  \vb ξ^T \\
		\vb ξ & ε_{ijk}θ^k
	] =
	\qty[\begin{array}{llll}
	0 & \phantom{±}ξ^1 & \phantom{±}ξ^2 & \phantom{±}ξ^3 \\
	ξ^1 & \phantom{±}0 & +θ^3 & -θ^2 \\
	ξ^2 & -θ^3 & \phantom{±}0 & +θ^1 \\
	ξ^3 & +θ^2 & -θ^1 & \phantom{±}0
	\end{array}]
	\in \liealg{so}(1,3)
	\label{eqn:lorentz-generator-matrix}
.\end{align}
In the spin representation, the transformation $\lin Λ$ corresponds to a rotor $\rotor L = e^σ$, and the generating bivector \eqref{eqn:bivector-generator} may be expressed via \cref{eqn:multivector-as-matrix} as the traceless complex matrix
\begin{align}
	\lin Σ = q^kσ_k = \mqty[
		+q^3 & q^1 - iq^2 \\
		q^1 + iq^2 & -q^3
	]
	\label{eqn:bivector-as-matrix}
,\end{align}
where $q^k \coloneqq \frac12(ξ^k + iθ^k) ∈ \CC$.
% This represents the restriction of \eqref{eqn:multivector-as-matrix} to spacetime bivectors.
Note that, since the square of a spacetime bivector is a $\qty{0,4}$\hyp multivector, its representative matrix $\lin Σ$ squares to a complex scalar multiple of the identity matrix.



Given two generators $σ_i$ with matrix representations $\lin Σ_i$, the geometric \x{BCH} formula \eqref{eqn:geometric-BCH} reads
\begin{align}
	\lin Σ_3 ≔ \bch{\lin Σ_1}{\lin Σ_2} = \tanh^{-1}\qty(
		\lin{\frac{ T_1 + T_2 + A }{ I + S }}
	)
	\label{eqn:matrix-BCH}
,\end{align}
where $\lin A ≔ \frac12[\lin T_1, \lin T_2]$, $\lin S ≔ \frac12\qty{\lin T_1, \lin T_2}$ and $\lin T_i ≔ \tanh \lin Σ_i$.

To efficiently compute $\lin T_i$, make use of the fact that $\lin Σ_i^2 = λ_i^2\lin I$ is a complex multiple of the identity matrix and evaluate $\lin T_i = (\tanh λ_i)λ_i^{-1}\lin Σ_i$.
In the null case $\lin Σ_i^2 = λ = 0$, the power series \eqref{eqn:trig-power-series} truncate and $\tanh \lin Σ_i = \tanh^{-1} \lin Σ_i = \lin Σ_i$ are equal.
The commutator and anti-commutator terms $\lin A$ and $\lin S$ may be efficiently computed by separating the single matrix product $\lin{ Π ≔ T_1T_2 = A + S }$ into off-diagonal and diagonal components, respectively; i.e.,
\begin{align}
	\lin A_{ij} = (1 - δ_{ij})\lin Π_{ij}
	\qqtext{and}
	\lin S_{ij} = δ_{ij}\lin Π_{ij}
.\end{align}
The numerator of \cref{eqn:matrix-BCH} is therefore a matrix with zeros on the diagonal, and the denominator is a complex scalar multiple of the identity, so the argument of $\tanh^{-1}$, call it $\lin M$, is in the form \eqref{eqn:bivector-as-matrix}.
Computing $\tanh^{-1} \lin M$ again simply amounts to
\begin{math}
	\lin Σ_3 = \tanh^{-1} \lin M = (\tanh^{-1} λ)λ^{-1}\lin M
\end{math}
where $\lin M^2 = λ^2\lin I$.

The Lorentz generator in the standard vector representation \eqref{eqn:lorentz-generator-matrix} can then be recovered from $\lin Σ_3$ with the relations $ξ^k = 2\Re(q^k)$ and $θ^k = 2\Im(q^k)$, and the final $\SO^+(1,3)$ vector transformation is its $4×4$ matrix exponential.









\subsection{Relativistic 3-velocities and the Wigner angle}

As an example of its theoretical utility, we shall use the geometric \x{BCH} formula \eqref{eqn:geometric-BCH} to derive the composition law for arbitrary relativistic $3$-velocities.

The innocuous problem of composing relativistic velocities has been called ``paradoxical'' \cite{ungar1989sr-velocity-composition,mocanu1992sr-velocity-composition,visser2011sr-velocity-composition}, owing in part to the fact that \emph{irrotational} boosts are not closed under composition, and that explicit matrix analysis becomes cumbersome.
Of course, in reality there is no paradox, and the full description of the composition of boosts is pedagogically valuable as it highlights aspects of special relativity which differ from spatial intuition.


We may speak of a rotation or boost as being \textdef{pure} relative to the $K$ frame.
Technically, $σ$ generates a pure rotation (or pure boost) if, under the space\slash time split relative to the $K$ frame, $σ = \grade[2]{σ}$ is a pure bivector (or a pure vector) in $\GA(3)$.
A pure rotation or pure boost relative to $K$ is \emph{not} pure in all other frames.

The restriction of the \x{BCH} formula to pure boosts is not as simple as the restriction to rotations \eqref{eqn:rodrigues-formula}, because pure boosts do not form a closed subgroup of $\SO^+(1,3)$ as pure rotations do.
Instead, the composition of two pure boosts $\rotor B_i$ is a pure boost composed with a pure rotation (or vice versa),
\begin{align}
	\rotor B_1\rotor B_2 = \rotor B\rotor R
	\label{eqn:boost-boost-makes-boost-rotation}
.\end{align}
The direction of the boost $\rotor B$ lies within the plane defined by the boost directions of $\rotor B_1$ and $\rotor B_2$, and $\rotor R$ is a rotation through this plane by the \emph{Wigner angle} \cite{visser2011sr-velocity-composition}.
Applying \cref{eqn:geometric-BCH} to this case immediately yields formulae for the resulting boost and rotation.\sidenote{
	These results are equivalent to those in \cite{berry2020quat-sr} which are formulated using complexified quaternions.
}



For ease of algebra, we conduct the following analysis under a space\slash time split with respect to the $K$ frame.
Under this split, a pure boost $\rotor B$ is generated by an $\RR^3$ vector $\frac{\vb ξ}{2}$, and a pure rotation $\rotor R$ is generated by an $\RR^3$ bivector $\fracθ2\hat{r}$.
Here, $\vb ξ ∈ \GA[1](3)$ is the \emph{vector rapidity}, related to the velocity by $\vb v/c = \vb β = \tanh \vb ξ$, and the rotation is through an angle $θ$ in the plane spanned by the bivector $\hat{r} ∈ \GA[2](3)$.
\Cref{eqn:geometric-BCH} with two pure boosts $\vb ξ_1$ and $\vb ξ_2$ is
\begin{align}
	\tanh(\bch{\frac{\vb ξ_1}{2}}{\frac{\vb ξ_2}{2}})
	= \frac{𝒘_1 + 𝒘_2 + 𝒘_1 ∧ 𝒘_2}{1 + 𝒘_1⋅𝒘_2}
	\label{eqn:boost-boost}
\end{align}
where $𝒘_i ≔ \tanh\frac{\vb ξ_i}{2}$ are the \emph{relativistic half-velocities}, also defined in \cite{berry2020quat-sr,berry2021quat-sr}.
The generator \eqref{eqn:boost-boost} has vector and bivector (namely $𝒘_1 ∧ 𝒘_2$) parts, indicating that the Lorentz transformation it describes is indeed some combination of a boost and a rotation.


Similarly, for an arbitrary pure boost and pure rotation,
\begin{align}
	\tanh(\bch{\frac{\vb ξ}2}{\fracθ2\hat{r}})
	= \frac{𝒘 + ρ + \frac12[𝒘, ρ]}{1 + 𝒘\wedge ρ}
	\label{eqn:boost-rotation}
\end{align}
where $ρ ≔ \tanh \frac{θ\hat{r}}2 = \hat{r}\tan\frac θ2$ is a bivector.
In general, \cref{eqn:boost-rotation} has vector, bivector \emph{and} pseudoscalar parts (the commutator $\frac12[𝒘, ρ] = \grade[1]{𝒘ρ} + 𝒘 \wedge ρ$ and the denominator both have grade-three part $𝒘\wedge ρ$).
However, \cref{eqn:boost-boost,eqn:boost-rotation} are equal by supposition of \cref{eqn:boost-boost-makes-boost-rotation}.
By comparing parts of equal grade, we deduce the pseudoscalar part of \cref{eqn:boost-rotation} is zero.
This requires $𝒘\wedge ρ = 0$ or, equivalently, that $𝒘$ lies in the plane defined by $ρ$ --- meaning the resulting boost is coplanar with the Wigner rotation as expected.
Hence, for a coplanar boost and rotation, \cref{eqn:boost-rotation} is simply
\begin{align}
	\tanh(\bch{\frac{\vb ξ}2}{\fracθ2\hat{r}})
	= 𝒘 + ρ + 𝒘ρ
	\label{eqn:boost-rotation-coplanar}
.\end{align}
The term $𝒘ρ = \grade[1]{𝒘ρ} = -ρ𝒘$ is a vector orthogonal to $𝒘$ in the plane defined by $ρ$.




Equating the bivector parts of \cref{eqn:boost-boost,eqn:boost-rotation-coplanar} determines the rotation
\begin{align}
	ρ &= \frac{𝒘_1 \wedge 𝒘_2}{1 + 𝒘_1 · 𝒘_2}
,&	&\text{implying}
&	θ = 2\tan^{-1}\qty(\frac{w_1w_2\sinϕ}{1 + w_1w_2\cosϕ})
\end{align}
where $ϕ$ is the angle between the two initial boosts (in the $K$ frame).
The angle $θ$ is precisely the Wigner angle.
Equating the vector parts determines the boost
\begin{align}
	\marginnote{
	Note that $1 + 𝒘_1⋅𝒘_2 ∈ \RR$ commutes and may be written as a denominator, while $1 + ρ$ cannot.
	}
	𝒘 &= \frac{𝒘_1 + 𝒘_2}{1 + 𝒘_1⋅𝒘_2}(1 + ρ)^{-1}
,\end{align}
noting that $𝒘_i$ and $ρ$ do not commute.
Substituting $ρ$ leads to the remarkably succinct composition law
\begin{math}
	𝒘 = (𝒘_1 + 𝒘_2)(1 + 𝒘_1𝒘_2)^{-1}
\end{math}
exhibited in \cite{berry2020quat-sr}, with the final relativistic velocity being $\vb β = \tanh \vb ξ = \tanh (2\tanh^{-1} 𝒘)$.
