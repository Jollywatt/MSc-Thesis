\chapter{The Algebra of Spacetime}
\label{cha:sta}


Special relativity is geometry with a Lorentzian signature.
The \textdef{spacetime algebra} (STA) is the name given to the geometric algebra of a Minkowski vector space, $\GA(1,3) ≡ \GA(\RR^4, η)$, where $η = \op{diag}(+---{})$.
Other introductory material on the STA can be found in \cite{hestenes2003sta,gull1993sta,dressel2015sta}.


We denote the standard vector basis by $\qty{\vb γ_μ}$, where Greek indices run over $\qty{0,1,2,3}$.
This is a deliberate allusion to the Dirac $γ$-matrices, whose algebra is isomorphic to the STA --- however, the $\vg_μ ∈ \RR^{1+3}$ of STA are real, genuine spacetime vectors.
A basis for the entire $2^4$-dimensional STA is then
\begin{fullwidth}
	\newcommand{\below}[2]{\underset{#1}{\vphantom{\int}#2}}
	\begin{align}
		\overset{\text{1 scalar}}{
			\set[\big]{ \below{+}{\vb 1} }
		}
	∪	\overset{\text{4 vectors}}{
			\set[\big]{ \below{+}{\vg_0}, \; \below{-}{\vg_i}}
		}
	∪	\overset{\text{6 bivectors}}{
			\set[\big]{ \below{+}{\vg_0\vg_i}, \; \below{-}{\vg_j\vg_k}}
		}
	∪	\overset{\substack{\text{4 trivectors}}}{
			\set[\big]{ \below{-}{\vg_0\vg_j\vg_k}, \; \below{+}{\vg_1\vg_2\vg_3}}
		}
	∪	\overset{\substack{\text{1 pseudoscalar}}}{
			\set[\big]{ \vol ≔ \below{-}{\vg_0\vg_1\vg_2\vg_3}}
		}
	\end{align}
\end{fullwidth}
where lowercase Latin indices range over spacelike components, $\set{1,2,3}$.
Blades shown on the left-hand side of $\set{\quad,\quad}$ are called \textdef{timelike}, and those in on right-hand side \textdef{spacelike}.
The sign below each basis blade shows its signature (the sign of its scalar square).
Multivectors of any kind which square to zero are called \textdef{null}.


\subsubsection{The pseudoscalar and duality}

The right-handed unit pseudoscalar $\vol$ represents an oriented unit $4$-volume.
It anticommutes with odd elements of the STA (vectors and trivectors) and commutes with even elements (bivectors and \paren{pseudo}scalars).

Since $\vol^2 = -1$, the scalar--pseudoscalar plane $\GA[0,4](1,3) = \op{span}_\RR\set{1, \vol}$ is isomorphic to the complex plane $\CC$.
Thus, for the sake of computation, operations on $\set{0, 4}$-multivectors may be regarded as operations on complex numbers.
In particular, we define the principal root $\sqrt{a}$ of a $\set{0,4}$-multivector $a ∈ \GA[0,4](1,3)$ in the same way as it is defined in $\CC$ with a branch cut at $θ = π$.
It is worth emphasising that there are many square roots of $-1$ in the spacetime algebra, each with distinct geometrical meanings.\sidenote{
	E.g., the spacelike bivector $(\vg_i\vg_j)^2 = -1$ represents a directed spacelike plane; the timelike vector $\vg_0^2 = -1$ a particle's velocity.
}
We single to single out $\sqrt{-1} = \vol$ as `the' principal root as this proves to be useful notationally.\sidenote{
	In electromagnetism, the imaginary unit $i$ often represents the volume element $\vol$.
	E.g., in the Riemann--Silberstein vector \cite{silberstein}, both $i$ and $\vol$ play roles similar to the Hodge dual \cite{dressel2015sta}.
}

The volume element ... \todo{is the Hodge dual}


\section{The space/time split}
\label{sec:spacetime-split}

While we actually live in $\RR^{1,3}$ spacetime, to any particular observer it appears that space is $\RR^3$ with a separate scalar time parameter.
This is reflected in the fact that $\GA[+](1,3)$ and $\GA(3)$ are isomorphic, from \cref{lem:even-subalg-isos}.
In fact, there is a separate isomorphism associated to each timelike direction, corresponding to each inertial observer's personal spacetime split.
Such a \textdef{space/time split} identifies \emph{even} multivectors in the spacetime algebra $\GA[+](1,3)$ with $\GA(3)$ multivectors, providing an efficient, purely algebraic method for switching between inertial frames \cite{hestenes2003sta}.


Let $K$ be an inertial observer and for simplicity choose the standard basis $\qty{\vg_μ}$ so that $\vg_0$ is the instantaneous velocity of the $K$ frame.
The three \textdef{relative vectors} $\vs_i ≔ \vg_i\vg_0$ form a vector basis for $\GA(3)$, since the $\vg_i\vg_0$ indeed satisfy $\vs_i^2 = -\vg_i^2\vg_0^2 = 1$ and $\vs_i\vs_j = -\vs_j\vs_i$ for $i ≠ j$.
Because of the dependence on the frame's velocity vector $\vg_0$, the relative vectors $\vs_i$ are particular to the $K$ frame.
With respect to the $K$ frame, we may view $\GA(3) ⊂ \GA(1,3)$ as embedded in the STA, allowing us to consider multivectors as belonging to both spaces.
Note that the same volume element $\vol = \vs_1\vs_2\vs_3 = \vg_0\vg_1\vg_2\vg_3$ is shared by the algebras.

For example a spacetime bivector $F = F^{μν}\vg_μ\vg_ν$ may be separated into timelike $F^{i0}$ and spacelike $F^{ij}$ components with respect to the $K$ frame and viewed as a $\qty{1,2}$-multivector in $\GA(3)$,
\begin{align}
	\label{eqn:sta-bivector-split}
	F = F^{i0}\vg_i\vg_0 + F^{ij}\vg_i\vg_j
	= E^i\vs_i + B^i\vol\vs_i = \vb{E} + \vol \vb{B}
,\end{align}
where we use
\begin{math}
	\vg_i\vg_j
	= (\vg_i\vg_0)(\vg_j\vg_0)
	= -\vs_i\vs_j
	= -ε_{ijk}\vol\vs_k
.\end{math}
Note that the relativistic representation $F$ is \emph{equal} to the frame-dependent representation --- they are the same spacetime object.
\Cref{eqn:sta-bivector-split} performs the frame-dependent decomposition of a spacetime bivector (or ``2-form'') into two $\RR^3$ vectors familiar from electromagnetic theory.


Of particular interest are space/time splits on bivector generators associated to rotors.
A proper orthochronous Lorentz transformation $\lin Λ ∈ \SO^+(1,3)$ acts as a `sandwich' product $\lin Λ(A) = e^σ A e^{-σ}$, where the rotor $e^σ \in \op{Spin}^+(1,3)$ is generated by a spacetime bivector $σ ∈ \GA[2](1,3)$.
This bivector $σ$ can be represented in the $K$ frame as
\begin{align}
	σ = \frac12(ξ^i \vg_i + θ^i \vol \vg_i)\vg_0
	= \frac12(\vb ξ + \vol \vb θ)
	\label{eqn:bivector-generator}
\end{align}
where $\vb ξ = ξ^i\vs_i ∈ \GA[1](3)$ is a rapidity vector and $\vol \vb θ ∈ \GA[2](3)$ is a rotation bivector.




\section{The invariant bivector decomposition}
\label{sec:invariant-bivector-decomposition}

There is a clear analogy between the space/time split of a bivector \eqref{eqn:sta-bivector-split}, with its spacelike and timelike components, and the Cartesian form of a complex number, $x + iy$, with its real and imaginary parts.
This similarity can be made more precise: just as we may express complex numbers in polar form $re^{iϕ} = x + iy$, we may use the invariant bivector decomposition to write $ρe^{\vol σ} = \vb E + \vol \vb B$.


Non-null spacetime bivectors $σ \in \GA[2](1,3)$ may be \emph{normalised}, in the sense that there always exists some $N_σ \in \GA[0,4](1,3)$ such that
\begin{align}
	σ = N_σ\hat{σ} = \hat{σ}N_σ
	\qqtext{where}
	\hat{σ}^2 = 1
.\end{align}
In the null case $σ^2 = 0$, we let $\hat{σ}^2 = 0$ instead.
This is possible because the square of a bivector is a $\set{0, 4}$-multivector (\cref{lem:grades-of-square}), which always has a principle square root (since $\GA[0,4](1,3) ≅ \CC$).
Explicitly, let $σ^2 = α + \vol β = ρ^2e^{2\vol ϕ}$ for scalars $α, β, ρ, ϕ$, so that
\begin{align}
	N_σ
	\coloneqq \sqrt{σ^2}
	= ρe^{\vol ϕ}
	\label{eqn:spacetime-bivector-normaliser}
,\end{align}
assuming without loss of generality that $ρ > 0$ and $ϕ \in (-π/2, π/2]$.
Thus, the \textdef{invariant bivector decomposition}
\begin{align}
	\label{eqn:invariant-bivector-decomposition}
	σ = ρe^{\vol ϕ}\hat{σ}
	&= \underbrace{(ρ\cos ϕ)\hat{σ}}_{σ_+} + \underbrace{(ρ\sin ϕ)\vol\hat{σ}}_{σ_-}
\end{align}
separates $σ$ into commuting parts, $[σ_+, σ_-] = 0$, each of which satisfy $±σ_±^2 > 0$.
This makes it a useful device for algebraic manipulations.
Furthermore, the decomposition is unique, and does not depend on any particular space/time split.



The decomposition can be used to show the non-injectivity of the exponential map in the STA.
Take some bivector written in decomposed form,
\begin{math}
	σ = λ_+\hat σ + λ_-\vol\hat σ
.\end{math}
Each bivector in the family
\begin{align}
	σ_{n} = λ_+\hat σ + (λ_- + nπ)\vol\hat σ
	\label{eqn:equivalent-generators}
\end{align}
for $n ∈ \ZZ$ exponentiates to the same rotor, up to an overall sign: 
\begin{align}
	e^{σ_{n}} = e^{σ_{0}}e^{nπ\vol\hat σ} = (-1)^ne^{σ_{0}}
	\label{eqn:equivalent-rotors}
\end{align}
Note that $e^{\hat σ + \vol\hat σ} = e^{\hat σ}e^{\vol\hat σ}$ since $[\hat σ, \vol\hat σ] = 0$.
All the rotors in \cref{eqn:equivalent-rotors} correspond to the same $\SO^+(1,3)$ Lorentz transformation.
\Cref{eqn:equivalent-rotors} shows that every Lorentz rotor $±e^{σ_0}$ is equal to a pure bivector exponential $e^{σ_n}$ with a shifted rotational part $λ_- \mapsto λ_- + n\pi$.


\section{Lorentz Conjugacy Classes}

As shown above, every proper Lorentz transformation $\lin Λ ∈ \SO^+(1, 3)$ is generated by a bivector exponential $\lin Λ(𝒖) = e^σ 𝒖 e^{-σ}$.
This rotor formulation of the Lorentz group makes some of its more subtle properties clear, including its decomposition into five categories of \emph{conjugacy class}.
\begin{definition}
	The \textdef{conjugacy class} of a group element $g ∈ G$ is the set
	\begin{align}
		[g] ≔ \set{hgh^{-1} | h ∈ G} = \set{g' ∈ G | g' \sim g}
	\end{align}
	of elements conjugate\sidenote{
		Group elements $g \sim g'$ are conjugate iff there extists $h ∈ G$ such that $g = hg'h^{-1}$.
	} to $g$.
\end{definition}
Since conjugacy $\sim$ is an equivalence relation, the conjugacy classes partition the group $G$.

In the case of the proper Lorentz group, the set of conjugacy classes further partitions into five categories, or `kinds', according to basis-invariant properties of the constituent Lorentz transformations.
Using the STA, the `kind' of a Lorentz transformation (or its associated rotors) is given by simple properties of its generating bivector.\sidenote{
	One rotor has many generating bivectors, but any one will do.
}

\begin{definition}
	\label{def:lorentz-kinds}
	Let $σ ∈ \GA[2](1, 3)$ be a bivector.
	If $σ^2$ is a scalar, then $σ$ is called
	\begin{itemize}
		\item \textdef{trivial} if $σ = 0$; and if $σ ≠ 0$,
		\item \textdef{elliptic} if $σ^2 < 0$;
		\item \textdef{parabolic} if $σ^2 = 0$;
		\item \textdef{hyperbolic} if $σ^2 > 0$; and
		\item \textdef{loxodromic} if $σ^2 = α + \vol β$ is not a scalar but a $\set{0, 4}$-multivector.
	\end{itemize}
	% If $σ^2 = α + \vol β$ is not a scalar but a $\set{0, 4}$-multivector, then call $σ$ \textdef{loxodromic}.
\end{definition}

\begin{lemma}
	The square of a bivector is constant within each conjugacy class.
\end{lemma}
\begin{proof}
	Let $\lin Λ : 𝒖 ↦ e^σ 𝒖 e^{-σ}$ be a proper Lorentz transformation, and consider its conjugation with some other transformation $\lin Γ$,
	\begin{align}
		\lin{Γ Λ Γ}^{-1} : 𝒖 ↦ e^ρ e^σ e^{-ρ} 𝒖 e^{-ρ} e^{-σ} e^ρ
	.\end{align}
	Note that
	\begin{math}
		e^ρ e^σ e^{-ρ} =  e^{e^ρ σ e^{-ρ}}
	\end{math}
	by the automorphism property of rotor application.
	Therefore, conjugacy of $\lin Λ \sim \lin{Γ Λ Γ}^{-1}$ translates to bivectors as
	\begin{align}
		σ \sim σ' ≔ e^ρ σ e^{-ρ}
	\end{align}
	for some $ρ$.
	Hence, the conjugate bivectors have common square,
	\begin{align}
		σ'^2 = (e^ρ σ e^{-ρ})^2 = e^ρ σ^2 e^{-ρ} = σ^2
	\end{align}
	since $e^{±ρ}$ commutes with the $\set{0, 4}$-multivector $σ^2$.
\end{proof}

\begin{corollary}
	Conjugacy classes of $\SO^+(1, 3)$ fall into the five categories in \cref{def:lorentz-kinds} by considering the generating bivector of any representative Lorenz rotor.
\end{corollary}


Elliptical Lorentz transformations are \emph{rotations}, whose rotors are generated by spacelike $2$-blades; hyperbolic transformations are \emph{boosts}, with timelike $2$-blades generators.
Parabolic transformations are sometimes called \emph{null rotations}, and fall in between the previous two, with null $2$-blades as generators.
The final class of loxodromic transformations are a combination of a rotation and a boost where the axis of rotation is parallel with the direction of boost (in a particular frame).
A loxodromic generator is not s $2$-blade, but a bivector comprising mutually $2$-orthogonal\sidenote{
	in the sense of \cref{def:Δ-orthogonal}, \cref{sec:higher-orthogonal}
} $2$-blades, one timelike and one spacelike.

\todo{	Give examples of each kind, along with their matrix representations.}

\todo{Explain how classes are 1- or 2-parameter families}
% \begin{table}
% 	\centering
% 	\begin{tabular}
		
% 	\end{tabular}
% \end{table}