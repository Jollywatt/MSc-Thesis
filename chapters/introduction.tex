\chapter{Introduction}

The Special Theory of Relativity is a model of \emph{spacetime} --- the geometry in which physical events take place.
Spacetime comprises the Euclidean dimensions of space and time, but only in a way relative to each observer moving through it: there exists no single `universal' ruler or clock.
Instead, two observers in relative motion define different decompositions of spacetime, and their respective clocks and rulers are found to disagree according to the Lorentz transformation laws.
The insight of special relativity is that one should focus not on the observer-dependent notions of space and time, but on the Lorentzian geometry of spacetime itself.

Seven years after Albert Einstein introduced this theory,\sidenote{
	Einstein’s paper \cite{einstein1905electrodynamics} was published in 1905, the so-called \emph{Annus Mirabilis} or ``miracle year'' during which he also published on the photoelectric effect, Brownian motion and the mass-energy equivalence.
	Each of the four papers was a monumental contribution to modern physics.
} he succeeded in formulating a relativistic picture which included gravity.
In this General Theory of Relativity, gravitation is identified with the curvature of spacetime over astronomical distances.
Both theories coincide locally when confined to sufficiently small extents of spacetime, over which the effects of curvature are negligible.
In \cref{part:1}, we will focus on special relativity, leaving gravity and curvature to \cref{part:2}.

From the Erlangen programme,\sidenote{
	Introduced by Felix Klein in 1872 \cite{klein1893erlangen}, the Erlangen program characterises geometries (Euclidean, hyperbolic, projective, etc.) by their symmetry groups and invariants.
	E.g., Euclidean geometry studies the invariants of rigid transformations.
} the study of local spacetime geometry amounts to the study of its intrinsic symmetries.
These symmetries form the Poincaré group, and consist of spacetime translations and Lorentz transformations, the latter being the extension of the rotation group for Euclidean space to relativistic rotations of spacetime.
The standard matrix representation of the Lorentz group, $\SO^+(1, 3)$, is the connected component of the orthogonal group
\begin{align}
	\op{O}(1,3) = \set{\lin Λ ∈ \GL(\RR^4) | \lin Λ\trans\lin η\lin Λ = \lin η}
\end{align}
with respect to the bilinear form $η = ±\op{diag}(-1,+1,+1,+1)$.
The rudimentary tools of matrix algebra are sufficient for an analys the Lorentz group, and are familiar to any physicist.

However, the last century has seen many other mathematical tools be applied to the study of generalised rotation groups such as $\SO^+(1,3)$ or the rotation group $\SO(3)$ of $\RR^3$.
Among these tools is the \emph{geometric algebra}, invented\sidenote{
	Clifford algebra was independently discovered by Rudolf Lipschitz two years later \cite{lipschitz1880clifford-alg}. 
	He was the first to use them to the study the orthogonal groups.
} by William Clifford in 1878 \cite{clifford1878grassmann}.
Geometric algebra remains largely unknown in the physics community, despite arguably being far superior for the description of rotations than traditional matrix techniques.
It is good to glean some of the history that led to this (perhaps unfortunate) state of the field.

\subsubsection{The quest for an optimal formalism for rotations}

Mathematics has seen the invention of a variety of vector formalisms since the 1800s, and the question of which is best suited to physics has a long contentious history.

Complex numbers had been known for a long time\sidenote{
	Since Wessel, Argand and Gauss in the 1700s \cite{chappell2016quat-history}.
} to be useful descriptions of planar rotations.
William Hamilton's efforts to extend the same ideas into three dimensions by inventing a ``multiplication of triples'' bore fruition in 1843, when the quaternion algebra $\ii^2 = \jj^2 = \kk^2 = \ii\jj\kk = -1$ came to him in revelation.
In following decades, William Gibbs developed the vector calculus of $\RR^3$ with the usual vector cross and dot products.
The ensuing vector algebra ``war'' of 1890--1945 saw Hamilton's prized\sidenote{
	Hamilton had dedicated following in the time that quaternions were in fasion: the \emph{Quaternion Society} existed from 1895 to 1913.
} quaternion algebra $\HH$, hailed as the optimal tool for describing $3$d rotations, struggle for popularity against Gibbs' easier-to-visualise vector calculus.
Today, quaternions are generally regarded in physics as an old-fashioned mathematical curiosity.


