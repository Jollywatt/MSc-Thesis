\chapter{Introduction}

The Special Theory of Relativity is a model of \emph{spacetime} --- the geometry in which physical events take place.
Spacetime comprises the Euclidean dimensions of space and time, but only in a way relative to each observer moving through it: there exists no single `universal' ruler or clock.
Instead, two observers in relative motion find their respective clocks and rulers are found to disagree, according to the Lorentz transformation laws.
The insight of special relativity is that one should focus not on the observer-dependent notions of space and time, but on the Lorentzian geometry of spacetime itself.

Seven years after Albert Einstein introduced this theory,\sidenote{
	Einstein’s paper \cite{einstein1905electrodynamics} was published in 1905, the so-called \emph{Annus Mirabilis} or ``miracle year'' during which he also published on the photoelectric effect, Brownian motion and the mass-energy equivalence.
	Each of the four papers was a monumental contribution to modern physics.
} he succeeded in formulating a relativistic picture which included gravity.
In this General Theory of Relativity, gravitation is identified with the curvature of spacetime over astronomical distances.
Both theories coincide locally when confined to sufficiently small extents of spacetime, over which the effects of curvature are negligible.
In \cref{part:1}, we will focus on special relativity, leaving gravity and curvature to \cref{part:2}.

The study of local spacetime geometry amounts to the study of its intrinsic symmetries.\sidenote{
	This insight is part of Felix Klein's Erlangen programme of 1872 \cite{klein1893erlangen}, wherein geometries (Euclidean, hyperbolic, projective, etc.) are studied in terms of their symmetry groups and invariants thereof.
}
These symmetries form the Poincaré group, and consist of spacetime translations and Lorentz transformations, the latter being the extension of the rotation group for Euclidean space to relativistic rotations of spacetime.
The standard matrix representation of the Lorentz group, $\SO^+(1, 3)$, is the connected component of the orthogonal group
\begin{align}
	\op{O}(1,3) = \set{\lin Λ ∈ \GL(\RR^4) | \lin Λ\trans\lin η\lin Λ = \lin η}
\end{align}
with respect to the bilinear form $η = ±\op{diag}(-1,+1,+1,+1)$.
The rudimentary tools of matrix algebra are sufficient for an analysis the Lorentz group, and are familiar to any physicist.
However, they are not always the most suitable tool available for problems of relativity.

The last century has seen many other mathematical objects be applied to the study of generalised rotation groups such as $\SO^+(1,3)$ or the $\RR^3$ rotation group $\SO(3)$.
Among these tools is the \emph{geometric algebra}, invented\sidenote{
	Clifford algebra (an alias) was independently discovered by Rudolf Lipschitz two years later \cite{lipschitz1880clifford-alg}. 
	Lipschitz was the first to use them to the study the orthogonal groups.
} by William Clifford in 1878 \cite{clifford1878grassmann}, which constinute the main theme of this thesis.

Geometric algebra remains largely unknown in the physics community, despite arguably being far superior for algebraic descriptions of rotations than traditional matrix techniques.
To appreciate this, it is interesting to glean some of the history that led to this (perhaps unfortunate) circumstance.

\subsubsection{The quest for an optimal formalism for rotations}

Mathematics has seen the invention of a variety of vector formalisms since the 1800s, and the question of which is best suited to physics has a long and contentious history.
Complex numbers had been known for a long time\sidenote{
	Since Wessel, Argand and Gauss in the 1700s \cite{chappell2016quat-history}.
} to be useful descriptions of planar rotations.
William Hamilton's efforts to extend the same ideas into three dimensions by inventing a ``multiplication of triples'' bore fruition in 1843, when the quaternion algebra
\begin{align}
	\ii^2 = \jj^2 = \kk^2 = \ii\jj\kk = -1
\end{align}
came to him in revelation.
In following decades, William Gibbs developed the vector calculus of $\RR^3$ with the usual vector cross and dot products.
The ensuing vector algebra ``war'' of 1890--1945 saw Hamilton's prized\sidenote{
	Hamilton had dedicated following in the time that quaternions were in fasion: the \emph{Quaternion Society} existed from 1895 to 1913.
} quaternion algebra $\HH$, hailed as the optimal tool for describing $3$d rotations, struggle for popularity against Gibbs' easier-to-visualise vector calculus.
Gibbs' eventually won, and today quaternions are generally regarded as an old-fashioned mathematical curiosity.

Despite this, various authors, in appreciating the elegant handling of $\RR^3$ rotations, have tried coercing quaternions into Minkowski space $\RR^{1,3}$ for application to special relativity \cite{silberstein1912quat-sr,deleo1996quat-sr,dirac1944quat-sr}.
This has been done in various ways, but usually by complexifying $\HH$ into an eight-dimensional algebra $\CC\otimes\HH$ and then restricting the number of degrees of freedom as seen fit \cite{berry2020quat-sr,berry2021quat-sr}.
However, it is fair to say that quaternionic formulations of special relativity never gained notable traction.
Today, relativists are most familiar with tensor calculus, differential forms and the Dirac $γ$-matrix formalism, and have relatively little to do with quaternions or derived algebras.\sidenote{
	See \cite{chappell2016quat-history,altmann1989quat-history} for more historical discussion of quaternions and their adoption in physics.
} 

Arguably, this outcome of history is unfortunate: matrix descriptions of rotations cannot match the efficiency of quaternions, yet quaternions remain \emph{peculiar} and intrinsically tied to three dimensions.
The answer to this discontented call is the geometric algebra.
\todo{...}