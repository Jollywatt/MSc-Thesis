\chapter{Introduction}



The Special Theory of Relativity is a model of \emph{spacetime} --- the geometry in which physical events take place.
Spacetime comprises the Euclidean dimensions of space and time, but only in a way relative to each observer moving through it: there exists no single `universal' ruler or clock.
Instead, two observers in relative motion find their respective clocks and rulers are found to disagree, according to the Lorentz transformation laws.
The insight of special relativity is that one should focus not on the observer-dependent notions of space and time, but on the Lorentzian geometry of spacetime itself.

The study of local spacetime geometry amounts to the study of its intrinsic symmetries.\sidenote{
	This insight is part of Felix Klein's Erlangen programme of 1872 \cite{klein1893erlangen}, wherein geometries (Euclidean, hyperbolic, projective, etc.) are studied in terms of their symmetry groups and their invariants.
}
These consist of spacetime translations and Lorentz transformations, the latter including rotations in space and hyperbolic rotations in spacetime, or boosts.
% To the Poincaré group, consist of spacetime translations and Lorentz transformations, the latter being the extension of the rotation group for Euclidean space to relativistic rotations of spacetime.
The standard matrix representation of the Lorentz group, $\SO^+(1, 3)$, familiar to any relativist is the connected component of the orthogonal group
\begin{align}
	\op{O}(1,3) = \set{\lin Λ ∈ \GL(\RR^4) | \lin Λ\transpose\lin η\lin Λ = \lin η}
\end{align}
with respect to the bilinear form $η = ±\op{diag}(-1,+1,+1,+1)$.
The rudimentary tools of matrix algebra are sufficient for an analysis the Lorentz group, but are not always the most suitable tool available.

The last century has seen many other mathematical objects be applied to the study of generalised rotation groups such as $\SO^+(1,3)$ or the $\RR^3$ rotation group $\SO(3)$.
Among these tools is the \emph{geometric algebra}, invented\sidenote[][-4ex]{
	Clifford algebra (an alias) was independently discovered by Rudolf Lipschitz two years later \cite{lipschitz1880clifford-alg}. 
	Lipschitz was the first to use them to the study the orthogonal groups.
} by William Clifford in 1878 \cite{clifford1878grassmann} building upon the work of Hamilton and Grassmann, which constitute the main theme of this thesis.

Geometric algebra remains largely unknown in the physics community, despite arguably being far superior for algebraic descriptions of rotations than traditional matrix techniques.
To appreciate this, we ought to glean the history that led to the relative obscurity of Clifford algebras.


\subsubsection{The quest for an optimal formalism for rotations}

Mathematics has seen the invention of a variety of vector formalisms since the 1800s, and the question of which is best suited to physics has a long and contentious history.
Complex numbers had been long known\sidenote{
	Since Wessel, Argand and Gauss in the 1700s \cite{chappell2016quat-history}.
} to be useful descriptions of planar rotations.
William Hamilton's efforts to extend the same ideas into three dimensions by inventing a ``multiplication of triples'' bore fruition in 1843 when the quaternion algebra $\HH$, defined by
\begin{align}
	\ii^2 = \jj^2 = \kk^2 = \ii\jj\kk = -1
,\end{align}
famously came to him in revelation.
In following decades, William Gibbs developed the competing vector calculus of $\RR^3$ with the usual vector cross and dot products.
The ensuing vector algebra ``war'' of 1890--1945 saw Hamilton's prized\sidenote{
	Hamilton had a dedicated following in the time: the \emph{Quaternion Society} existed from 1895 to 1913.
} quaternion algebra pitted against Gibbs' easier\hyp to\hyp visualise vector calculus, with Gibbs' calculus eventually dominating because of their relatively easier learning curve.
Today, quaternions are generally regarded as an old-fashioned mathematical curiosity.

Despite this, various authors, in appreciating quaternions' elegant handling of $\RR^3$ rotations, have tried coercing them into Minkowski space $\RR^{1,3}$ for application to special relativity \cite{silberstein1912quat-sr,deleo1996quat-sr,dirac1944quat-sr}.
This has been done in various ways, usually by complexifying $\HH$ into an eight-dimensional algebra $\CC\otimes\HH$ and then restricting the number of degrees of freedom as seen fit \cite{berry2020quat-sr,berry2021quat-sr}.
However, it is fair to say that quaternionic formulations of special relativity never gained notable traction.

\subsubsection{The superior vector formalism for physics}

Today, relativists are most familiar with tensor calculus, differential forms and the Dirac $γ$-matrix formalism, and have relatively little to do with quaternions or derived algebras.\sidenote{
	See \cite{chappell2016quat-history,altmann1989quat-history} for more historical discussion of quaternions and their adoption in physics.
}
Arguably, this outcome of history is unfortunate: matrix descriptions of rotations cannot match the efficiency of quaternions, yet quaternions remain `peculiar' to many and are intrinsically tied to three dimensions.

In this respect, geometric algebra is a perfect middle-ground.
Its rotor formulation of rotations is algebraically efficient like the quaternions, but is not specific to $\RR^3$ --- indeed, geometric algebra is general to any dimension or metric signature.
Furthermore, objects like vectors, bivectors and $k$-vectors (familiar from exterior differential calculus) are first-class objects in the geometric algebra, yet obey identical rotor transformation laws.
Unlike exterior calculus, multivectors are often invertible, making algebraic manipulation easy.

In quantum theory, Dirac's $γ$-matrix formalism is simply a matrix representation of a geometric algebra (see \cref{sec:common-alg-isos}).
Although some physicists come away from quantum theory with the impression that Clifford algebra is something \emph{inherently quantum}, this is a misconception: geometric algebra is applicable to vast areas of geometry and physics, classical and quantum, and from elementary levels.\sidenote{
	See \cite{doran2003ga} for discussion of diverse applications of geometric algebra.
}


\subsubsection{Outline of this thesis}


\Cref{part:1} of this thesis introduces geometric algebra with emphasis on its relation to other common structures in physics.
The principal focus is then on its applications to special relativity, where Lorentz transformations are described as rotors in the geometric algebra.
In \cref{cha:bch}, this leads to a novel technique for composing Lorentz transformations in terms of rotor generators (also described in \cite{wilson2021ga-bch}).
% Later, the focus is on calculus and the extension of the ideas of \cref{part:1} to manifolds and general relativity in \cref{part:2}.

Seven years after Albert Einstein introduced this theory,\sidenotemark\ he succeeded in formulating a relativistic picture which included gravity.
In this General Theory of Relativity, gravitation is identified with the curvature of spacetime over astronomical distances.
\sidenotetext{
	Einstein’s paper \cite{einstein1905electrodynamics} was published in 1905, the so-called \emph{Annus Mirabilis} or ``miracle year'' during which he also published on the photoelectric effect, Brownian motion and the mass-energy equivalence.
	Each of the four papers was a monumental contribution to modern physics.
}
Both theories coincide locally (i.e., when confined to sufficiently small extents of spacetime, over which the effects of curvature are negligible).
In \cref{part:2}, we extend the ideas of \cref{part:1} to curved manifolds, and investigate some applications of the geometric algebra formalism in differential geometry.
 % present an interesting non-Abelian Stokes' theorem for relating a manifold's curvature across a surface to parallel transport around the surfaces boundary.