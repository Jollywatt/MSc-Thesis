\chapter{Calculus in Flat Space(time)}

\todo{
	\begin{enumerate}
		\item Exterior derivative and vector derivative
		\item Stokes' theorem and the GA equivalent
		\item Maxwell's equations
	\end{enumerate}
}


\section{Integration}


\subsection{Exterior Calculus}

\begin{theorem}[Stokes' theorem in $\RR^n$]
	\label{thm:flat-stokes}
	If $R ⊆ \RR^n$ is a compact $k$-dimensional hypersurface with boundary $∂R$, then a smooth differential form $ω ∈ \forms[k - 1](R)$ satisfies
	\begin{align}
		\label{eqn:stokes}
		\int_R \dd ω = \int_{∂R} ω
	.\end{align}
\end{theorem}
\begin{proof}
	Since $R$ is a $k$-dimensional region with boundary, every point $x ∈ R$ has a neighbourhood diffeomorphic to a neighbourhood of the origin in either $\RR^k$ or $H^k ≔ [0, ∞) ⊕ \RR^{k - 1}$, depending on whether $x$ is an interior point or a boundary point, respectively.
	Let $\set{U_i}$ be a cover of $R$ consisting of such neighbourhoods.
	Since $R$ is compact, we may assume $\bigcup_{i=1}^N \set{U_i} = R$ to be a finite covering.
	Thus, we have finitely maps $h_i : U_i → X$ where $X$ is either $\RR^k$ or the half-space $H^k$, where $U_i \cong h_i(U_i)$ are diffeomorphic.

	Finally, let $\set{ϕ_i : R → [0, 1]}$ be a partition of unity subordinate to $\set{U_i}$, so that $\set{x ∈ R | ϕ_i(x) > 0} ⊆ U_i$ and $ω = \sum_{i=1}^N ϕ_iω$.
	We need only prove the equality \eqref{eqn:stokes} for each $ω_i ≔ ϕ_iω$, and the full result follows be linearity.
	
	With respect to the canonical coordinates of $X$, the $(k - 1)$-form $h_i^*ω_i$ can be written
	\begin{align}
		h_i^*ω_i = \sum_{j=1}^k f_j (-1)^{j - 1} \dd x^{1\cdots\hat{j}\cdots k}
	\end{align}
	using the multi-index notation $\dx^{\etc{i_\i}{}k} ≡ \etc{\dx^{i_\i}}∧k$, where the hat denotes an omitted term.
	The factor of $(-1)^{j - 1}$ gives the $(k - 1)$-form the boundary orientation induced by the volume form $\dx^{\etc\i{}k}$ for convenience.
	Since pullbacks commute with $\dd$,
	\begin{align}
		h^*\ddω_i = \dd(h_i^*ω_i) = \sum_{j=1}^k \pdv{f_j}{x^j} \dd x^{1\cdots n}
	.\end{align}
	There are then two cases to consider.
	\begin{itemize}
		\item \emph{Interior case.}
		If $h_i : U_i → \RR^k$, then the right-hand side of \cref{eqn:stokes} vanishes because $ω_i$ is zero outside the neighbourhood $U_i ⊂ R$ which nowhere meets the boundary $∂R$.
		\begin{align}
			\int_{∂R} ω_i = \int_{∂U_i} ω_i = \int_∅ ω_i = 0
		\end{align}
		The left-hand side evaluates to
		\begin{align}
			\int_R \dd ω_i
			&= \int_X \dd (h_i^*ω_i)
			= \int_{\RR^k} \sum_{j=1}^k \pdv{f_j}{x^j} \dd x^{1\cdots n}
		\\	&= \underbrace{\etc{\int_{-∞}^{+∞}}{}{}}_k \sum_{j=1}^k \pdv{f_j}{x^j} \etc{dx^\i}{}{k}
		\\	&= \underbrace{\etc{\int_{-∞}^{+∞}}{}{}}_{k - 1} \sum_{j=1}^k \eval{f_j}_{x^j=-∞}^{+∞} (-1)^{j-1} dx^1\cdots\widehat{dx^j}\cdots dx^k
			= 0
		,\end{align}
		which vanishes because $h_i^*ω_i$, and hence the $f_j$, vanish outside the neighbourhood $h_i(U_i) ⊂ \RR^k$.

	\item \emph{Boundary case.}
	If $h_i : U_i → H^k$, then the boundary $∂U_i ⊂ ∂R$ is mapped onto the hyperplane $∂H^k = \set{(0, \etc[2]{x^\i},k) | x^j ∈ \RR}$.
	Thus, $dx^1 = 0$ on this boundary, and the right-hand side of \cref{eqn:stokes} becomes
	\begin{align}
		\int_{∂R}ω_i
		&= \int_{∂U_i} h_i^*ω_i
		= -\int_{\RR^{k - 1}} f_1 \etc[2]{dx^\i}{}k
	\\	&= -\underbrace{\etc{\int_{-∞}^{+∞}}{}{}}_{k - 1} f_1(0, \etc[2]{x^\i},k) \etc[2]{dx^\i}{}k
	.\end{align}
	The factor of $-1$ comes from the induced orientation of the boundary $∂H^k$, which is outward-facing, so in the \emph{negative} $x^1$ direction.
	For the left-hand side of \cref{eqn:stokes},
	\begin{align}
		\int_R \dd ω_i
		&= \int_{H^k} h_i^*\ddω_i
		= \int_0^∞ \etc{\int_{-∞}^{+∞}}{}{} \sum_{j=1}^k \pdv{f_j}{x^j} \etc{dx^\i}{}k
	\intertext{
		All terms $\pdv{f_j}{x^j}dx^j$ in the sum for $j > 1$ integrate to boundary terms $x_j → ±∞$ where $f_j$ vanishes.
		This leaves the single term from the integration of $dx^1$,
	}
		&= -\etc{\int_{-∞}^{+∞}}{}{} \eval{f_1}_{x^1=0}^∞ \etc[2]{dx^\i}k
	.\end{align}

	\end{itemize}

	Thus, we have equality for all $ω_i$, so
	\begin{align}
		\int_R \dd ω = \sum_{i=1}^N \int_R \dd ω_i = \sum_{i=1}^N \int_{∂R} ω_i = \int_{∂R} ω
	\end{align}
	by linearity.
\end{proof}

