\layout

\setcounter{chapter}{-1}
\chapter{Proof of Document Features}

\section{Referencing}

\subsection{Automatic equation labels}
Unlabelled, unreferenced:
\begin{equation}
	a^2 = π
\end{equation}
Labelled, unreferenced:
\begin{equation}
	\label{eqn:1}
	b^2 = ρ
\end{equation}
Labelled, referenced:
\begin{equation}
	\label{eqn:2}
	c^2 = σ
\end{equation}
See \cref{eqn:2}.

\subsection{Reference naming}
Suppose
\begin{equation}
	\label{eqn:3}
	d^2 = η
.\end{equation}
\Cref{eqn:3} proves.
\begin{theorem}[Diogenes]
	\label{thm:1}
	Something.
\end{theorem}
By \cref{thm:1}, something. \Cref{thm:1} states something.
\begin{definition}
	\label{def:1}
	Deduction.
\end{definition}
See \cref{def:1}. \Cref{def:1} defines.
\begin{lemma}
	\label{lem:1}
	Little.
\end{lemma}
A small result is \cref{lem:1}. \Cref{lem:1} is nice.


\section{Side margins}

\lipsum[1][1-6]\sidenote{
	This is a long sidenote on two lines.
}
\lipsum[2][1-3]\sidenote{
	\lipsum[4][1-4]
}
\lipsum[3][1]\sidenote{
	Does this fit?
}
\lipsum[3][2-4]

\section{Links and citations}

This is a \url{http://url.com}.
A like to think this will turn out OK \cite{misner1973gravitation}.
All my inspiration is from \cite{gallian2021abstract-algebra,spivak1975dg,lee2012diffgeo}.


\section{Mathematical macros}

Set builders:
\begin{align}
	\set{}, \set{1}, \set{1, 9\frac34}, \set{x^2 | x ∈ \RR}
\end{align}
Custom sizing:
\begin{align}
	\set[\bigg]{1}, \set[]{\int}
\end{align}
Misc.
\begin{align}
	\grade[p]{A + B}, \TA{(\TT^*ℳ)}, \EA[k]{V}, \SA{\RR^n}, \GA[2](V, η)
\end{align}
